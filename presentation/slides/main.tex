%%%%%%%%%%%%%%%%%%%%%%%%%%%%%%%%%%%%%%%%%
% Beamer Presentation
% LaTeX Template
% Version 1.0 (10/11/12)
%
% This template has been downloaded from:
% http://www.LaTeXTemplates.com
%
% License:
% CC BY-NC-SA 3.0 (http://creativecommons.org/licenses/by-nc-sa/3.0/)
%
%%%%%%%%%%%%%%%%%%%%%%%%%%%%%%%%%%%%%%%%%

%----------------------------------------------------------------------------------------
%	PACKAGES AND THEMES
%----------------------------------------------------------------------------------------

\mode<presentation> {

% The Beamer class comes with a number of default slide themes
% which change the colors and layouts of slides. Below this is a list
% of all the themes, uncomment each in turn to see what they look like.

%\usetheme{default}
%\usetheme{AnnArbor}
%\usetheme{Antibes}
%\usetheme{Bergen}
%\usetheme{Berkeley}
%\usetheme{Berlin}
%\usetheme{Boadilla}
%\usetheme{CambridgeUS}
%\usetheme{Copenhagen}
%\usetheme{Darmstadt}
%\usetheme{Dresden}
%\usetheme{Frankfurt}
%\usetheme{Goettingen}
%\usetheme{Hannover}
%\usetheme{Ilmenau}
%\usetheme{JuanLesPins}
%\usetheme{Luebeck}
%\usetheme{Madrid}
%\usetheme{Malmoe}
%\usetheme{Marburg}
%\usetheme{Montpellier}
%\usetheme{PaloAlto}
%\usetheme{Pittsburgh}
%\usetheme{Rochester}
%\usetheme{Singapore}
%\usetheme{Szeged}
%\usetheme{Warsaw}

% As well as themes, the Beamer class has a number of color themes
% for any slide theme. Uncomment each of these in turn to see how it
% changes the colors of your current slide theme.

%\usecolortheme{albatross}
%\usecolortheme{beaver}
%\usecolortheme{beetle}
%\usecolortheme{crane}
%\usecolortheme{dolphin}
%\usecolortheme{dove}
%\usecolortheme{fly}
%\usecolortheme{lily}
%\usecolortheme{orchid}
%\usecolortheme{rose}
%\usecolortheme{seagull}
%\usecolortheme{seahorse}
%\usecolortheme{whale}
%\usecolortheme{wolverine}

% To remove the footer line in all slides uncomment this line
%\setbeamertemplate{footline}
% To replace the footer line in all slides with a simple slide count uncomment
% this line
\setbeamertemplate{footline}[page number]
% To remove the navigation symbols from the bottom of all slides uncomment
% this line
\setbeamertemplate{navigation symbols}{}
}

% Beamer font style
\setbeamerfont{title}{size=\huge}
\setbeamerfont{frametitle}{size=\LARGE}
\setbeamerfont{itemize/enumerate body}{size=\large}
\setbeamerfont{normal text}{size=\large}

\makeatother{}

% Input encoding
\usepackage[utf8]{inputenc}
% Fonts and symbols
\usepackage{amsfonts, amsmath, amssymb, amsthm}
\usepackage{eurosym}
% Image
\usepackage{graphicx}
% Colors
\usepackage{color}
% Allows the use of \toprule, \midrule and \bottomrule in tables
\usepackage{booktabs}
% ITE package (for conditional compilation)
\usepackage{ifthen}

% Listings (code)
\usepackage{listings}
\usepackage{courier}
\lstset{basicstyle=\ttfamily\footnotesize
	,tabsize=2
	%,frame=single
	%,numbers=left
	,numberstyle=\tiny
	,numbersep=5pt
	,numberfirstline=true
	,firstnumber=1
	,stepnumber=5
	,showspaces=false
	,showstringspaces=false
	,showtabs=false
	,breaklines=false
	,breakatwhitespace=true
}
\lstdefinestyle{highlight}{
	keywordstyle=\color{red},
	commentstyle=\color{green},
}
\lstdefinestyle{beamer}{
 	%preset=\large,
    breaklines=true,
	basicstyle=\ttfamily\footnotesize\color{black!50},
	keywordstyle=\color{red!50},
	commentstyle=\color{green!50},
	moredelim=**[is][{\color{black}\lstset{style=highlight}}]{@}{@},
}

% Macro to include images
\newcommand{\includeimage}[2][scale=1.0]{
	\begin{center}
	\includegraphics[keepaspectratio=true,#1]{img/#2}
	\end{center}
}

% Macro: separation between bad and good examples in attacks
\newcommand{\badgoodsep}{
\vspace{6mm}
\pause
\hrule
\vspace{6mm}
}

% Macro for good and bad code
\newenvironment{embeddedcode}[1]{
\begin{columns}[T]
\begin{column}{3mm}
\includeimage[width=5mm]{code-#1}
\end{column}
\begin{column}{0.9\linewidth}
}{
\end{column}
\end{columns}
}

% Begin each section with a dedicated frame
\AtBeginSection[]{
	\begin{frame}
	%\sectionpage
	\usebeamercolor[fg]{title}
	\usebeamerfont{section title}
	\centering
	\insertsection
	\note{\Large New section: \emph{\insertsection}}
	\end{frame}
}


%-----------------------------------------------------------------------------
%	TITLE PAGE
%-----------------------------------------------------------------------------

\title[Web Security]{Security in Web Applications}
\author[]{Fran\c{c}ois Roland
\and Krzysztof Magusiak
\and Vincent Gribomont
\and Micha\"{e}l Lacroix
}
\institute[Hermes ECS]{Hermes Engineering Consultancy Services}
\date{\today}

\begin{document}

%-----------------------------------------------------------------------------

\begin{frame}
\author{}
\institute{}
\date{}
\titlepage
% Insert the logo
\begin{picture}(0,0)
\put(160,0){\includegraphics[keepaspectratio,width=5cm]{img/hermes}}
\end{picture}
\end{frame}

%\begin{frame}[allowframebreaks]
%\frametitle{Overview}
%\setcounter{tocdepth}{2}
%\tableofcontents
%\end{frame}
% For the rest of the document set depth to 3
\setcounter{tocdepth}{3}

%-----------------------------------------------------------------------------
%	PRESENTATION SLIDES
%-----------------------------------------------------------------------------




\begin{frame}
\begin{columns}
\begin{column}{0.65\linewidth}
{\Large François Roland}
\vspace{0.5cm}\par
Hermès Engineering consultant \\
Software architect at bpost
\\[2\baselineskip]
francois.roland@hermes-ecs.com \\
@FrRoland
\end{column}
\begin{column}{0.35\linewidth}
\includeimage[width=0.6\linewidth]{photo-francois-roland}
\end{column}
\end{columns}
\note{
	Hello, my name is François Roland. I graduated as a civil engineer in computer sciences and management in 2003.
	I've worked for 10 years in the domain of web applications, 4 years as an employee and 6 as a consultant for Hermès Engineering.
	I'm currently working at bpost as software architect.
}
\end{frame}

\begin{frame}
\frametitle{Hermès Engineering}
\begin{itemize}
	\item IT consultancy services
	\item About 80 IT specialists
	\item Belgium and Luxembourg
\end{itemize}
\note{
	Hermès Engineering is an IT consultancy services company. We are about 80 IT specialists working in Bergium and Luxembourg.
}
\end{frame}

\begin{frame}
\frametitle{Hermès Engineering}
\Huge business intelligence
\\[2\baselineskip]
\hfill operational IT

\note{
	Our main domains are business intelligence and operational IT. Operational IT is about managing our customer core businesses. Business intelligence is about gathering and transforming operational data into reports and dashboards to help them take better decisions.
}
\end{frame}

\begin{frame}
\includeimage[width=0.8\linewidth,height=0.8\textheight]{various-environments}


\note{
	In the software engineering world, web applications share some specifics:
	\begin{itemize}
		\item make use of many elements
		\item some of these elements are beyond the control of the developer
	\end{itemize}
}
\end{frame}

\begin{frame}
\begin{columns}
\begin{column}{0.5\linewidth}
systems
\includeimage[width=0.5\linewidth]{icon-server.png}
\end{column}
\begin{column}{0.5\linewidth}
cryptography
\pause\includeimage[width=\linewidth]{https.jpg}
\end{column}
\end{columns}
\note{
	The objective of this presentation is to show that the security concern is not only about systems and cryptography.
}
\end{frame}

\begin{frame}
\begin{columns}
\begin{column}{0.5\linewidth}
\includeimage[width=\linewidth]{developer.jpg}
\end{column}
\begin{column}{0.5\linewidth}
\includeimage[width=\linewidth]{code.png}
\end{column}
\end{columns}
\note{
	Security concerns the developer when he writes his code\ldots
}
\end{frame}

\begin{frame}
\begin{columns}
\begin{column}{0.5\linewidth}
  \includeimage[width=\linewidth]{architect.jpg}
\end{column}
\begin{column}{0.5\linewidth}
\includeimage[width=\linewidth]{blueprint.jpg}
\end{column}
\end{columns}
\note{
	the architects when he designs his complete solution\ldots
}
\end{frame}

\begin{frame}
\begin{columns}
\begin{column}{0.5\linewidth}
\includeimage[width=\linewidth]{business_analyst.jpg}
\end{column}
\begin{column}{0.5\linewidth}
\begin{center}
\includeimage[width=0.5\linewidth]{easy}

\includeimage[width=0.2\linewidth]{vs}

\includeimage[width=\linewidth]{secure-website}
\end{center}
\end{column}
\end{columns}
\note{
	but also the analyst who has to choose the right trade-off between usability and security\ldots
}
\end{frame}

\begin{frame}
\begin{columns}
\begin{column}{0.45\linewidth}
\includeimage[width=\linewidth]{project-manager}
\end{column}
\begin{column}{0.55\linewidth}
\includeimage[width=\linewidth]{Triangle}
\end{column}
\end{columns}
\note{
	or even the project manager who has to think about it while keeping the balance between cost, time and quality.
}
\end{frame}

\begin{frame}
\begin{columns}
\begin{column}{0.3\linewidth}
\includeimage[width=\linewidth]{identify}
\end{column}
\begin{column}{0.3\linewidth}
\includeimage[width=0.9\linewidth]{investigation.jpg}
\end{column}
\begin{column}{0.3\linewidth}
\includeimage[width=\linewidth]{diagnose}
\end{column}
\end{columns}
\note{
	Moreover, it is also important to be able to identify an eventual attack and to diagnose it afterwards. Both of theses capabilities require money, competency but above all it requires preparation.
}
\end{frame}

\begin{frame}
\frametitle{Agenda}
\begin{itemize}
	\item Psycho-sociological Aspect
	\item Authentication and Identity
	\item Attacks
	\item Prevention
	\item Detection and Analysis
\end{itemize}
\end{frame}

\begin{frame}
\begin{columns}
\begin{column}{0.3\linewidth}
\includeimage[width=\linewidth]{photo-krzysztof-magusiak}
Krzysztof Magusiak
\end{column}
\begin{column}{0.3\linewidth}
\includeimage[width=\linewidth]{photo-michael-lacroix}
Michaël Lacroix
\end{column}
\begin{column}{0.3\linewidth}
\includeimage[width=\linewidth]{photo-vincent-gribomont}
Vincent Gribomont
\end{column}
\end{columns}
\note{
	At Hermès we value quality. That's why we set up a small team to create this seminar. Today I've come with \ldots and \ldots, two of the men that did the hard work behind this presentation.
	I leave you now with \ldots that will begin this presentation with the psycho-sociological aspect of the security.
}
\end{frame}


\section{Psycho-sociological Aspect}

\subsection{Security concerns}

\begin{frame}
\frametitle{Internet Security History}
Army, Universities, ...
\pause\par
\emph{Security?}
\pause
\includeimage[width=\linewidth,height=0.5\textheight]{mainframe}
\note{
The web:
\begin{itemize}
\item Designed in a \emph{trusted environment}
\item Became global
\end{itemize}
Security was not a concern in a closed environment.
}
\end{frame}

\begin{frame}
\frametitle{Nowadays...}
\begin{columns}
\begin{column}{0.3\linewidth}
\includeimage[width=\linewidth]{logo-facebook}
\end{column}
\begin{column}{0.4\linewidth}
\includeimage[width=\linewidth]{visa}
\end{column}
\begin{column}{0.3\linewidth}
\includeimage[width=\linewidth]{logo-tax-on-web}
\end{column}
\end{columns}
\begin{center}
{\Huge It's \textbf{everywhere}!}
\end{center}
\note{
\textbf{Web services} \\
Security concerns various aspects of our internet life.
\begin{itemize}
\item Personal (e-mail, Facebook, ...)
\item Financial (online banking, Tax-On-Web)
\item Professional (e-mail, calendars, ...)
\item etc.
\end{itemize}
}
\end{frame}

\subsection{Sentiment of security}

\begin{frame}
\begin{columns}
\begin{column}{0.45\linewidth}
\includeimage[width=0.9\linewidth]{https}
\end{column}
\begin{column}{0.1\linewidth}
{\centering\Huge $\nRightarrow$}
\end{column}
\begin{column}{0.45\linewidth}
\includeimage[width=0.9\linewidth]{secure-website}
\end{column}
\end{columns}
\note{
\textbf{False sentiment of security} \\
Gain some level of trust...
\\ but, do not lure the users in a false sentiment of security.
\begin{exampleblock}{HTTPS}
Users may think that an HTTPS web site is secured.
\\ HTTPS prevents only some of the attacks.
\end{exampleblock}
}
\end{frame}

\begin{frame}
\begin{columns}
\begin{column}{0.3\linewidth}
\includeimage[width=\linewidth]{logo-psn}
\end{column}
\begin{column}{0.6\linewidth}
\begin{exampleblock}{PlayStation Network outage}
77 million accounts affected.
\end{exampleblock}
\end{column}
\end{columns}
\note{
A web site must earn the confidence of its users.
\begin{block}{Why?}
Users will send their data to the server.
\end{block}
\begin{block}{However...}
The web site could be compromised.
\end{block}
\begin{exampleblock}{PlayStation Network outage}
Information from approximately 77 million accounts was stolen.
\end{exampleblock}
}
\end{frame}

\subsection{Web payment}

\begin{frame}
\includeimage[width=0.8\linewidth]{secure-payment-logos}
\begin{center}
\begin{minipage}{0.5\linewidth} % to get the list centered
\begin{itemize}
\item Security measures
\item Guarantees
\item Protection
\end{itemize}
\end{minipage}
\end{center}
\note{
\begin{center}
You don't trust everyone with your money.
\end{center}
\begin{itemize}
\item Proper security measures
\item Guarantees
\item Protection against fraud
\end{itemize}
\begin{example}
Visa, MasterCard, PayPal, etc.
\end{example}
}
\end{frame}

\begin{frame}
\begin{columns}
\begin{column}{0.4\linewidth}
\Large
\centering
\emph{password}
\end{column}
\begin{column}{0.2\linewidth}
\includeimage[height=8mm]{vs}
\end{column}
\begin{column}{0.4\linewidth}
\includeimage[width=\linewidth,height=3cm]{pen}
\end{column}
\pause
\end{columns}
\begin{exampleblock}{Infosecurity Europe 2003 survey}
\begin{quote}
\small{In a 2003 information security survey, {\Large 90\%} of office workers gave
researchers what they claimed was their {\Large password} in answer to a
survey question in exchange for a {\Large cheap pen}.}
\end{quote}
\end{exampleblock}
\note{
	\begin{exampleblock}{What do you trade your password for?}
	\begin{quote}
	90\% of people gave away their passwords in exchange for a pen.
	--~Infosecurity~Europe~2003
	\end{quote}
	\end{exampleblock}
}
\end{frame}


\subsection{Users as a security flaw}

\begin{frame}
\frametitle{Security}
\includeimage[height=0.75\textheight,width=\linewidth]{xkcd538_security-1}
\note{
How can we hack that security?
}
\end{frame}

\begin{frame}
\frametitle{Security}
\includeimage[height=0.75\textheight,width=\linewidth]{xkcd538_security-2}
\note{
\begin{itemize}
\item Users can divulge their credentials
\item They may use unsecured web sites, compromised software, ...
\item Users don't always understand what they do
\item "Security is the IT job"
\item etc...
\end{itemize}
\begin{block}{Social engineering}
\begin{quote}
We can pay him on some social network to collect information from him
and eventually steal his identity.
\end{quote}
\end{block}
}
\end{frame}

\begin{frame}
\begin{columns}
\begin{column}{0.4\linewidth}
\begin{scriptsize}
\fontsize{12pt}{20}\selectfont
\begin{enumerate}
\item password
\item 123546
\item 12345678
\item abc123
\item qwerty
\item monkey
\item letmein
\item dragon
\item 111111
\item baseball
\item ...
\end{enumerate}
\end{scriptsize}
\end{column}
\begin{column}{0.6\linewidth}
\includeimage[height=0.6\textheight,width=\linewidth]{most-common-password-list}
\end{column}
\end{columns}
\note{
	Remembering multiple passwords is... \emph{difficult}.
	\begin{itemize}
	\item How many different passwords do you have? More than 5?
	\item Is it better to have a larger alphabet or a longer password?
	\end{itemize}
	Most used passwords source:
	\url{http://www.cbsnews.com/8301-205_162-57539366/the-25-most-common-passwords-of-2012/}
}
\end{frame}


\begin{frame}
\includeimage[width=2cm]{code-bad}
\includeimage[scale=0.6]{password-post-its}
\note{
Don't do that...
}
\end{frame}

\begin{frame}
\includeimage[width=2cm]{code-good}
\fontsize{16pt}{20}\selectfont
\begin{itemize}
\item 1password \\ \url{https://agilebits.com/onepassword}
\item lastpass \\ \url{https://lastpass.com}
\item pwdhash \\ \url{https://www.pwdhash.com}
\end{itemize}

\note{
\textbf{Bad password storage}
\par
Do this instead.
\par
Online services for storing passwords:
\begin{itemize}
\item \href{https://agilebits.com/onepassword}{1password}
\item \href{https://lastpass.com}{lastpass}
\item \href{https://www.pwdhash.com}{pwdhash}
\end{itemize}
}
\end{frame}

\subsection{Trade-off: security vs usability}
\begin{frame}
\frametitle{Security versus Usability}
\begin{center}
\begin{tabular}{rcl}
\pause
secure password & \includegraphics[width=5mm]{img/vs} & simple password
\\[2em]
\pause
authorization checks & \includegraphics[width=5mm]{img/vs} & no authentication
\\[2em]
\pause
encoded data & \includegraphics[width=5mm]{img/vs} & plain text
\end{tabular}
\end{center}
\note{
\begin{itemize}
\item The system should be easy to use
\item Security measures are in the way of what the user wants to achieve
\end{itemize}
Security measures:
\begin{itemize}
\item Too many: make a web site unusable
\item Too few: the web site can be compromised
\end{itemize}
}
\end{frame}



\section{Authentication and Identity}

\begin{frame}
\frametitle{Authentication vs Identity}
\begin{block}{Identity}
The identity of somebody/something is who/what he/it is.
\end{block}
\begin{block}{Authentication} Authentication is the process of
  verification that an individual or an entity is who it claims to
  be \textit{(OWASP)}. 
\end{block}
\end{frame}

%------------------------------------------------

\begin{frame}
\frametitle{Why authentication?}

Authentication is needed when you want to transmit confidential
information, you want to be sure that your correspondant isn't
impersonated by somebody else.

Authentication by itself doesn't mean confidentiality as it doesn't
prevent eavesdropping.

\end{frame}


%------------------------------------------------
\subsection{Different means of authentication}

%------------------------------------------------
\begin{frame}
\frametitle{Passwords}

\begin{itemize}
\item The most common form of authentication
\item Password should be easily changeable
\item Peoples should have different passwords for different services
\item Complexity must be sufficient
\end{itemize}
\end{frame}


%------------------------------------------------

\begin{frame}
\frametitle{Password complexity}

Enforcing a minimum password strength:
\begin{itemize}
\item State the rules clearly(e.g. minimum 10 character, minimum a
  capital letter, \ldots)
\item Check the complexity in the browser to prevent him to submit a
  form with an ``invalid'' password
\item Check again on the server side
\item If the password doesn't comply, give all the violated rules in
  the error message.
\end{itemize}

\begin{figure}
  \includegraphics[width= 0.3\linewidth]{img/browserPasswordValidation.jpg}
\end{figure}

\end{frame}

%------------------------------------------------


\begin{frame}
\frametitle{One time passwords}

One time password (OTP for short) are passwords which are valid only
once. 
\begin{itemize}
\item Prevent replay attack.
\item Main difficulty: giving the user his passwords. 
\item Often cause logistical problems 
\item Used mainly by those who can afford big infrastructure (States,
  banks, other big companies, \ldots)
\end{itemize}
\end{frame}

%------------------------------------------------

\begin{frame}
\frametitle{One time passwords: choosing and distribution}

To avoid prediction, passwords must be chosen using a randow or at
least pseudorandom way.

There are various ways to distribute OTPs:
\begin{itemize}
\item If OTP generation is time based. Password are only valid for a
  short time. The generation can be done by a small electronic
  device which can be carried by the user.
\begin{figure}
  \includegraphics[width= 0.3\linewidth]{img/SecureID_token_new.JPG}
\end{figure}
\item You can send them out-of-band to the user (by SMS, mail, \ldots)
\item You can use a software on mobile phone that generate the passwords
\end{itemize}
\end{frame}

%------------------------------------------------

\begin{frame}
\frametitle{Certificates}
A trusted authority (Certification Authority) issues certificates to
confirm the ID of something. Those certificates may be of varying
quality and are most often used in SSL/TLS by web browsers.
\begin{figure}
  \includegraphics[width= 0.8\linewidth]{img/certificatesVariety.png}
\end{figure}
\end{frame}

%------------------------------------------------

\begin{frame}[fragile]
\frametitle{Certificates: What are they made of?}
Classical web certificates are using X.509 v3 standard.

\small
\begin{verbatim}

Certificate:
   Data:
       Version: 1 (0x0)
       Serial Number: 7829 (0x1e95)
       Signature Algorithm: md5WithRSAEncryption
       Issuer: C=ZA, ST=Western Cape,...
               OU=Certification Services Division,
               CN=Thawte Server CA/emailAddress=...
       Validity   
           Not Before: Jul  9 16:04:02 1998 GMT
           Not After : Jul  9 16:04:02 1999 GMT
\end{verbatim}

\end{frame}

%------------------------------------------------


\begin{frame}[fragile]

\small
\begin{verbatim}
       Subject: C=US, ST=Maryland,...
                OU=FreeSoft, CN=www.freesoft.org/emailAddress=...
       Subject Public Key Info:
           Public Key Algorithm: rsaEncryption
           RSA Public Key: (1024 bit)
               Modulus (1024 bit):
                   00:be ...:41:8f
               Exponent: 65537 (0x10001)
   Signature Algorithm: md5WithRSAEncryption
       93:5f...:68:9f
\end{verbatim}

\end{frame}

%------------------------------------------------


\begin{frame}
\frametitle{Certificates: How are they used?}
Certificates are often used as a mean to distribute key in a public
key infrastructure. A typical exemple is for SSL/TLS in web browsers.
\begin{figure}
  \includegraphics[width= 0.9\linewidth]{img/Usage-of-Digital-CertificatePart1.png}
\end{figure}

\end{frame}

%------------------------------------------------

\begin{frame}

\begin{figure}
  \includegraphics[width= 0.9\linewidth]{img/Usage-of-Digital-CertificatePart2.png}
\end{figure}

\end{frame}

%------------------------------------------------

\begin{frame}

\begin{figure}
  \includegraphics[width= 0.9\linewidth]{img/Usage-of-Digital-CertificatePart3.png}
\end{figure}
\textit{Wikipedia}
\end{frame}

%------------------------------------------------

\begin{frame}
\frametitle{Certificates: Two way authentication}

\begin{itemize}
\item In the previous scenario we have shown that the browser autenticates
the server but the opposite isn't true.

\item It's possible for both participant to authenticate \ldots

\item \ldots but it's more complicated: two good one way
  authentication don't make a valid two way authentication
\end{itemize}

\end{frame} 

%------------------------------------------------

\begin{frame}
\frametitle{Tokens}

\begin{block}{Token} Token are something which ownership gives a form
  of authentication.
\end{block}
\begin{itemize}
\item In our everyday life, token can take various forms: keys, bank card,
badges, \ldots
\item In electronic authentication schemes they are most often smart
  card or digipass
\end{itemize}
\end{frame}

%------------------------------------------------

\begin{frame}
\frametitle{Multi factor authentication}
 An application should use at least two factors of
 authentication. Those factors can be:
\begin{itemize}
\item what the authenticated entity is (biometrics, \ldots),
\item what it owns (a bank card, \ldots) 
\item or what he knows (a password, a pin code, \ldots)
\end{itemize}
\end{frame}

%------------------------------------------------

\subsection{Signature challenge}

%------------------------------------------------
\begin{frame}
\frametitle{Signature challenge: principle}

\begin{itemize}
\item An authentication based upon ``answering'' a question about a secret
known by the participants. For example, when I encrypt something (a nonce)
with my correspondent's public key and ask him for the decrypted and encrypted
message.

\item This form of authentication is often used to sign online
  transaction.
\end{itemize}

\end{frame}
%------------------------------------------------

\subsection{Single sign-on}

%------------------------------------------------
\begin{frame}
\frametitle{Single sign-on: principle}

\begin{block}{Single sign-on}
Single sign-on (SSO) is a property of access control of multiple
related, but independent software systems. With this property a user
logs in once and gains access to all systems without being prompted to
log in again at each of them. \textit{Wikipedia}
\end{block}

\end{frame}
%------------------------------------------------

\begin{frame}
\frametitle{Single sign-on: pro and cons}

\begin{block}{Pro}
\begin{itemize}
\item Easier for the user
\item Not trivial to build a secure authentication scheme
\item If they have only one password users tend to treat it with more care
\item Enter the password less often
\end{itemize}
\end{block}

\begin{block}{Cons}
\begin{itemize}
\item All your eggs in the same basket (impact if compromised)
\item You are dependant upon your provider (confidence, availability, \ldots)
\end{itemize}
\end{block}

\end{frame}

%------------------------------------------------

\subsection{Kerberos}

%------------------------------------------------

\begin{frame}
\frametitle{Kerberos}

Kerberos is a SSO developped at MIT to solve the problem of allowing
some users to use restricted ressources. MIT provide a free
implementation of the protocol but it's also found in many commercial
products.

\end{frame}

%------------------------------------------------

\begin{frame}
\frametitle{Kerberos: How it works}

\begin{figure}
  \includegraphics[scale = 0.35]{img/kerberos.png}
\end{figure}
\textit{Wikipedia}
\end{frame}

%------------------------------------------------
\subsection{Provider extensions}
%------------------------------------------------

\begin{frame}
\frametitle{Provider extensions}
\begin{itemize}
\item Single sign-on is provided for free by organisation like Facebook, OpenID and
Google to other websites for free.
\item The usage of those services is usually much easier for the
  developper than developping his own sign-on mechanism.
\item But it ties you to this provider so some sites use services from
  different organisations
\end{itemize}

\end{frame}

%------------------------------------------------
\subsection{Compromised authentication mean}
%------------------------------------------------

\begin{frame}
\frametitle{Compromised authentication mean}
 When you implement authentication you have to take into account that
 the user may loose its authentication mean or it can get stolen. So,
 you should be able to:

\begin{itemize}
\item Deactivate a compromised authentication mean,
\item Authenticate the user by an uncompromised mean,
\item Give him a new primary authentication mean.
\end{itemize}

\end{frame}

%------------------------------------------------

\begin{frame}
\frametitle{Example: Lost password}
 If one of your users loose his password, he has to get a new one but
 doing this in a secure way can be challenging. Here is a procedure
 that you can follow.

\begin{itemize}
\item use some predefined security questions,
\item send a token over a side channel,
\item allow the user to change password,
\item confirm change
\end{itemize}

\end{frame}
%------------------------------------------------


\section{Attacks}

%%%%%%%%%%%%%%%%%%%%%%%%%%%%%%%%%%%%%%%%%%%%%%%%%%%%%%%%%%%%%%%%%%%%%%%%%%%%%%
\subsection{Attack trees}

\begin{frame}
\frametitle{An attack tree}

\begin{block}{What is it}
Model describing by what means an \emph{asset} can be attacked.
\end{block}
\begin{block}{Goal}
Used to determine and understand threats that may arise.
\end{block}
\end{frame}

\begin{frame}{Example}
\includeimage[width=0.8\linewidth,height=6cm]{attack-tree-admin}
\end{frame}

%%%%%%%%%%%%%%%%%%%%%%%%%%%%%%%%%%%%%%%%%%%%%%%%%%%%%%%%%%%%%%%%%%%%%%%%%%%%%%
\subsection{Threat agents}

\begin{frame}
\frametitle{Who are they?}
\begin{itemize}
\item External threats
	\begin{itemize}
	\item internet users
	\item viruses
	\item organized crime
	\end{itemize}
\item Internal threats
	\begin{itemize}
	\item employees
    \item intranet users
    \item administrators
    \end{itemize}
\item Natural threats (disasters)
\item Technical failures
	\begin{itemize}
	\item loss of essential services
		% (example: no electricity)
	\item hardware failure
	\item software failure
	\end{itemize}
\end{itemize}
\end{frame}

\begin{frame}
\frametitle{Why do they attack?}
\begin{itemize}
\item Gain
	\begin{itemize}
	\item financial
	\item information
	\item competition
	\end{itemize}
\item To show off
\item By accident (unintentional)
\end{itemize}
\begin{block}{Impact}
Threat agents have different skills, resources and different motivations that
may have a different influence on the risk.
\end{block}
\end{frame}

%%%%%%%%%%%%%%%%%%%%%%%%%%%%%%%%%%%%%%%%%%%%%%%%%%%%%%%%%%%%%%%%%%%%%%%%%%%%%%
\subsection{Risks and rating}

\begin{frame}
\frametitle{What is a \emph{risk}?}
\begin{quote}
The potential that a given threat will exploit vulnerabilities of
an asset or group of assets and thereby cause harm to the organization. It
is measured in terms of a combination of the probability of occurrence of an
event and its consequence.
\textit{-- Wikipedia}
\end{quote}
\end{frame}

\begin{frame}
\frametitle{What is a \emph{risk}?}
\[ \text{risk} = \text{likelihood} * \text{impact} \]
\begin{itemize}
\item Likelihood: probability of a successful attack
	\\ Influenced by threat agent skills and vulnerability factors
\item Impact: how much damage the attack causes
\end{itemize}
\end{frame}

\begin{frame}
\frametitle{Vulnerability factors}
\begin{itemize}
\item ease of discovery
\item ease of exploit
\item awareness
\item intrusion detection
\end{itemize}
\end{frame}

\begin{frame}
\frametitle{Impact factors}
\begin{itemize}
\item loss of confidentiality
\item loss of integrity
\item loss of availability
\item loss of accountability
\item financial damage
\item reputation damage
\item privacy violation
\end{itemize}
\end{frame}

%%%%%%%%%%%%%%%%%%%%%%%%%%%%%%%%%%%%%%%%%%%%%%%%%%%%%%%%%%%%%%%%%%%%%%%%%%%%%%
\subsection{Context of web applications}

\subsubsection{Network}

\begin{frame}
\frametitle{Internet}
\begin{itemize}
\item Applications available on the internet
\item Anyone can access the application
\end{itemize}
\begin{block}{Anonymity}
No identifier for a web user.
\\ Example: IP address shared by multiple users
\end{block}
\end{frame}

\subsubsection{Architecture of a web application}

\begin{frame}
\frametitle{A web application}
\includeimage[width=\linewidth,height=6cm]{archi-webapplication}
%- Browser: the client, uses scripting
%- Router: connected to the browser (by WiFi) - LAN
%- Internet: not detailed here - WAN
%- Web Application Firewall, IDS, Load Balancer
%- Application Server: with the contained application
%- Database: where the application stores its data
\end{frame}

%\subsubsection{Points of failure}

\begin{frame}
\frametitle{Points of failure}
Any component may fail; will the web site continue to work after an attack?
\begin{block}{Single point of failure}
A part of a system that prevents the entire system from working when it fails.
\\ Solution: duplication
\end{block}
\end{frame}

%%%%%%%%%%%%%%%%%%%%%%%%%%%%%%%%%%%%%%%%%%%%%%%%%%%%%%%%%%%%%%%%%%%%%%%%%%%%%%
\subsection{OWASP - Top 10}

\begin{frame}
\frametitle{OWASP}
\begin{block}{OWASP}
\centering Open Web Application Security Project
\begin{itemize}
\item standards
\item libraries
\item books
\item etc.
\end{itemize}
\end{block}
\begin{block}{Project: Top 10}
OWASP compiles at top 10 of most critical web application \emph{risks}.
\end{block}
\end{frame}

\begin{frame}
\frametitle{The top 10}
% TODO list the top 10
\end{frame}

\subsubsection{Attack 1 - Injection}

\begin{frame}
\frametitle{1. Injection}
\textbf{Injection} consists in sending untrusted data to an interpreter.
Attacker enters code to execute into input fields.
\begin{block}{Impacts}
Data loss, corruption, lack of accountability, etc.
\end{block}
\begin{block}{Risks}
\alert{Easy to exploit}, has a \alert{severe impact}, it is quite common.
\end{block}
\end{frame}

\begin{frame}
\frametitle{1. Injection - Attack scenario}
\begin{itemize}
\item The user enters a sequence of characters that is invalid for the
	interpreter. The server will send an invalid result.
	Probably, the executed code or error will be sent.
\item Complete the query with what you need the interpreter to execute.
\end{itemize}
\end{frame}

\begin{frame}[fragile]
\frametitle{1. Injection - Vulnerability}
\begin{exampleblock}{Bypassing authentication}
This password may work:
\begin{lstlisting}
' or '' = '
\end{lstlisting}
When authentication checks the user as follows:
\begin{lstlisting}
String query = "select * from users"
	+ " where user_name = '" + name + "'"
	+ " and password = '" + password + "'";
\end{lstlisting}
\end{exampleblock}
\pause
\begin{exampleblock}{SQL injection - demo}
Getting the users' passwords as administrator.
\end{exampleblock}
\pause
\begin{exampleblock}{JPQL injection - demo}
Searching hotels that have a manager that has the same password as a given
user.
\end{exampleblock}
\end{frame}

\begin{frame}[fragile]
\frametitle{1. Injection - Preventing the attack}
\begin{block}{Using parametrized interface}
Usage of placeholders for values let's the interpreter escape input values.
\begin{lstlisting}
String query = "select * from users where user_name = ?";
PreparedStatement st = con.prepareStatement(query);
st.setString(1, name);
ResultSet rs = st.executeQuery();
\end{lstlisting}
\end{block}
\begin{block}{Escaping routines}
Some languages have special function to escape manually values.
\end{block}
\begin{block}{White list validation}
List of valid input patterns.
\end{block}
\end{frame}

\subsubsection{Attack 2 - XSS}

\begin{frame}
\frametitle{2. XSS}
\textbf{Cross-site scripting} allows attackers to inject code into the page
sent to the user. The injected code can be a script interpreted by the
browser.
\begin{block}{Impacts}
Hijack user sessions, change content, redirect the user.
\end{block}
\begin{block}{Risks}
This is \alert{the most widespread} vulnerability.
It requires an average knowledge to be exploited and the impacts are moderate.
\end{block}
\end{frame}

\begin{frame}
\frametitle{2. XSS - Attack scenario}
\begin{itemize}
\item The attacker sends some text containing a script to the server
\item The user downloads the page with the script as it was posted by the
	attacker
\item The script is executed on the user's browser
\end{itemize}
\end{frame}

\begin{frame}[fragile]
\frametitle{2. XSS - Vulnerability}
\begin{exampleblock}{JSP page}
The description of the hotel is sent as is.
\begin{lstlisting}
<div>
	${hotel.descriptionHTML}
</div>
\end{lstlisting}
\end{exampleblock}
\pause
\begin{exampleblock}{Getting the user's cookies - demo}
Just use \lstinline!document.cookie! in JavaScript and send it somewhere.
\end{exampleblock}
\end{frame}

\begin{frame}[fragile]
\frametitle{2. XSS - Preventing the attack}
\begin{block}{Escaping the values}
All the values should be escaped before sending them to the users.
\begin{lstlisting}
<div>
	<c:out value="${hotel.descriptionHTML}" />
</div>
\end{lstlisting}
\end{block}
\begin{block}{White list validation}
Output values can be white listed, but this is not a complete defence against
XSS as sometimes special characters must be accepted.
\end{block}
\end{frame}

\subsubsection{Attack 3 - Authentication and session management}

\begin{frame}
\frametitle{3. Authentication and session management}
The attacker uses flaws in the \textbf{authentication or session management}
implemented for a given web site to steal someone else's identity.
\begin{block}{Impacts}
Once an account stolen, the attacker may do \emph{anything} the user can do.
\end{block}
\begin{block}{Risks}
The impact is \alert{severe}, this attack is common.
\end{block}
\end{frame}

\begin{frame}
\frametitle{3. Authentication and session management - Attack scenario}
\begin{itemize}
\item In older application servers the session ID could be set using the URL.
	\\ You could let someone login using your session.
\item The user could forget to log out on a public computer.
\item The attacker could gain access to the users' database.
\end{itemize}
\end{frame}

\begin{frame}
\frametitle{3. Authentication and session management - Vulnerability}
\begin{exampleblock}{XSS session hijacking - demo}
The users' sessions can be obtained using XSS.
\end{exampleblock}
\end{frame}

\begin{frame}
\frametitle{3. Authentication and session management - Preventing the attack}
\begin{block}{Good authentication mechanism}
\begin{itemize}
\item Use a proven authentication mechanism
\item Prevent \emph{XSS attacks}
\item Good session timeouts and accessible log out buttons
\end{itemize}
\end{block}
\end{frame}

\subsubsection{Attack 4 - Insecure Direct Object References}

\begin{frame}
\frametitle{4. Insecure Direct Object References}
The attacker changes a parameter in the request to obtain a \textbf{direct
object reference} that they should not be able to access.
\begin{block}{Impacts}
Compromise the data that can be referenced.
\end{block}
\begin{block}{Risks}
\alert{Easy} to exploit and detect. The impact is moderate.
\end{block}
\end{frame}

\begin{frame}
\frametitle{4. Insecure Direct Object References - Attack scenario}
\begin{itemize}
\item Find a URL with an identifier as a part of the URL
\item Replace that identifier with another value and try to send the request
\end{itemize}
\end{frame}

\begin{frame}
\frametitle{4. Insecure Direct Object References - Vulnerability}
\begin{block}{Missing access verification}
Often objects are retrieved by their ID, the developer can forget to check
whether the user is allowed to see that object.
\end{block}
\pause
\begin{exampleblock}{An unvalidated hotel - demo}
See an unvalidated hotel as a normal user.
\end{exampleblock}
\end{frame}

\begin{frame}
\frametitle{4. Insecure Direct Object References - Preventing the attack}
\begin{block}{\textbf{Check permissions}}
Check whether the user has the permissions to access the object before
manipulating it.
\end{block}
\begin{block}{References by session}
Use indirect references for each session.
\end{block}
\end{frame}

\subsubsection{Attack 5 - CSRF}

\begin{frame}
\frametitle{5. CSRF}
DESCRIPTION
\begin{block}{Impacts}
TODO Data loss, corruption, lack of accountability, etc.
\end{block}
\begin{block}{Risks}
TODO ease of exploitation
\end{block}
\end{frame}

\begin{frame}
\frametitle{5. CSRF - Attack scenario}
\begin{itemize}
\item TODO
\end{itemize}
\end{frame}

\begin{frame}[fragile]
\frametitle{5. CSRF - Vulnerability}
\begin{exampleblock}{TODO}
\end{exampleblock}
\end{frame}

\begin{frame}[fragile]
\frametitle{5. CSRF - Preventing the attack}
\begin{block}{TODO}
\end{block}
\end{frame}

\subsubsection{Attack 6 - Security Misconfiguration}

\begin{frame}
\frametitle{6. Security Misconfiguration}
DESCRIPTION
\begin{block}{Impacts}
TODO Data loss, corruption, lack of accountability, etc.
\end{block}
\begin{block}{Risks}
TODO ease of exploitation
\end{block}
\end{frame}

\begin{frame}
\frametitle{6. Security Misconfiguration - Attack scenario}
\begin{itemize}
\item TODO
\end{itemize}
\end{frame}

\begin{frame}[fragile]
\frametitle{6. Security Misconfiguration - Vulnerability}
\begin{exampleblock}{TODO}
\end{exampleblock}
\end{frame}

\begin{frame}[fragile]
\frametitle{6. Security Misconfiguration - Preventing the attack}
\begin{block}{TODO}
\end{block}
\end{frame}

\subsubsection{Attack 7 - Insecure Cryptographic Storage}

\begin{frame}
\frametitle{7. Insecure Cryptographic Storage}
DESCRIPTION
\begin{block}{Impacts}
TODO Data loss, corruption, lack of accountability, etc.
\end{block}
\begin{block}{Risks}
TODO ease of exploitation
\end{block}
\end{frame}

\begin{frame}
\frametitle{7. Insecure Cryptographic Storage - Attack scenario}
\begin{itemize}
\item TODO
\end{itemize}
\end{frame}

\begin{frame}[fragile]
\frametitle{7. Insecure Cryptographic Storage - Vulnerability}
\begin{exampleblock}{TODO}
\end{exampleblock}
\end{frame}

\begin{frame}[fragile]
\frametitle{7. Insecure Cryptographic Storage - Preventing the attack}
\begin{block}{TODO}
\end{block}
\end{frame}

\subsubsection{Attack 8 - Failure to Restrict URL Access}

\begin{frame}
\frametitle{8. Failure to Restrict URL Access}
DESCRIPTION
\begin{block}{Impacts}
TODO Data loss, corruption, lack of accountability, etc.
\end{block}
\begin{block}{Risks}
TODO ease of exploitation
\end{block}
\end{frame}

\begin{frame}
\frametitle{8. Failure to Restrict URL Access - Attack scenario}
\begin{itemize}
\item TODO
\end{itemize}
\end{frame}

\begin{frame}[fragile]
\frametitle{8. Failure to Restrict URL Access - Vulnerability}
\begin{exampleblock}{TODO}
\end{exampleblock}
\end{frame}

\begin{frame}[fragile]
\frametitle{8. Failure to Restrict URL Access - Preventing the attack}
\begin{block}{TODO}
\end{block}
\end{frame}

\subsubsection{Attack 9 - Insufficient Transport Layer Protection}

\begin{frame}
\frametitle{9. Insufficient Transport Layer Protection}
DESCRIPTION
\begin{block}{Impacts}
TODO Data loss, corruption, lack of accountability, etc.
\end{block}
\begin{block}{Risks}
TODO ease of exploitation
\end{block}
\end{frame}

\begin{frame}
\frametitle{9. Insufficient Transport Layer Protection - Attack scenario}
\begin{itemize}
\item TODO
\end{itemize}
\end{frame}

\begin{frame}[fragile]
\frametitle{9. Insufficient Transport Layer Protection - Vulnerability}
\begin{exampleblock}{TODO}
\end{exampleblock}
\end{frame}

\begin{frame}[fragile]
\frametitle{9. Insufficient Transport Layer Protection - Preventing the attack}
\begin{block}{TODO}
\end{block}
\end{frame}

\subsubsection{Attack 10 - Unvalidated Redirects}

\begin{frame}
\frametitle{10. Unvalidated Redirects}
DESCRIPTION
\begin{block}{Impacts}
TODO Data loss, corruption, lack of accountability, etc.
\end{block}
\begin{block}{Risks}
TODO ease of exploitation
\end{block}
\end{frame}

\begin{frame}
\frametitle{10. Unvalidated Redirects - Attack scenario}
\begin{itemize}
\item TODO
\end{itemize}
\end{frame}

\begin{frame}[fragile]
\frametitle{10. Unvalidated Redirects - Vulnerability}
\begin{exampleblock}{TODO}
\end{exampleblock}
\end{frame}

\begin{frame}[fragile]
\frametitle{10. Unvalidated Redirects - Preventing the attack}
\begin{block}{TODO}
\end{block}
\end{frame}

%%%%%%%%%%%%%%%%%%%%%%%%%%%%%%%%%%%%%%%%%%%%%%%%%%%%%%%%%%%%%%%%%%%%%%%%%%%%%%
\subsection{DoS}

%%%%%%%%%%%%%%%%%%%%%%%%%%%%%%%%%%%%%%%%%%%%%%%%%%%%%%%%%%%%%%%%%%%%%%%%%%%%%%
\subsection{MitM}

%%%%%%%%%%%%%%%%%%%%%%%%%%%%%%%%%%%%%%%%%%%%%%%%%%%%%%%%%%%%%%%%%%%%%%%%%%%%%%
\subsection{Social engineering}

%%%%%%%%%%%%%%%%%%%%%%%%%%%%%%%%%%%%%%%%%%%%%%%%%%%%%%%%%%%%%%%%%%%%%%%%%%%%%%
\subsubsection{Phishing}

%%%%%%%%%%%%%%%%%%%%%%%%%%%%%%%%%%%%%%%%%%%%%%%%%%%%%%%%%%%%%%%%%%%%%%%%%%%%%%
\subsection{Other attacks}

%%%%%%%%%%%%%%%%%%%%%%%%%%%%%%%%%%%%%%%%%%%%%%%%%%%%%%%%%%%%%%%%%%%%%%%%%%%%%%
\subsection{Common points}


\section{Prevention}

\begin{frame}
\sectionpage
\end{frame}

%------------------------------------------------
\subsection{Different moments}

%------------------------------------------------
\begin{frame}
\frametitle{Different moments}
Preventing security flaws is something that has to be done during the
whole life of the application.
\begin{columns}
\begin{column}{0.5\textwidth}
\begin{itemize}
\item Architecture
\item Development
\item Maintenance
\end{itemize}
\end{column}
\begin{column}{0.5\textwidth}
\includeimage[width=\linewidth,height=0.5\textheight]{milos3d}
\end{column}
\end{columns}
\end{frame}

%------------------------------------------------
\begin{frame}
\frametitle{Architecture}
\begin{itemize}
\item Security has to be taken into account as soon as
  possible. Retrofitting security in an unsecure application is much
  more difficult than conceiving it securely from the start
\item You have to ask yourself the good questions and those questions
  vary from one application to the other
\item Know the threats you are facing
\item There is no miracle receipt... just guidelines
\end{itemize}
\note{
	Question examples:
	\begin{itemize}
	\item Who are the end-users?
	\item In what environment will the application run?
	\item Who has access to the data?
	\end{itemize}
	Source:
	\url{https://www.owasp.org/index.php/Application_Security_Architecture_Cheat_Sheet}
}
\end{frame}

%------------------------------------------------
\begin{frame}
\frametitle{Development: implementation tricks}

There are lots of best practices that can make your application more
secure.

\begin{itemize}
\item Passwords: enforcing a minimum complexity policy, storing them securely,
having the user rotate them, preventing brute-force by implementing temporary
lock-out, ...
\item Session management: setting an inactivity time out, setting the
\emph{secure} flag for sensitive cookies, using the \emph{HTTP-Only} flag for
cookies, implementing an easily accessible logout, ...
\item Input validation: always on server side, use a positive
  approach
\item Use ad-hoc protection for known attacks
\end{itemize}
\note{
	Ad-hoc protection means a protection specialized for a particular attack
	in this context.
}
\end{frame}

%------------------------------------------------
\begin{frame}
\frametitle{Development: secure development cycle}
A secure development cycle should set aside some time for:
\begin{itemize}
\item Code review \small{(security oriented and general)}
\item Pentesting
\item Security training of the developers
\item Choosing a secure and well maintained framework
	%(more on frameworks later)
\item Devising a secure configuration for deployment
\end{itemize}

\end{frame}

%------------------------------------------------
\begin{frame}
\frametitle{Maintenance}
When vulnerabilities have been discovered they have to be patched as soon as
possible.

Beware of regressions and new security breaches.

\begin{block}{Virtual Patching}
Virtual Patching is a security policy layer that prevents the exploitation of a known
vulnerability until a patch is released. It does so by filtering the
traffic that goes to and from the application server and blocking
requests that are exploits.
\end{block}

\end{frame}

%------------------------------------------------
\subsection{Using frameworks}

%------------------------------------------------
\begin{frame}
\frametitle{Frameworks}
\begin{itemize}
\item Your application is not the first
\item Frameworks ease development while providing security mechanisms
\end{itemize}
\begin{block}{Some security mechanisms}
\begin{itemize}
\item Authentication support
\item Escaping of values
\item Standard input validation
\end{itemize}
\end{block}
\begin{block}{Disadvantages}
\begin{itemize}
\item Changing a framework may prove difficult
\item Bug in a framework is known
	(version management needed)
\end{itemize}
\end{block}
\end{frame}

%------------------------------------------------
\begin{frame}
\frametitle{Frameworks - examples}

\begin{exampleblock}{Spring}
Spring is one of the most used framework for developing web application using
Java. It does feature many components including some dedicated to security.
\end{exampleblock}

\begin{exampleblock}{ESAPI}
The OWASP Enterprise Security API is a free, open source, web application
security control library that is designed to make it easier for programmers
to retrofit security into existing applications.
The ESAPI libraries also serve as a solid foundation for new development.
\end{exampleblock}

\end{frame}

%------------------------------------------------
\subsection{Security in application servers}

%------------------------------------------------
\begin{frame}
\frametitle{Security in application servers}
Application servers offer various services that can help you configure
web applications.

\begin{itemize}
\item Authentication management
\item Directory listing disabling
\item User session management
\item Error handling
\item Input encoding
\item etc.
\end{itemize}
\end{frame}

%------------------------------------------------
\subsection{Application Firewall}

%------------------------------------------------
\begin{frame}
\frametitle{Web Application Firewall}
\textbf{Web Application Firewall} (WAF) are filters which inspects the traffic
that goes in and out your application and are able to block inappropriate
content.

\begin{block}{Some features}
\begin{itemize}
\item Virtual patching
\item Scanning content
%\item Detecting connection patterns
% Masked as it is explained in detection
\end{itemize}
\end{block}
\end{frame}

%------------------------------------------------
\subsection{HTTPS}

%------------------------------------------------
\begin{frame}
\frametitle{HTTPS}
\textbf{TLS} (Transport Layer Security) is the successor of \textbf{SSL}
(Secured Socket Layer).
Equal in terms of security.
The main difference: TLS begins with an insecure "hello".
\begin{block}{What provides HTTPS?}
HTTPS provides \emph{encryption} and \emph{authentication} for HTTP.
\newline Protects against eavesdropping and spoofing.
\end{block}
\vspace{1em}
HTTPS is a part of a solution, it does not ensure full security.
\end{frame}

%------------------------------------------------
\subsection{Browser restrictions}

%------------------------------------------------
\begin{frame}
\frametitle{Browser restrictions}
\begin{itemize}
\item Cross-domain XMLHttpRequest (XHR) is disabled by default
	\\ Exceptions defined using \texttt{Access-Control}
\item Possibility of disabling JavaScript and plug-ins (Flash)
\item Private navigation mode
	\\ Does not store data on the disk
\item Cookies access is limited by
	\begin{itemize}
	\item URL
	\item Timestamp
	\item Flags (Secure, HTTP-Only)
	\end{itemize}
\item SSL Certificates are validated
\item Blacklists of known rogue sites
\end{itemize}
\note{
	XHR allows retrieving pages using JavaScript.
}
\end{frame}

%------------------------------------------------
\subsection{E-mail}

%------------------------------------------------
\begin{frame}
\frametitle{E-mail}
\begin{center}
E-mail is one of the most used application.
\par
And it's plagued by spam and spoofing.
\end{center}
\begin{block}{Sender Policy Framework (SPF)}
SPF validates e-mails by verifying sender's IP address against DNS records.
Administrators can define allowed hosts.
\begin{quote}
ietf.org. IN SPF "v=spf1 ip4:64.170.98.0/26 ip4:64.170.98.64/28 ip4:64.170.98.80/28 ip6:2001:1890:1112:1::0/64 -all"
\end{quote}
\end{block}
\end{frame}

\begin{frame}
\frametitle{E-mail}
\begin{block}{Domain Keys Identified Mail (DKIM)}
DKIM associates domain names and email messages using a digital signature
added to the header of messages, a public key is added to the DNS record of the
domain for the recipient to read the message's header.
\begin{quote}
DKIM-Signature: v=1; a=rsa-sha256; d=example.net; s=brisbane;
c=relaxed/simple; q=dns/txt; l=1234; t=1117574938; x=1118006938;
h=from:to:subject:date:keywords:keywords;
bh=MTIzNDU2Nzg5MDEyMzQ1Njc4OTAxMjM0NTY3ODkwMTI=;
b=dzdVyOfAKCdLXdJOc9G2q8LoXSlEniSbav+yuU4zGeeruD00lszZVoG4ZHRNiYzR
\end{quote}
\end{block}
\end{frame}

%------------------------------------------------
\subsection{Variety of environments}

%------------------------------------------------
\begin{frame}
\frametitle{Variety of environments}
\begin{center}
Impossible to test all the setups.
\end{center}
\begin{itemize}
\item Servers \small{(and their configuration)}
\item Networks
\item Clients
	\begin{itemize}
	\item Browsers
		\\ hacks \small{(example: IE6)} are just more code to test
	\item Mobile
	\item M2M
	\end{itemize}
\end{itemize}
\note{
	\begin{itemize}
	\item Networks: proxy, caches, etc.
	\item M2M: Machine to Machine
	\end{itemize}
}
\end{frame}



\section{Detection and Analysis}

\subsection{Penetration testing}

\begin{frame}
\frametitle{Pentesting}
\includeimage[width=0.7\linewidth,height=0.7\textheight]{pentesting}
\note{
\textbf{Penetration testing} (or pentesting) is the action of attacking a
system with the consent of the owner in the goal of finding security holes.
\begin{itemize}
\item Identifying vulnerabilities
\item Exploit vulnerabilities
\item Testing ability to detect and respond to an attack
\item Social engineering
\end{itemize}
}
\end{frame}

\begin{frame}
\begin{center}
Avoiding this situation.
\end{center}
\includeimage[width=\linewidth,height=0.7\textheight]{fence-fail}
\end{frame}

\begin{frame}
\frametitle{Tools}
\includeimage[width=0.8\linewidth,height=0.7\textheight]{bt5-screenshot}
% TODO image nmap, ntop, etc.
\note{
\begin{itemize}
\item \href{http://nmap.org/}{nmap}:
	port scanning
\item \href{http://www.metasploit.com/about/penetration-testing-basics/}{Metasploit}:
	identifies vulnerabilities, automates assessments
\item \href{http://www.tenable.com/products/nessus}{Nessus}:
	vulnerability scanner
\item \href{https://www.owasp.org/index.php/Webscarab}{WebScarab}:
	intercepting proxy
\item \href{http://www.backtrack-linux.org/}{BackTrack}:
	Linux distribution
\end{itemize}
}
\end{frame}

\subsection{Honeypot}

\begin{frame}
\frametitle{Honeypot}
\includeimage[width=0.8\linewidth,height=0.7\textheight]{honeypot}
\note{
A \textbf{honeypot} is a mocked website or web application that is exposed for
attackers who try to penetrate some system. A honeypot is a non-patched server
that is monitored for attacks and serves no other purpose.
\begin{exampleblock}{\href{https://www.projecthoneypot.org/about_us.php}{Project Honey Pot}}
Anti-spammer project.
Web sites publish some fake e-mails and spammer's crawlers sending e-mail to
such addresses are blocked.
\end{exampleblock}
}
\end{frame}

\subsection{NIDS}

\begin{frame}
\frametitle{NIDS}
\includeimage[width=0.8\linewidth,height=0.7\textheight]{snort}
% TODO try to find some schema
\note{
\textbf{Network Intrusion Detection System} (NIDS) is a system that scans
network activity and tries to detect attacks.
\begin{exampleblock}{\href{http://www.snort.org/}{Snort}}
A NIDS, and an IPS (Intrusion Prevention System), snort can detect attacks and
take actions accordingly.
\end{exampleblock}
}
\end{frame}

\begin{frame}
\frametitle{NIDS - detection types}
\begin{itemize}
\item Signature-based detection
\item Statistical anomaly-based detection
\item Stateful protocol analysis
\end{itemize}
\note{
\begin{block}{Signature-based detection}
Finds a pattern in packets on the network.
Detects SQL injections in HTTP parameters, XSS attacks, etc.
\end{block}
\begin{block}{Statistical anomaly-based detection}
Looks for peaks in traffic, connection coming from a single host and other
anomalies.
Detects stealth port scans, brute force attacks, etc.
\end{block}
\begin{block}{Stateful protocol analysis}
Follows a protocol conversation.
Detects SYN port scans, etc.
\end{block}
}
\end{frame}

\subsection{Post-mortem analysis}

\begin{frame}
\frametitle{After an attack}
\includeimage[width=0.8\linewidth,height=0.7\textheight]{investigation}
% TODO better quality image
\note{
\begin{itemize}
\item \emph{When} did the attack happen
	\begin{itemize}
	\item Were we attacked?
	\item Is the attack still in progress?
	\end{itemize}
\item \emph{What} was compromised
	\begin{itemize}
	\item Data was compromised?
	\item Data was modified?
	\item What systems failed to work?
	\end{itemize}
\item \emph{How} the attacker succeeded
	\begin{itemize}
	\item How to fix the problem?
	\item Prevent future attacks?
	\end{itemize}
\end{itemize}
}
\end{frame}

\begin{frame}
\frametitle{How to know?}
\begin{itemize}
\item Traffic logs and graphs
\item Application logs
\end{itemize}
\begin{columns}
\begin{column}{0.5\textwidth}
\includeimage[width=\linewidth]{dos-load}
\end{column}
\begin{column}{0.5\textwidth}
\includeimage[width=\linewidth]{dos-bw}
\end{column}
\end{columns}
\note{
Read the logs and alerts.
\begin{exampleblock}{DoS}
Peak in server load and in input traffic.
Once the attack is effective, the outbound traffic drops.
\end{exampleblock}
}
\end{frame}

\begin{frame}
\frametitle{Analysis - tools}
\includeimage[width=0.8\linewidth,height=0.7\textheight]{weblog}
% TODO add another image
\note{
\begin{itemize}
\item \href{http://www.ntop.org/products/ntop/}{ntop}: traffic monitoring
platform
\item \href{http://www.intersectalliance.com}{Snare}: a collection of tools to
audit log data and facilitate log analysis
\item \href{http://www.ossec.net/}{OSSEC}: open-source alternative; performs
log analysis, file integrity checking, policy monitoring, rootkit detection,
real-time alerting and active response
\item Web log analysis software
	\begin{itemize}
	\item \href{http://www.webalizer.org/}{Webalizer}
	\item \href{http://awstats.sourceforge.net/}{AWStats}
	\item \href{http://www.weblogexpert.com/}{WebLog Expert}
	\end{itemize}
\end{itemize}
}
\end{frame}

\subsubsection{Noise as a diversion}

\begin{frame}
\frametitle{Noise as a diversion}
\includeimage[width=0.8\linewidth,height=0.7\textheight]{train_cache}
\note{
When an attack happens, it can be masked by another attack which would serve
as a diversion from the real attack.
\begin{exampleblock}{DoS as a diversion}
DoS generates lot of traffic and can mask attacks as generated logs are huge.
\end{exampleblock}
}
\end{frame}

\subsection{Performances}

\begin{frame}
\frametitle{Performances}
\includeimage[width=0.8\linewidth,height=0.7\textheight]{performance}
\note{
\begin{itemize}
\item Slow response time and high load might be a symptom of an attack
\item Slow application implies higher loads and may not scale and crash during
	an attack
\end{itemize}
}
\end{frame}



%-----------------------------------------------------------------------------

\section{Conclusion}

\begin{frame}
\includeimage[height=0.7\textheight]{security-cloud}
\note{
 We have reviewed various threat to web application
but this list is by no means exhaustive. Those are merely example, the
important point is the use of good security practice and procedure in
order to reduce the risk. Preventive action should be taken but you
should also be able to detect attack and to analyse them to prevent in
the future. It's also important to assess the damage.
}
\end{frame}

\begin{frame}
\begin{flushright}
\Huge
Any questions?
\end{flushright}
\vspace{1cm}
\includeimage[width=6cm]{hermes}
\end{frame}

%-----------------------------------------------------------------------------

\end{document}
