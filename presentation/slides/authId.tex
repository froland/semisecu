\section{Authentication and Identity}

%------------------------------------------------
\begin{frame}
\frametitle{Identity vs Authentication vs Authorization}
\begin{block}{Identity}
Who you are.
\end{block}
\pause
\begin{block}{Authentication}
Are you who you claim to be?
\end{block}
\pause
\begin{block}{Authorization}
What are you allowed to do?
\end{block}
\note{
\begin{block}{Identity}
The identity of somebody is who he is.
\end{block}
\begin{block}{Authentication}
Authentication is the act of asserting the identity of someone.
\end{block}
\begin{block}{Authorization}
Authorization is determining whether the user has the permission to do some
action.
\end{block}
}
\end{frame}

%------------------------------------------------
\begin{frame}
\frametitle{Why authentication?}
\begin{center}
\textbf{Identification and confidentiality}
\end{center}
\note{
Authentication is needed when you want to transmit
information and be sure that your correspondent isn't
impersonated by somebody else.
\begin{block}{Confidentiality}
Authentication by itself doesn't imply confidentiality.
For that, we have encryption.
\end{block}
}
\end{frame}

%------------------------------------------------
\subsection{Different means of authentication}

%------------------------------------------------
\begin{frame}
\frametitle{Password}
\includeimage[width=0.6\linewidth]{login}
\note{
\begin{itemize}
\item Password is the most common form of authentication
\item Password should be easily changeable
\item People should have different passwords for different services
\item Complexity must be sufficient
\end{itemize}
}
\end{frame}

\begin{frame}
\frametitle{Password complexity}
\includeimage[height=0.8\textheight,width=0.8\linewidth]{browserPasswordValidation}
\note{
Enforcing a minimum password strength:
\begin{itemize}
\item State the rules clearly (minimum 10 characters, at least one capital
letter, etc.)
\item Check the complexity in the browser and on the server
% feedback for the user
\end{itemize}
}
\end{frame}

\begin{frame}
\frametitle{Password strength}
\includeimage[height=0.75\textheight,width=\linewidth]{xkcd936_password_strength}
\note{
A passphrase vs a common password.
}
\end{frame}

%------------------------------------------------
\begin{frame}
\frametitle{One time passwords}
\includeimage[width=0.8\linewidth,height=0.7\textheight]{one-time-password}
\note{
\textbf{One time passwords} (OTP) are passwords which are valid only for a
single use.
\begin{itemize}
\item Prevent replay attack
\item Main difficulty: giving the users their passwords
	\small{(logistical problems)}
\item Used mainly by those who can afford big infrastructure
	\begin{itemize}
	\item states
	\item banks
	\item big companies
	\end{itemize}
\end{itemize}
}
\end{frame}

\begin{frame}
\frametitle{One time passwords - choice and distribution}
\begin{columns}
\begin{column}{0.5\linewidth}
\includeimage[width=\linewidth,height=0.7\textheight]{dice}
\end{column}
\begin{column}{0.5\linewidth}
\includeimage[width=\linewidth,height=0.7\textheight]{secure-id-token}
\end{column}
\end{columns}
\note{
To avoid prediction, passwords must be chosen using a random or at
least pseudo-random way.
\begin{itemize}
\item Time-based generation: passwords valid for a short time.
\item Sent by SMS, e-mail, etc.
\item Software on a mobile phone
\end{itemize}
}
\end{frame}

%------------------------------------------------
\begin{frame}
\frametitle{Certificates}
\includeimage[width=\linewidth,height=0.7\textheight]{certificatesVariety}
\note{
A trusted authority (Certification Authority) issues certificates to
confirm the identity of something. Those certificates may be of varying
quality and are most often used in SSL/TLS by web browsers.
}
\end{frame}

\begin{frame}[fragile]
\frametitle{Certificates - What are they made of?}
X.509~v3 standard
\begin{lstlisting}
Certificate:
	Data:
		Version: 1 (0x0)
		Serial Number: 7829 (0x1e95)
		Signature Algorithm: md5WithRSAEncryption
		Issuer: C=ZA, ST=Western Cape,...
		        OU=Certification Services Division,
		        CN=Thawte Server CA/emailAddress=...
		Validity
			Not Before: Jul  9 16:04:02 1998 GMT
			Not After : Jul  9 16:04:02 1999 GMT
\end{lstlisting}
\end{frame}
\begin{frame}[fragile]
\begin{lstlisting}
		Subject: C=US, ST=Maryland,...
		         OU=FreeSoft, CN=www.freesoft.org/emailAddress=...
		Subject Public Key Info:
			Public Key Algorithm: rsaEncryption
			RSA Public Key: (1024 bit)
				Modulus (1024 bit):
					00:be ...:41:8f
				Exponent: 65537 (0x10001)
	Signature Algorithm: md5WithRSAEncryption
		93:5f...:68:9f
\end{lstlisting}
\end{frame}

\begin{frame}
\frametitle{Certificates - How are they used?}
\includeimage[width=0.8\linewidth,height=0.75\textheight]{ssl-handshake}
\note{
Certificates are often used as a mean to distribute key in a public
key infrastructure: you start with a root CA certificate (usually
embedded into the application) and you check
the other certificate you need with it. A typical example is for
SSL/TLS in web browsers.
\begin{itemize}
\item Each CA does generate it's public-private key pair. They
  self-sign their certificates with their public key.
\item When the browser does require a CA certificates it does just ask
  the CA for it and validate it's integrity (by checking the hash).
\item When a web-site owner does want a certificate he does also
  generate a public key pair. He signs a request that he sends to a
  CA.
\item The CA check the integrity of the request and perform some
  checks on the identity of the owner.
\item If the CA validates the ID it makes and signs a certificate for
  the owner. 
\item When the websites want to sign a message, it does sign it with
  it's private key and send the signed object and it's certificate.
\item The browser checks the certificates against the signing CA
  certificate and the message against the checked certificate.
\end{itemize}
}
\end{frame}

\begin{frame}
\frametitle{Certificates - Distribution}
\includeimage[width=0.8\linewidth,height=0.75\textheight]{certification-process}
\note{
TODO merge with academic slides
}
\end{frame}

\begin{frame}
\frametitle{One way certification}
\includeimage[scale=0.7]{one-way-ssl}
\note{
\begin{block}{One-way authentication}
The browser authenticates the server but the opposite isn't true.
\end{block}
}
\end{frame}

\begin{frame}
\frametitle{Two-way certification}
\includeimage[scale=0.7]{two-way-ssl}
\note{
\begin{block}{Two-way authentication}
Both ends are authenticated.
\end{block}
}
\end{frame}

%------------------------------------------------
\begin{frame}
\frametitle{Tokens}
\includeimage[height=0.6\textheight]{token}
\note{
\textbf{Token} are something of which ownership gives a form of
authentication.
\begin{exampleblock}{Forms in everyday life}
\begin{itemize}
\item Keys
\item Bank cards
\item Badges
\item Digipass
\end{itemize}
\end{exampleblock}
}
\end{frame}

%------------------------------------------------
\begin{frame}
\frametitle{Multi-factor authentication}
\begin{columns}[T]
\begin{column}{0.3\linewidth}
\begin{center}
\bfseries
IS
\includeimage[width=\linewidth]{multifactor-is}
\end{center}
\end{column}
\begin{column}{0.3\linewidth}
\begin{center}
\bfseries
OWNS
\includeimage[width=\linewidth]{multifactor-own}
\end{center}
\end{column}
\begin{column}{0.3\linewidth}
\begin{center}
\bfseries
KNOWS
\includeimage[width=\linewidth]{multifactor-know}
\end{center}
\end{column}
\end{columns}
\note{
\begin{center}
More than one form of authentication for an application.
\end{center}
The entity is authenticated by:
\begin{itemize}
\item what it \emph{is} (biometrics, ...)
\item what it \emph{owns} (a bank card, ...)
\item what it \emph{knows} (a password, a pin code, ...)
\end{itemize}
}
\end{frame}

%------------------------------------------------
\subsection{Signature challenge}

%------------------------------------------------
\begin{frame}
\frametitle{Public-private keys}
\includeimage[width=0.8\linewidth]{ppk}
\note{
The classical cryptography approach is to use the same key to encrypt
and decrypt a message but there is a major drawback: in order to be
able to communicate with someone, you have to use a pre-established
key.

Public key cryptography was designed to solve that
problem. With those schemes, you encrypt (and sign) with the private
key and decrypt with the public one.Each participant generates a public
private key pair and give the public one to the other.
\begin{block}{PKI}
A PKI or Public Key Infrastructure is a mean to distribute the public
key in a secure way.
\end{block}
}
\end{frame}

%------------------------------------------------
\begin{frame}
\frametitle{Signature challenge - principle}
\includeimage[width=0.7\linewidth]{signature}
\note{
An authentication based upon "answering" a question about a secret
known by the participants.
\begin{exampleblock}{A nonce}
When I encrypt a nonce with my correspondent's public key and ask him for the
decrypted and encrypted message. Nonce stands for number used once.
It's a random number used once to avoid replay attack.
\end{exampleblock}
}
\end{frame}

%------------------------------------------------
\subsection{Single sign-on}

%------------------------------------------------
\begin{frame}
\frametitle{Single sign-on - principle}
\includeimage[width=0.8\linewidth,height=0.75\textheight]{sso}
\note{
\textbf{Single sign-on} (SSO) is a property of access control of multiple
related but independent software systems. With this property a user
logs in once and gains access to all systems without being prompted to
log in again for each of them.
\textit{-- Wikipedia}
\begin{block}{Pro}
\begin{itemize}
\item Easier for the user
\item COTS (commercial off-the-shelf: a ready product)
\item If they have only one password users tend to treat it with more care
\item Enter the password less often
\end{itemize}
\end{block}
\begin{block}{Cons}
\begin{itemize}
\item All your eggs in the same basket (impact if compromised)
\item You are dependent upon your provider (confidence, availability, ...)
\end{itemize}
\end{block}
}
\end{frame}

%------------------------------------------------
\subsection{Kerberos}

%------------------------------------------------
\begin{frame}
\frametitle{Kerberos}
\includeimage[width=3cm]{kerberos-logo}
\note{
\textbf{Kerberos} is a SSO developed at MIT to solve the problem of allowing
some users to use restricted resources.
MIT provides a free implementation of the protocol but it's also found in many
commercial products.
}
\end{frame}

\begin{frame}
\frametitle{Kerberos - How does it work?}
\includeimage[width=0.8\linewidth]{kerberos}
\note{
\begin{enumerate}
\item The client gets a \emph{ticket} for a distribution center to
authenticate itself
\item The client gets a \emph{service ticket}
\item The \emph{service ticket} can be used on another server
\end{enumerate}
}
\end{frame}

%------------------------------------------------
\subsection{Provider extensions}

%------------------------------------------------
\begin{frame}
\frametitle{Provider extensions}
\begin{columns}
\begin{column}[width=\linewidth]{0.3\linewidth}
\includeimage[width=0.8\linewidth]{openid}
\end{column}
\begin{column}[width=\linewidth]{0.3\linewidth}
\includeimage[width=0.8\linewidth]{google}
\end{column}
\begin{column}[width=\linewidth]{0.3\linewidth}
\includeimage[width=0.8\linewidth]{facebook}
\end{column}
\end{columns}
\note{
\begin{itemize}
\item Single sign-on provided using:
	\begin{itemize}
	\item Facebook account
	\item Google
	\item OpenID
		\begin{itemize}
		\item myOpenID
		\item ClaimID
		\item ...
		\end{itemize}
	\end{itemize}
\item Simpler to setup than developing your own single sign-on mechanism
\item Sometimes ties you to a single provider
\end{itemize}
}
\end{frame}

%------------------------------------------------
\subsection{Compromised authentication mean}

%------------------------------------------------
\begin{frame}
\frametitle{Compromised authentication mean}
\includeimage[width=0.8\linewidth]{lost-credentials}
\note{
When you implement authentication you have to take into account that
the user may lose its authentication mean or it can get stolen.
\begin{itemize}
\item Deactivate a compromised authentication mean
\item Authenticate the user by an uncompromised mean
\item Give him a new primary authentication mean
\end{itemize}
}
\end{frame}

%------------------------------------------------
\begin{frame}
\frametitle{Example: Lost password}
\begin{itemize}
\item Security questions
\item Another channel {\small (e-mail)}
\end{itemize}
\note{
If one of your users loses his password, he has to get a new one but doing
this in a secure way can be challenging.
\begin{itemize}
\item Use some predefined security questions
\item Send a token over a side channel \small{(e-mail)}
\item Confirm the change
\end{itemize}
}
\end{frame}

