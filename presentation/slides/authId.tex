\section{Authentication and Identity}

\begin{frame}
\sectionpage
\end{frame}

%------------------------------------------------
\begin{frame}
\frametitle{Identity vs Authentication vs Authorization}
\begin{block}{Identity}
The identity of somebody/something is who/what he/it is.
\end{block}
\begin{block}{Authentication}
Authentication is the act of asserting the identity of something/someone.
\end{block}
\begin{block}{Authorization}
Authorization is determining whether the user has the permission to do some
action.
\end{block}
\end{frame}

%------------------------------------------------
\begin{frame}
\frametitle{Why authentication?}
Authentication is needed when you want to transmit confidential
information, you want to be sure that your correspondent isn't
impersonated by somebody else.
\begin{block}{Confidentiality}
Authentication by itself doesn't imply confidentiality.
For that, we have encryption.
\end{block}
\end{frame}

%------------------------------------------------
\subsection{Different means of authentication}

%------------------------------------------------
\begin{frame}
\frametitle{Password}

\begin{itemize}
\item The most common form of authentication
\item Password should be easily changeable
\item People should have different passwords for different services
\item Complexity must be sufficient
\end{itemize}
\end{frame}

\begin{frame}
\frametitle{Password - complexity}
Enforcing a minimum password strength:
\begin{itemize}
\item State the rules clearly (minimum 10 characters, at least one capital
letter, etc.)
\item Check the complexity in the browser
% feedback for the user
\end{itemize}
\includeimage[height=0.4\textheight]{browserPasswordValidation}
\end{frame}

\begin{frame}
\frametitle{Password strength}
\includeimage[height=0.75\textheight,width=\linewidth]{xkcd936_password_strength}
\end{frame}

%------------------------------------------------
\begin{frame}
\frametitle{One time passwords}
\textbf{One time passwords} (OTP) are passwords which are valid only for a
single use.
\begin{itemize}
\item Prevent replay attack
\item Main difficulty: giving the users their passwords
	\small{(logistical problems)}
\item Used mainly by those who can afford big infrastructure
	\begin{itemize}
	\item states
	\item banks
	\item big companies
	\end{itemize}
\end{itemize}
\end{frame}

\begin{frame}
\frametitle{One time passwords - choice and distribution}
To avoid prediction, passwords must be chosen using a random or at
least pseudo-random way.
\begin{itemize}
\item Time-based generation: passwords valid for a short time.
	\includeimage[height=0.2\textheight]{SecureID_token_new}
\item Sent by SMS, e-mail, etc.
\item Software on a mobile phone
\end{itemize}
\end{frame}

%------------------------------------------------
\begin{frame}
\frametitle{Certificates}
A trusted authority (Certification Authority) issues certificates to
confirm the identity of something. Those certificates may be of varying
quality and are most often used in SSL/TLS by web browsers.
\includeimage[width=0.8\linewidth]{certificatesVariety}
\end{frame}

\begin{frame}[fragile]
\frametitle{Certificates - What are they made of?}
Classical web certificates are using X.509 v3 standard.
\begin{lstlisting}
Certificate:
   Data:
       Version: 1 (0x0)
       Serial Number: 7829 (0x1e95)
       Signature Algorithm: md5WithRSAEncryption
       Issuer: C=ZA, ST=Western Cape,...
               OU=Certification Services Division,
               CN=Thawte Server CA/emailAddress=...
       Validity
           Not Before: Jul  9 16:04:02 1998 GMT
           Not After : Jul  9 16:04:02 1999 GMT
\end{lstlisting}
\end{frame}
\begin{frame}[fragile]
\begin{lstlisting}
       Subject: C=US, ST=Maryland,...
                OU=FreeSoft, CN=www.freesoft.org/emailAddress=...
       Subject Public Key Info:
           Public Key Algorithm: rsaEncryption
           RSA Public Key: (1024 bit)
               Modulus (1024 bit):
                   00:be ...:41:8f
               Exponent: 65537 (0x10001)
   Signature Algorithm: md5WithRSAEncryption
       93:5f...:68:9f
\end{lstlisting}
\end{frame}

\begin{frame}[allowframebreaks]
\frametitle{Certificates - How are they used?}
Certificates are often used as a mean to distribute key in a public
key infrastructure. A typical example is for SSL/TLS in web browsers.
% TODO explain how the first certificate is distributed => included in browser
% TODO unreadable images
\includeimage[width=0.9\linewidth]{Usage-of-Digital-CertificatePart1}
\includeimage[width=0.9\linewidth]{Usage-of-Digital-CertificatePart2}
\includeimage[width=0.9\linewidth]{Usage-of-Digital-CertificatePart3}
\end{frame}

\begin{frame}
\frametitle{Certificates - Two-way authentication}
\begin{block}{One-way authentication}
The browser authenticates the server but the opposite isn't true.
\end{block}
\begin{block}{Two-way authentication}
Both ends are authenticated.
The server also checks the user's certificate.
% TODO explain why two one-way authentication <> two-way authentication
% if too complicated, explain in LaTeX comments
\end{block}
\end{frame}

%------------------------------------------------
\begin{frame}
\frametitle{Tokens}
\textbf{Token} are something of which ownership gives a form of
authentication.
\begin{exampleblock}{Forms in everyday life}
\begin{itemize}
\item Keys
\item Bank cards
\item Badges
\item Digipass
\end{itemize}
\end{exampleblock}
\end{frame}

%------------------------------------------------
\begin{frame}
\frametitle{Multi-factor authentication}
\begin{center}
More than one form of authentication for an application.
\end{center}
The entity is authenticated by:
\begin{itemize}
\item what it \emph{is} (biometrics, ...)
\item what it \emph{owns} (a bank card, ...)
\item what it \emph{knows} (a password, a pin code, ...)
\end{itemize}
\end{frame}

%------------------------------------------------
\subsection{Signature challenge}

%------------------------------------------------
\begin{frame}
\frametitle{Public-private keys}
% TODO cryptography introduction: public-private key difference, why 2 keys,
% PKI
\end{frame}

%------------------------------------------------
\begin{frame}
\frametitle{Signature challenge - principle}
An authentication based upon "answering" a question about a secret
known by the participants.
\begin{exampleblock}{A nonce}
When I encrypt a nonce with my correspondent's public key and ask him for the
decrypted and encrypted message.
% Explain how it works and what is a nonce.
\end{exampleblock}
This form of authentication is often used to sign online transactions.
\end{frame}

%------------------------------------------------
\subsection{Single sign-on}

%------------------------------------------------
\begin{frame}
\frametitle{Single sign-on - principle}
\textbf{Single sign-on} (SSO) is a property of access control of multiple
related, but independent software systems. With this property a user
logs in once and gains access to all systems without being prompted to
log in again at each of them.
\textit{-- Wikipedia}
\end{frame}

\begin{frame}
\frametitle{Single sign-on - pro and cons}
\begin{block}{Pro}
\begin{itemize}
\item Easier for the user
\item COTS % Commercial off-the-shelf
\item If they have only one password users tend to treat it with more care
\item Enter the password less often
\end{itemize}
\end{block}
\begin{block}{Cons}
\begin{itemize}
\item All your eggs in the same basket (impact if compromised)
\item You are dependent upon your provider (confidence, availability, ...)
\end{itemize}
\end{block}
\end{frame}

%------------------------------------------------
\subsection{Kerberos}

%------------------------------------------------
\begin{frame}
\frametitle{Kerberos}
\textbf{Kerberos} is a SSO developed at MIT to solve the problem of allowing
some users to use restricted resources.
MIT provides a free implementation of the protocol but it's also found in many
commercial products.
\end{frame}

\begin{frame}
\frametitle{Kerberos - How does it work?}
\includeimage[scale=0.35]{kerberos}
\textit{Wikipedia}
\end{frame}

%------------------------------------------------
\subsection{Provider extensions}

%------------------------------------------------
\begin{frame}
\frametitle{Provider extensions}
\begin{itemize}
\item Single sign-on provided using:
	\begin{itemize}
	\item Facebook account
	\item Google
	\item OpenID
		\begin{itemize}
		\item myOpenID
		\item ClaimID
		\item ...
		\end{itemize}
	\end{itemize}
\item Simpler to setup than developing your own sing-on mechanism
\item Sometimes ties you to a single provider
\end{itemize}
\end{frame}

%------------------------------------------------
\subsection{Compromised authentication mean}

%------------------------------------------------
\begin{frame}
\frametitle{Compromised authentication mean}
When you implement authentication you have to take into account that
the user may lose its authentication mean or it can get stolen.
\begin{itemize}
\item Deactivate a compromised authentication mean
\item Authenticate the user by an uncompromised mean
\item Give him a new primary authentication mean
\end{itemize}
\end{frame}

%------------------------------------------------
\begin{frame}
\frametitle{Example: Lost password}
If one of your users loses his password, he has to get a new one but doing
this in a secure way can be challenging.
\begin{itemize}
\item Use some predefined security questions
\item Send a token over a side channel \small{(e-mail)}
\item Confirm the change
\end{itemize}
\end{frame}

