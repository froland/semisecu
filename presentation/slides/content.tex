%% Content of websecurity-document

%%%%%%%%%%%%%%%%%%%%%%%%%%%%%%%%%%%%%%%%%%%%%%%%%%%%%%%%%%%%%%%%%%%%%%%%%%%%%%
\section{Psycho-sociological Aspects}

During the early years of the Web, when it still was a trusted and closed
environment originally developped by the U.S Department of Defense, security was
not a concern. At this time, only a few research and government sites were
linked together to share data. Later, in the middle of the eighties, the first
real network security incident appeared, causing several computer to crash down
among those research and government sites. The Network then continued to grow to
finally become the Internet as we know it today, but security problems continued
to appear in the same time, with both aggressive and defensive technologies
becoming more sophisticated. Nowadays security is present everywhere in our
electronic life: from our personal web applications (YouTube, Facebook) to our
financial web services (Online Banking, Tax-On-Web) and of course in our
professional lifes (e-mail, calendars). Security efforts are made to bring trust
to users, giving them a false sentiment of security. They need to gain some
level of trust but not to be lure by this false security feeling. Indeed, don't
we believe that HTTPS websites could be seen as fully secured web sites? In
reality they just prevent some of the attacks. Though, every web site should
indeed earn the confidence of its users because these ones will share their
personal data throught the network and it should be unacceptable to see these
data stolen because of a compromised web site.

That's why security aspect requires more specific measures, more guarantees and
a stronger protection against all kinds of fraud. But security is also a user's
responsability. They can divulgue their credentials, use unsecured web sites or
compromised software and, most of the time, they don't always understand what
they do. In addition, each critical web service now requires to use a password.
But with a number of services getting more and more important, remenbering
multiple passwords can become a hard task. A consequence is that most users use
the same password for several web sites. More, these passwords are generally
shorter than height characters and pass-phrases are most of the time not used.
Some online passwords storage services are existing to do the job for users, but
these services are also a way to track all yout different passwords. A
compromise between high security and ease of usability must then be found to
give users a trust in their web sites. This implies to make a choice between the
use of a secure or simple password, between using encoded data or, on the other
hand, using plain text but also to make the choice to use no authentication at
all or, on the other hand, some authorization checks. We discuss in the next
chapter several of these points through the study of different attacks.

%%%%%%%%%%%%%%%%%%%%%%%%%%%%%%%%%%%%%%%%%%%%%%%%%%%%%%%%%%%%%%%%%%%%%%%%%%%%%%
\section{Attack}

\subsection{Attack trees}
We will start our study of different attacks by describing the attack tree model
which give a description of how an asset can be attacked. In other words, this
kind of model is used to determine and understand threats that may arise on a
system. For an example, administration access to a database can be obtained
thanks to the knowledge of the password. One way to get this password is to
steal it. To steal it we can either use social engineering or sniffing
techniques. This reflexion able us to build an attack tree model showing threats
for the administration access asset. Threats are numerous and diversified and we
can distinguish several reasons to these attacks: \begin{itemize}{} \item
Threats can exists with the only purpose to break thinks. These attacks can come
from internet hackers or viruses but also from internal threats in your
enterprise such as unsatisfied employees. \item The second category of threats
has a different purpose which is to obtain a gain. That's the case with
organized crime or competitors wanting to acquire financial or technical
information. \item Last but not least, threats can simply be unintentional or
accidental such as loss of electricity, a software failure, a hardware failure
but also due to a natural disaster. \end{itemize}

To understand the risk encountered by systems vulnerable to such attacks,
let's first define what is a risk. A risk is the potential that a given threat
will exploit vulnerabilities of an asset or group of assets and thereby cause
harm to the organization. Is is measured in terms of a combination of the
probability of occurrence of an event and its consequence (Wikipedia). In
other terms, the risk is the relation between the probability of a successful
attack on an asset and the damages caused by this attack such as the loss of
confidentiality, integrity, availability and accountability, but also some
financial or reputation damages or even privacy violation.
Web applications are particularly sensitive to these attacks. Indeed, the
context of execution of web applications is quite complex and implies a lot of
components: browser, router, firewall, application server or database. A
single point of failure from one of these components would prevent the entire
system from correctly working. A worldwide organization known as OWASP (Open
Web Application Security Project) focuses on improving the security of
software through the publication of standards, librairies and books. They also
provided a top ten of most critical web applications risks. From higher risk
to lower risk we can find:
\begin{itemize}{}
\item Injection
\item Cross-site scripting (XSS)
\item Authentication and Session Management
\item Insecure Direct Object References
\item Cross-site Request Forgery (CSRF)
\item Security Misconfiguration
\item Insecure Cryptographic Storage
\item Failure to Restrict URL Access
\item Insufficient Transport Layer Protection
\item Unvalidated Forwards and Redirects
\end{itemize}

Let's study each security flaw listed above, their risks, their impacts and
how they can be prevented using adapted mechanisms.

\subsubsection{Injection}
First of all, "Injection" is the mechanism consisting in sending untrusted
data to an interpreter. In other words, hackers enter code to execute directly
into input fields of a web form. That is a very common attack, easy to
exploit, that can produces a lot of impacts such as data loss, corruption or
even lack of accountability. To avoid such security flaws, it is recommended
to use a parametrized interface, to escape routines and to use white list
validation to specify valid input patterns.

\subsubsection{Cross-site scripting (XSS)}
Cross-site scripting is an attack consisting in code injection directly into
the page that will be sent to the user. This code can be a script that will
finally be interpreted by the browser. This kind of attack requires some
knowledge from the attacker but impacts are generally moderated. Let's include
the hijack of a user session, the change of a page content or also the user
redirection to another web site. Once again, using escaping values before
sending them to the users or using white list validation can reduce risks
against this kind of attack.

\subsubsection{Authentication and Session Management}
Our third attack is the use of flaws in authentication and session management
implemented for a given web site to steal someone else's identity such as a
session hijacking or a session fixation. This can be possible, for an example,
by using a stolen session id directly into the URL. This type of attack is
very common and impacts are often important. Indeed, with the identity of
someone else, the attacker can do anything the real user can do. To avoid this
kind of attack, a few practices should be used. First do not forget to log out
on public computer because your session could be easily used by someone else.
A good practice is also to use session timeout and visible log out buttons.
Then, it is important to prevent against cross-site scripting attack because a
session id could be easily stolen thanks to this mechanism.

\subsubsection{Insecure Direct Object References}
Insecure direct object references is an attack that requires changes in the
URL request to obtain a direct object reference that should not be able to
access. For example by replacing an identifier into the request referencing an
accessible object by another identifier referencing an initially non
accessible object. This kind of attack is very simple to exploit and can
compromise referenced data. To bring more security, source code in web
applications should always checks if the user has the rights to access the
object before manipulating it.

\subsubsection{Cross-site Request Forgery (CSRF)}
Cross-site request forgery is a mechanism that consists in generating requests
when a user, the victim, visits the attacker's web page. The visited web page
then contains malicious code that can submit requests to the attacked web site
using the visiting user identity. This is a widespread attack but fortunately
easy to detect. To counter it, each request should contains an unique token
linked to the user session. This way, the attacker's website will not be able
to generate the correct requests, and so, will be unable to act as the victim.

\subsubsection{Security Misconfiguration}
Security flaws can also appear through security misconfiguration. Default
configuration used for application, server or database are not necessaraly a
good choice. It can let the possibility to gain information such as versions,
stack traces or debug messages that can lead to bigger threats which could
compromised the whole system (gain unauthorized access, use known bugs to hack
an application, etc). Keeping an up-to-date system is the first recommendation
to prevent such attacks. In combination, a good separation between components
is also more secure.

\subsubsection{Insecure Cryptographic Storage}
Insecure cryptographic storage is the next threat. Most sensitive data are
indeed generally stored insecurely without crypting. Gaining access to data is
the first step but once it is done, a cryptographic barrier should be used to
prevent attackers to read it. Risks are high for unencrypted data which will
be compromised. This kind of attack is fortunately quite difficult to perform
for attackers.

\subsubsection{Failure to Restrict URL Access}
Another threat implies missing URL access restriction. Attackers can change
their URL to switch to a privilieged page with functions requiring
permissions. This is a quite uncommon flaw but easy to exploit. Main impact is
of course the threat to give administrative functions available to the
attackers. To bring more security, authorisation mechanisms should be used
even if administration pages are not necesseraly referenced from any other
page.

\subsubsection{Insufficient Transport Layer Protection}
Next case concerns insufficient transport layer protection which is the
possibility for an attacker to sniff the data over the network to access it.
Examples of common attacks are sniffing a WiFi network, scannng for HTTP
requests or even analysing conversations. Any sniffer can be used to sniff a
network, however, results are often difficult to exploit. HTTPS protocol can
secure connections using SSL/TLS. Secure cookies will not be sent through an
unencrypted connection and certificates will be verify to ensure their
validity.

\subsubsection{Unvalidated Forwards and Redirects}
The last attack implies unvalidated forwards and redirects. This means that an
URL may appear as trustworthy but actually contains unvalidated forward or
redirect giving the opportunity to the attacker to bypass authorization
functionnalities. That's an easy attack to detect and nowadays a rarely
present kind of attack, impacting victims  into disclosing information or
permitting access control bypass. A barrier against these attacks should be to
avoid redirects and forwards as much as possible, and in the case if this is
not possible, to validate and authorize the user after each redirection.

Besides these attacks listed by the OWASP organization, we can also mentionned
other common attacks.

Denial of service is one of them. This attack aims to bring down (or at least
make unavailable) some service usually by causing server overload. Common
denial of service attacks are SYN flood, application level flood, DDoS or
unintentional DoS.

Another well knwon attack is the man in the middle attack. In this attack, the
attacker plays the role of the server to the client and the role of the client
to the server. This way, all communications transit trhough the attacker who
can modily messages as they are sent.

A less technical attack is social engineering. This attack simply consists in
manipulationg people so they divulgue confidential information. Different
scenarios are used by attackers in social engineering to achieve thier goal.
Pretexting is one of them and consist in using invented scenario to engage a
targeted victim in a manner that increases the chance the victim will divulgue
information or perform actions that would be unlikely ordinary circumstances.
The only way to fight against social engineering is to educate users and use a
framework of trust.

Phishing is the attemps to acquire user information through maquerading as a
legitimate website. Tricks used can be cloned websites, link manipulation or
even phone phishing.

Brute force is an attack whose purpose is to try all possible combinations of
a password to discover it. A dictionary attack is based on the same principle
but only try a list of common passwords.

As we can see, many forms of attacks exist and, as technology evolves, new
ones appear. So, instead of trying to fix vulerabilities, we should focus on
establishing strong security controls.

%%%%%%%%%%%%%%%%%%%%%%%%%%%%%%%%%%%%%%%%%%%%%%%%%%%%%%%%%%%%%%%%%%%%%%%%%%%%%%
\section{Authentication and Identity}

%%%%%%%%%%%%%%%%%%%%%%%%%%%%%%%%%%%%%%%%%%%%%%%%%%%%%%%%%%%%%%%%%%%%%%%%%%%%%%
\section{Prevention}

\section{Detection and Analysis}
To find security holes in a system, penetration testing, or pentesting, is
used to attack this system with the consent of the owner. The purpose is to
identify vulnerabilities, to exploit them as well as to test the ability to
detect them and give an appropriate response.

Several tools have been created to perform these tasks. NMap is a port
scanning tool, Metasploit is able to identify vulnerabilities and automates
assesments. Nessus is a vulnerability scanner. WebScarab is an intercepting
proxy.

Internet provides what we call Honeypot. A Honeypot si a mocked website or web
application running on a non patched server and that is exposed for attackers
who try to penetrate some system.

Some network Intrusion Detection Systems (NIDS) are systems that can scan
network activity and tries to detect attacks. Snort is a NIDS and can detect
attacks such as and take actions accordingly.

Among those attacks, Snort can perform signature-based detection, which means
it can find patterns in packets on the network, detects SQL injection or XSS
attacks. Snort can also perform statistical anomaly-based detection. It looks
for peaks in traffic, connection coming from a single host and other anomalies
or detect brute force attacks. In the end, Snort can realize stateful protocol
detection to catch SYN port scans or follow a protocol conversation.

If such an attack is detected on a system, several questions can be asked to
try to solve the problem. First, when did the attack happen? Maybe is this
attack still in progress? Then, what was compromised? Is it compromised or
modified? What systems failed to work? Finally, how to fix the problem to
prevent furure attacks?

To discover if an attack happened, a first clue could be to read traffic logs
and graphs as well as application logs.
Analysis tools such as ntop, Snare or OSSEC also exist to perform this kind of
job.

But looking for a specific attack can masked another one. Denial of service
can be used to generates lot of traffic and can mask attacks as generated logs
are huge.

Testing performances can also be used to detect possible attacks. Indeed, slow
response time and high load might be a symptom of an attack.

%%%%%%%%%%%%%%%%%%%%%%%%%%%%%%%%%%%%%%%%%%%%%%%%%%%%%%%%%%%%%%%%%%%%%%%%%%%%%%
\section{Detection}
