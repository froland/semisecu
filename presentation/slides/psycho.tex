\section{Psycho-sociological Aspects}
% TODO this chapter needs to be developed

\begin{frame}
\sectionpage
\end{frame}

\subsection{Security concerns}

\begin{frame}
\frametitle{Web security}
The web:
\begin{itemize}
\item Designed in a trusted environment
\item Became global
\end{itemize}
Security was not a concern in a closed environment.
\end{frame}

\begin{frame}
\frametitle{Web services}
Security concerns various aspects of our internet life.
\begin{itemize}
\item Personal (e-mail, Facebook, ...)
\item Financial (online banking)
\item Fiscal (Tax-On-Web)
\item etc.
\end{itemize}
\end{frame}

\subsection{Sentiment of security}

\begin{frame}
\frametitle{User confidence}
A web site must earn the confidence of its users.
\begin{block}{Why?}
Users will send their data to the server.
\end{block}
\begin{block}{However...}
The web site could be compromised.
\end{block}
\begin{exampleblock}{PlayStation Network outrage}
Information from approximately 77 million accounts was stolen.
\end{exampleblock}
\end{frame}

\begin{frame}
\frametitle{False sentiment of security}
Gain some level of trust...
\\ but, do not lure the users in a false sense of security.
\begin{exampleblock}{HTTPS}
Users may think that an HTTPS web site is secured.
\\ HTTPS prevents only some of the attacks.
\end{exampleblock}
\end{frame}

\begin{frame}
\frametitle{Security}
\includeimage[height=0.75\textheight,width=\linewidth]{xkcd538_security}
\end{frame}

\subsection{Web payment}

\begin{frame}
\frametitle{Web payment}
\begin{center}
You don't trust everyone with your money.
\end{center}
\begin{itemize}
\item Proper security measures
\item Guaranties
\item Protection against fraud
\end{itemize}
\begin{example}
Visa, MasterCard, PayPal, etc.
\end{example}
\end{frame}

\subsection{Users as a security flaw}

\begin{frame}
\frametitle{Role of the user}
\begin{itemize}
\item Users can divulge their credentials
\item They may use unsecured web sites
\item etc.
\end{itemize}
\begin{block}{Social engineering}
Manipulating people so they divulge confidential information.
\\ More on that later...
\end{block}
\end{frame}

\begin{frame}
\frametitle{Passwords example}
Remembering multiple passwords is... \emph{difficult}.
\begin{itemize}
\item Who has a different password for each web site?
\item Who has passwords shorter than 8 characters?
\item Who uses pass-phrases?
\end{itemize}
\pause
\begin{exampleblock}{What do you trade your password for?}
From a survey by Infosecurity~Europe~2003:
\begin{quote}
90\% of people gave away their passwords in exchange for a pen.
\end{quote}
\end{exampleblock}
\end{frame}

\subsection{Trade-off: security vs usability}

\begin{frame}
\frametitle{Security vs Usability}
\begin{itemize}
\item The system should be easy to use
\item Security measures are in the way of what the user wants to achieve
\end{itemize}
Security measures:
\begin{itemize}
\item Too many: make a web site unusable
\item Too few: the web site can be compromised
\end{itemize}
\begin{example}
\begin{center}
\begin{tabular}{rcl}
secure password & vs & simple password \\
authorization checks & vs & no authentication \\
encoded data & vs & plain text \\
\end{tabular}
\end{center}
\end{example}
\end{frame}

