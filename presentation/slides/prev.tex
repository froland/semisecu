\section{Prevention}
%------------------------------------------------
\subsection{Different moments}
%------------------------------------------------
\begin{frame}
\frametitle{Different Moments}

Preventing security flaws is something that has to be done during the
whole life of the application. The application life can be split in
three parts:

\begin{itemize}
\item Architecture
\item Development
\item Maintenance
\end{itemize}
\end{frame}

%------------------------------------------------
\begin{frame}
\frametitle{Architecture}
\begin{itemize}
\item Security has to be taken into account as soon as
  possible. Retrofitting security in an unsecure application is much
  more difficult than conceiving it securely from the start.
\item You have to ask yourself the good questions and those questions
  vary from one application to the other.
\item Know the threats you are facing.
\item There are no miracle receipts\ldots
\item \ldots just a few guidelines.
\end{itemize}
\end{frame}

%------------------------------------------------
\begin{frame}
\frametitle{Development: Implementation tricks}

There are lots of "tricks" that can make your application more
secure. Some of the more important are

\begin{itemize}
\item Passwords: enforcing a minimum complexity policy, storing them
  securely, have the user rotate them, prevent brute-force by
  implementing temporary lock-out, \ldots
\item Session management: Set an inactivity time out, set the
  "secure" flag when setting cookies, use the "HTTP-Only" flag for
  cookies, implement an easily accessible logout, \ldots
\item Input Validation: Always on server side, use a positive
  approach, search \emph{All} user inputs.
\item Use Ad-hoc protection for known attacks.
\end{itemize}
\end{frame}

%------------------------------------------------
\begin{frame}
\frametitle{Development: Secure development cycle}
A secure development cycle should set aside some time to do the
following:
\begin{itemize}
\item Security code review
\item Pentesting
\item Security training of the developers
\item Choose a secure and well maintained framework (more on
  frameworks later)
\item Devising a secure configuration for deployment
\end{itemize}

\end{frame}

%------------------------------------------------
\begin{frame}
\frametitle{Maintenance}
When vulnerabilities have been discovered they have to be patched as soon as
possible.

\begin{block}{Virtual Patching}
Virtual Patching is a security policy layer that prevents the exploitation of a known
vulnerability until a patch is released. It does so by filtering the
traffic that goes to and from the application server and blocking the
request that are exploits.
\end{block}

Moreover, care has to be taken when fixing a bug in a hurry not to open
new security breaches.

\end{frame}

%------------------------------------------------
\subsection{Using frameworks}
%------------------------------------------------

\begin{frame}
\frametitle{Using frameworks}

You do not develop the first web application. Frameworks already exists to
ease the development of your application while providing several security
mechanisms.

Often a framework will come with authentication support, escaping of values,
input validation, etc.

But you have to choose carefully because the choice if difficult to
change later on. Moreover bugs in the framework are better known so
they have to be patched has soon as possible which imply a rapid
reaction from the framework developer and from the maintaining team.
\end{frame}

%------------------------------------------------
\begin{frame}
\frametitle{Framework examples}

\begin{block}{Spring}
Spring is one of the most used framework for developing web
application using Java. It does feature many components including same
dedicated to security.
\end{block}

\begin{block}{ESAPI}
ESAPI (The OWASP Enterprise Security API) is a free, open source, web
application security control library that makes it easier for
programmers to write lower-risk applications. The ESAPI libraries are
designed to make it easier for programmers to retrofit security into
existing applications. The ESAPI libraries also serve as a solid
foundation for new development.
\end{block}

\end{frame}

%------------------------------------------------
\subsection{Security in application servers}
%------------------------------------------------
\begin{frame}
\frametitle{Security in application servers}
Application servers offer various services that can help you to configure
secured web applications.

Those typically are:
\begin{itemize}
\item Authentication management
\item directory listing disabling
\item user session management
\item error-handling
\item input encoding
\item \ldots
\end{itemize}
\end{frame}

%------------------------------------------------
\subsection{Application Firewall}
%------------------------------------------------
\begin{frame}
\frametitle{Application Firewall}
Application Firewall are filters which inspects the traffic that goes in and out your application and are able to block inappropriate content.

They provide an additional layer of protection, for example by scanning html forms for SQL content.

They can detect connection patterns and can be used to prevent non-resolved security holes to be exploited (virtual patching). 
\end{frame}

%------------------------------------------------
\subsection{HTTPS}
%------------------------------------------------
\begin{frame}
\frametitle{HTTPS}
TLS(Transport Layer Security) is the successor of SSL. Both are equal in terms of security, the main difference is that TLS begin with an insecure "hello" and switch to secured communications after the handshake, where SSL proceed directly to secured communications.\linebreak
HTTP over SSL/TLS allows the encryption and the authentication of an HTTP connection. It provides protection against eavesdropping and spoofing.\linebreak 
\textbf{HTTPS is a part of a solution, it does not ensure full security. }
\end{frame}

%------------------------------------------------
\subsection{Browser restrictions}
%------------------------------------------------
\begin{frame}
\frametitle{Browser restrictions}
\begin{itemize}
\item Cross-domain XMLHttpRequest(XHR) allows a website to contact other websites through the use of Javascript. Browsers disallow it but this feature can be selectively lifted by a website if it ask the browser via an "Access Control" header
\item Some browsers allow to block selectively the execution of Flash, Javascript to protect the user against malicious code hosted on websites
\end{itemize}
\end{frame}
\begin{frame}
\begin{itemize}
\item Private navigation mode of browsers prevents it from storing data
\item Cookies are stored on the computer of the user, they are limited by the URL and the timestamp
\item SSL Certificates are validates by the browser, which prompt the user whenever a dubious certificate is found. Some browsers also provide blacklist of known rogue sites and correct misspelled addresses to protect users from phishing.
\end{itemize}
\end{frame}

%------------------------------------------------
\subsection{E-mail}
%------------------------------------------------
\begin{frame}
\frametitle{E-mail}
E-mail is one of the most used application. It's plagued by spam and spoofing. \linebreak
Defensive measures can be taken:
\begin{itemize}
\item Sender Policy Framework(SPF) which validates emails by verifying sender IP Addresses, the DNS record of the domain contain which ip addresses ranges are allowed or forbidden to send mails for the considered domain \linebreak e.g. the SPF record of ietf.org\linebreak
\begin{quote}
ietf.org. IN SPF "v=spf1 ip4:64.170.98.0/26 ip4:64.170.98.64/28 ip4:64.170.98.80/28 ip6:2001:1890:1112:1::0/64 -all"
\end{quote}
\end{itemize}
\end{frame}
\begin{frame}
\begin{itemize}
\item DomainKeys Identified Mail(DKIM) associates domain names and email messages using a digital signature added to the message's header, a public key is added to the DNS record of the domain for the recipient to read the message's header
\textit{DKIM-Signature: v=1; a=rsa-sha256; d=example.net; s=brisbane;
c=relaxed/simple; q=dns/txt; l=1234; t=1117574938; x=1118006938;
h=from:to:subject:date:keywords:keywords;
bh=MTIzNDU2Nzg5MDEyMzQ1Njc4OTAxMjM0NTY3ODkwMTI=;
b=dzdVyOfAKCdLXdJOc9G2q8LoXSlEniSbav+yuU4zGeeruD00lszZ
VoG4ZHRNiYzR}
\end{itemize}
\end{frame}

%------------------------------------------------
\subsection{Variety of environments}
%------------------------------------------------
\begin{frame}
\frametitle{Variety of environments}
The variety of environments available to the users leads to difficulties when it comes to test possible setups for vulnerabilities. More, some browsers (e.g. IE6) require workarounds to display pages correctly, this extends the amount of code needed and thus increases vulnerability likeliness.
\end{frame}

