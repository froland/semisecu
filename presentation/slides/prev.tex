\section{Prevention}
%------------------------------------------------
\subsection{Different moments}
%------------------------------------------------
\begin{frame}
\frametitle{Different Moments}

Preventing security flaws is something that has to be done during the
whole life of the application. The application life can be split in
three parts:

\begin{itemize}
\item Architecture
\item Development
\item Maintenance
\end{itemize}
\end{frame}

%------------------------------------------------
\begin{frame}
\frametitle{Architecture}
\begin{itemize}
\item Security has to be taken into account as soon as
  possible. Retrofitting security in an insecure application is much
  more difficult than conceiving it securely from the start.
\item You have to ask yourself the good questions and those questions
  vary from one application to the other.
\item Know the threats you are facing.
\item There are no miracle receipts\ldots
\item \ldots just a few guidelines.
\end{itemize}
\end{frame}

%------------------------------------------------
\begin{frame}
\frametitle{Development: Implementation tricks}

There are lots of ``tricks'' that can make your application more
secure. Some of the more important are

\begin{itemize}
\item Passwords: enforcing a minimum complexity policy, storing them
  securely, have the user rotate them, prevent brute-force by
  implementing temporary lock-out, \ldots
\item Session management: Set an inactivity time out, set the
  ``secure'' flag when setting cookies, use the ``HTTP-Only'' flag for
  cookies, implement an easily accessible logout, \ldots
\item Input Validation: Always on server side, use a positive
  approach, search \emph{All} user inputs.
\item Use Ad-hoc protection for known attacks.
\end{itemize}
\end{frame}

%------------------------------------------------
\begin{frame}
\frametitle{Development: Secure development cycle}
A secure development cycle should set aside some time to do the
following:
\begin{itemize}
\item Security code review
\item Pentesting
\item Security training of the developers
\item Choose a secure and well maintained framework (more on
  frameworks later)
\item Devising a secure configuration for deployment
\end{itemize}

\end{frame}

%------------------------------------------------
\begin{frame}
\frametitle{Maintenance}
When vulnerabilities have been discovered they have to be patched as soon as
possible.

\begin{block}{Virtual Patching}
Virtual Patching is a security policy layer that prevents the exploitation of a known
vulnerability until a patch is released. It does so by filtering the
traffic that goes to and from the application server and blocking the
request that are exploits.
\end{block}

Moreover, care has to be taken when fixing a bug in a hurry not to open
new security breaches.

\end{frame}

%------------------------------------------------
\subsection{Using frameworks}
%------------------------------------------------

\begin{frame}
\frametitle{Using frameworks}

You do not develop the first web application. Frameworks already exists to
ease the development of your application while providing several security
mechanisms.

Often a framework will come with authentication support, escaping of values,
input validation, etc.

But you have to choose carefully because the choice if difficult to
change later on. Moreover bugs in the framework are better known so
they have to be patched has soon as possible which imply a rapid
reaction from the framework developer and from the maintaining team.
\end{frame}

%------------------------------------------------
\begin{frame}
\frametitle{Framework examples}

\begin{block}{Spring}
Spring is one of the most used framework for developing web
application using Java. It does feature many components including same
dedicated to security.
\end{block}

\begin{block}{ESAPI}
ESAPI (The OWASP Enterprise Security API) is a free, open source, web
application security control library that makes it easier for
programmers to write lower-risk applications. The ESAPI libraries are
designed to make it easier for programmers to retrofit security into
existing applications. The ESAPI libraries also serve as a solid
foundation for new development.
\end{block}

\end{frame}

%------------------------------------------------
\subsection{Security in application servers}
%------------------------------------------------
\begin{frame}
\frametitle{Security in application servers}
Application servers offer various services that can help you to configure
secured web applications.

Those typically are:
\begin{itemize}
\item Authentication management
\item directory listing disabling
\item user session management
\item error-handling
\item input encoding
\item \ldots
\end{itemize}
\end{frame}

