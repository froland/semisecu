\section{Detection and Analysis}

\subsection{Honeypot}
\begin{frame}
\frametitle{Honeypot}
\includeimage[width=0.8\linewidth,height=0.7\textheight]{honeypot}
\note{
A \textbf{honeypot} is a mocked website or web application that is exposed for
attackers who try to penetrate some system. A honeypot is a non-patched server
that is monitored for attacks and serves no other purpose. Thus, all
the traffic can be considered as an attack. The drawback of this
approach is that you have a vulnerable server that can be used as a
stepstone to attack other server in your network.
}
\end{frame}

\subsection{NIDS}
\begin{frame}
\centering
\Large
{\bfseries\usebeamercolor[fg]{block title}N}etwork
{\bfseries\usebeamercolor[fg]{block title}I}ntrusion
{\bfseries\usebeamercolor[fg]{block title}D}etection
{\bfseries\usebeamercolor[fg]{block title}S}ystem
\vspace{1em}
\includeimage[width=0.8\linewidth,height=0.7\textheight]{logo-snort}
\note{
\textbf{Network Intrusion Detection System} (NIDS) is a system that scans
network activity and tries to detect attacks.
\begin{exampleblock}{\href{http://www.snort.org/}{Snort}}
A NIDS, and an IPS (Intrusion Prevention System), snort can detect attacks and
take actions accordingly.
\end{exampleblock}
}
\end{frame}

\begin{frame}
\frametitle{NIDS - detection types}
\begin{itemize}
\Large
\item Signature-based detection
\item Statistical anomaly-based detection
\item Stateful protocol analysis
\end{itemize}
\note{
\begin{block}{Signature-based detection}
Finds a pattern in packets on the network.
Detects SQL injections in HTTP parameters, XSS attacks, etc.
\end{block}
\begin{block}{Statistical anomaly-based detection}
Looks for peaks in traffic, connection coming from a single host and other
anomalies.
Detects stealth port scans, brute force attacks, etc.
\end{block}
\begin{block}{Stateful protocol analysis}
Follows a protocol conversation.
Detects SYN port scans, etc.

A port scan or portscan is "An attack that sends client requests to a
range of server port addresses on a host, with the goal of finding an
active port and exploiting a known vulnerability of that service."
\url{http://encyclopedia.thefreedictionary.com/Port+scanner}

\end{block}
}
\end{frame}

\subsection{Post-mortem analysis}
\begin{frame}
\frametitle{After an attack}
\includeimage[width=0.8\linewidth,height=0.7\textheight]{investigation}
\note{
\begin{itemize}
\item \emph{When} the attack happened
	\begin{itemize}
	\item Were we attacked?
	\item Is the attack still in progress?
	\end{itemize}
\item \emph{What} was compromised
	\begin{itemize}
	\item Data was compromised?
	\item Data was modified?
	\item What systems failed to work?
	\end{itemize}
\item \emph{How} the attacker succeeded
	\begin{itemize}
	\item How to fix the problem?
	\item Prevent future attacks?
	\end{itemize}
\end{itemize}
}
\end{frame}

\begin{frame}
\frametitle{How to know?}
\begin{columns}
\begin{column}{0.5\textwidth}
\includeimage[width=\linewidth]{dos-load}
\end{column}
\begin{column}{0.5\textwidth}
\includeimage[width=\linewidth]{dos-bw}
\end{column}
\end{columns}
\note{
Read the logs and alerts.
\begin{exampleblock}{DoS}
Peak in server load and in input traffic.
Once the attack is effective, the outbound traffic drops.
\end{exampleblock}
}
\end{frame}

\begin{frame}
\frametitle{Analysis - tools}
\includeimage[width=0.8\linewidth,height=0.7\textheight]{analysis-tools}
\note{
\begin{itemize}
\item \href{http://www.ntop.org/products/ntop/}{ntop}: traffic monitoring
platform GPL
\item \href{http://www.intersectalliance.com}{Snare}: a collection of tools to
audit log data and facilitate log analysis NON FREE
\item \href{http://www.ossec.net/}{OSSEC}: open-source alternative; performs
log analysis, file integrity checking, policy monitoring, rootkit detection,
real-time alerting and active response
\item Web log analysis software
	\begin{itemize}
	\item \href{http://www.webalizer.org/}{Webalizer} GPL
	\item \href{http://awstats.sourceforge.net/}{AWStats} Open source
	\item \href{http://www.weblogexpert.com/}{WebLog Expert} NON FREE
	\end{itemize}
\end{itemize}
}
\end{frame}

\subsubsection{Noise as a diversion}

\begin{frame}
\frametitle{Noise as a diversion}
\includeimage[width=\linewidth,height=0.7\textheight]{firewall-log}
\note{
When an attack happens, it can be masked by another attack which would serve
as a diversion from the real attack.
\begin{exampleblock}{DoS as a diversion}
DoS generates lot of traffic and can mask attacks as generated logs are huge.
\end{exampleblock}
}
\end{frame}

\subsection{Performances}

\begin{frame}
\frametitle{Performances}
\includeimage[width=0.8\linewidth,height=0.7\textheight]{performance}
\note{
\begin{itemize}
\item Slow response time and high load might be a symptom of an attack
\item Slow application implies higher loads and may not scale and crash during
	an attack
\end{itemize}
}
\end{frame}

