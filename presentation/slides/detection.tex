\section{Detection and Analysis}

\subsection{Penetration testing}

\begin{frame}
\frametitle{Pentesting}
\includeimage[width=0.7\linewidth,height=0.7\textheight]{pentesting}
\note{
\textbf{Penetration testing} (or pentesting) is the action of attacking a
system with the consent of the owner in the goal of finding security holes.
\begin{itemize}
\item Identifying vulnerabilities
\item Exploit vulnerabilities
\item Testing ability to detect and respond to an attack
\item Social engineering
\end{itemize}
}
\end{frame}

\begin{frame}
\frametitle{Pentesting}
\begin{center}
Avoiding this situation.
\end{center}
\includeimage[height=0.6\textheight]{fence-fail}
\end{frame}

\begin{frame}
\frametitle{Tools}
\includeimage[width=0.7\linewidth,height=0.7\textheight]{bt5-screenshot}
\note{
\begin{itemize}
\item \href{http://nmap.org/}{nmap}:
	port scanning
\item \href{http://www.metasploit.com/about/penetration-testing-basics/}{Metasploit}:
	identifies vulnerabilities, automates assessments
\item \href{http://www.tenable.com/products/nessus}{Nessus}:
	vulnerability scanner
\item \href{https://www.owasp.org/index.php/Webscarab}{WebScarab}:
	intercepting proxy
\item \href{http://www.backtrack-linux.org/}{BackTrack}:
	Linux distribution
\end{itemize}
}
\end{frame}

\subsection{Honeypot}

\begin{frame}
\frametitle{Honeypot}
\includeimage[width=0.7\linewidth,height=0.7\textheight]{honeypot}
\note{
A \textbf{honeypot} is a mocked website or web application that is exposed for
attackers who try to penetrate some system. A honeypot is a non-patched server
that is monitored for attacks and serves no other purpose.
\begin{exampleblock}{\href{https://www.projecthoneypot.org/about_us.php}{Project Honey Pot}}
Anti-spammer project.
Web sites publish some fake e-mails and spammer's crawlers sending e-mail to
such addresses are blocked.
\end{exampleblock}
}
\end{frame}

\subsection{NIDS}

\begin{frame}
\frametitle{NIDS}
\includeimage[width=0.7\linewidth,height=0.7\textheight]{snort}
\note{
\textbf{Network Intrusion Detection System} (NIDS) is a system that scans
network activity and tries to detect attacks.
\begin{exampleblock}{\href{http://www.snort.org/}{Snort}}
A NIDS, and an IPS (Intrusion Prevention System), snort can detect attacks and
take actions accordingly.
\end{exampleblock}
}
\end{frame}

\begin{frame}
\frametitle{NIDS - detection types}
\begin{itemize}
\item Signature-based detection
\item Statistical anomaly-based detection
\item Stateful protocol analysis
\end{itemize}
\includeimage[width=0.4\linewidth,height=0.4\textheight]{detection}
\note{
\begin{block}{Signature-based detection}
Finds a pattern in packets on the network.
Detects SQL injections in HTTP parameters, XSS attacks, etc.
\end{block}
\begin{block}{Statistical anomaly-based detection}
Looks for peaks in traffic, connection coming from a single host and other
anomalies.
Detects stealth port scans, brute force attacks, etc.
\end{block}
\begin{block}{Stateful protocol analysis}
Follows a protocol conversation.
Detects SYN port scans, etc.
\end{block}
}
\end{frame}

\subsection{Post-mortem analysis}

\begin{frame}
\frametitle{After an attack}
\includeimage[width=0.7\linewidth,height=0.7\textheight]{investigation}
\note{
\begin{itemize}
\item \emph{When} did the attack happen
	\begin{itemize}
	\item Were we attacked?
	\item Is the attack still in progress?
	\end{itemize}
\item \emph{What} was compromised
	\begin{itemize}
	\item Data was compromised?
	\item Data was modified?
	\item What systems failed to work?
	\end{itemize}
\item \emph{How} the attacker succeeded
	\begin{itemize}
	\item How to fix the problem?
	\item Prevent future attacks?
	\end{itemize}
\end{itemize}
}
\end{frame}

\begin{frame}
\frametitle{How to know?}
\begin{itemize}
\item Traffic logs and graphs
\item Application logs
\end{itemize}
\begin{columns}
\begin{column}{0.5\textwidth}
\includeimage[width=\linewidth]{dos-load}
\end{column}
\begin{column}{0.5\textwidth}
\includeimage[width=\linewidth]{dos-bw}
\end{column}
\end{columns}
\end{frame}

\begin{frame}
\frametitle{Analysis - tools}
\includeimage[width=0.7\linewidth,height=0.7\textheight]{weblog}
\note{
\begin{itemize}
\item \href{http://www.ntop.org/products/ntop/}{ntop}: traffic monitoring
platform
\item \href{http://www.intersectalliance.com}{Snare}: a collection of tools to
audit log data and facilitate log analysis
\item \href{http://www.ossec.net/}{OSSEC}: open-source alternative; performs
log analysis, file integrity checking, policy monitoring, rootkit detection,
real-time alerting and active response
\item Web log analysis software
	\begin{itemize}
	\item \href{http://www.webalizer.org/}{Webalizer}
	\item \href{http://awstats.sourceforge.net/}{AWStats}
	\item \href{http://www.weblogexpert.com/}{WebLog Expert}
	\end{itemize}
\end{itemize}
}
\end{frame}

\subsubsection{Noise as a diversion}

\begin{frame}
\frametitle{Noise as a diversion}
\includeimage[width=0.7\linewidth,height=0.7\textheight]{train_cache}
\note{
When an attack happens, it can be masked by another attack which would serve
as a diversion from the real attack.
\begin{exampleblock}{DoS as a diversion}
DoS generates lot of traffic and can mask attacks as generated logs are huge.
\end{exampleblock}
}
\end{frame}

\subsection{Performances}

\begin{frame}
\frametitle{Performances}
\includeimage[width=0.7\linewidth,height=0.7\textheight]{performance}
\note{
\begin{itemize}
\item Slow response time and high load might be a symptom of an attack
\item Slow application implies higher loads and may not scale and crash during
	an attack
\end{itemize}
}
\end{frame}

