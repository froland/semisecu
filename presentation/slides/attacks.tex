\section{Attacks}

%%%%%%%%%%%%%%%%%%%%%%%%%%%%%%%%%%%%%%%%%%%%%%%%%%%%%%%%%%%%%%%%%%%%%%%%%%%%%%
\subsection{Attack trees}

\begin{frame}
\includeimage[width=\linewidth,height=0.8\textheight]{attack-tree-admin}
\note{
\begin{block}{What is it?}
Model describing how an \emph{asset} can be attacked.
\end{block}
\begin{block}{Goal}
Used to determine and understand threats that may arise.
An asset is vulnerable when there exists a path from the root to the leaf of
the tree.
\end{block}
}
\end{frame}

%%%%%%%%%%%%%%%%%%%%%%%%%%%%%%%%%%%%%%%%%%%%%%%%%%%%%%%%%%%%%%%%%%%%%%%%%%%%%%
\subsection{Threat agents}

\begin{frame}
\frametitle{Threat agents}
\begin{columns}
\begin{column}{0.6\linewidth}
\includeimage[width=\linewidth,height=4cm]{hacker}
\end{column}
\begin{column}{0.4\linewidth}
\includeimage[width=\linewidth,height=3cm]{yes-i-can}
\end{column}
\end{columns}
\note{
\begin{description}
\item[External threats]
	\hfill
	\begin{itemize}
	\item internet users {\small (and hackers)}
	\item viruses
	\end{itemize}
\item[Internal threats]
	Unsatisfied employees (intranet users)
\item[Reason]
	To show off, break things
\end{description}
}
\end{frame}

\begin{frame}
\frametitle{Threat agents}
\includeimage[height=0.8\textheight]{lock-virus}
\note{
\begin{description}
\item[External threats]
	\hfill
	\begin{itemize}
	\item organized crime
	\item competitor
	\end{itemize}
\item[Reason]
	Gain
	\begin{itemize}
	\item financial
	\item information
	\item competition
	\end{itemize}
\end{description}
\begin{block}{Impact}
Threat agents have different skills, resources and different motivations that
may influence risk.
\end{block}
\begin{exampleblock}{Lock virus}
Computer locked until you pay.
\end{exampleblock}
}
\end{frame}

\begin{frame}
\frametitle{Threat agents}
\begin{columns}
\begin{column}{0.4\linewidth}
\includeimage[width=\linewidth,height=4cm]{no-electricity}
\end{column}
\begin{column}{0.6\linewidth}
\includeimage[width=\linewidth,height=4cm]{power-plug}
\end{column}
\end{columns}
\note{
\begin{description}
\item[Internal threats]
	Employees, crew
\item[Technical failure]
	\hfill
	\begin{itemize}
	\item loss of essential services (electricity, etc.)
	\item hardware failure
	\item software failure
	\end{itemize}
\item[Reason]
	Unintentional, accident
\end{description}
}
\end{frame}

\begin{frame}
\frametitle{Threat agents}
\includeimage[width=\linewidth,height=0.75\textheight]{disaster}
\note{
\begin{description}
\item[Natural threats]
	(disasters)
	\newline Storms, floods, earthquakes, ...
	\newline Insignificant
\end{description}
}
\end{frame}

%%%%%%%%%%%%%%%%%%%%%%%%%%%%%%%%%%%%%%%%%%%%%%%%%%%%%%%%%%%%%%%%%%%%%%%%%%%%%%
\subsection{Risk}

\begin{frame}
\frametitle{What is a \emph{risk}?}
\begin{quote}
{\LARGE
The probable {\color{emphasis} frequency} and probable {\color{emphasis}
magnitude} of {\color{emphasis} future loss}.
}
\newline \textit{-- The Open Group}
\end{quote}
% Source:
% http://www.riskmanagementinsight.com/media/docs/FAIR_introduction.pdf
\note{
Just a good definition.
Explanation in the next slide.
}
\end{frame}

\begin{frame}
\begin{Huge}
\[ \text{risk} = \text{likelihood} * \text{impact} \]
\end{Huge}
\note{
\begin{itemize}
\item Likelihood: probability of a successful attack
	\\ Influenced by threat agent skills and vulnerability factors
\item Impact: how much damage the attack causes
\end{itemize}
\begin{columns}
\begin{column}{0.5\linewidth}
\par\textbf{Vulnerability factors}
\begin{itemize}
\item Ease of discovery
\item Ease of exploit
\item Awareness
\item Intrusion detection
\end{itemize}
\end{column}
\begin{column}{0.5\linewidth}
\par\textbf{Impact factors}
\begin{itemize}
\item Loss of confidentiality, integrity, availability, accountability
\item Financial damage
\item Reputation damage
\item Privacy violation
\end{itemize}
\end{column}
\end{columns}
% Explain in a KISS way
}
\end{frame}

%%%%%%%%%%%%%%%%%%%%%%%%%%%%%%%%%%%%%%%%%%%%%%%%%%%%%%%%%%%%%%%%%%%%%%%%%%%%%%
\subsection{Context of web applications}

\begin{frametransition}{How is the web structured?}
\end{frametransition}

\subsubsection{Network}
\begin{frame}
\includeimage[width=\linewidth,height=0.75\textheight]{client-server}
\note{
\begin{itemize}
\item Applications available on the internet
\item Anyone can access the application
\end{itemize}
}
\end{frame}

\subsubsection{Architecture of a web application}

\begin{frame}
\includeimage[width=\linewidth]{archi-webapplication}
\note{
\begin{itemize}
\item Browser: the client, uses scripting
\item Router: connected to the browser (by WiFi) - LAN
\item Internet: not detailed here - WAN
\item DMZ: demilitarized zone {\small (exposed subnetwork)}
\item Web Application Firewall, IDS, Load Balancer
\item Application Server: with the contained application
\end{itemize}
\begin{block}{Points of failure}
Any component may fail; will the web site continue to work?
\par It is possible to intervene at multiple levels.
\begin{description}
\item[Single point of failure]
A part of a system that prevents the entire system from working when it fails.
Solution: duplication.
\end{description}
\end{block}
}
\end{frame}

%%%%%%%%%%%%%%%%%%%%%%%%%%%%%%%%%%%%%%%%%%%%%%%%%%%%%%%%%%%%%%%%%%%%%%%%%%%%%%
\subsection{OWASP - Top 10}

\begin{frame}
\begin{center}
{\LARGE
{\bfseries\usebeamercolor[fg]{block title}O}pen
{\bfseries\usebeamercolor[fg]{block title}W}eb
{\bfseries\usebeamercolor[fg]{block title}A}pplication
{\bfseries\usebeamercolor[fg]{block title}S}ecurity
{\bfseries\usebeamercolor[fg]{block title}P}roject
}
\includeimage[width=0.5\linewidth,height=23mm]{logo-owasp}
\url{https://www.owasp.org/}
\end{center}
\note{
\begin{block}{OWASP}
\begin{itemize}
\item Standards
\item Libraries
\item Books
\item Etc.
\end{itemize}
\end{block}
}
\end{frame}

\begin{frame}
\frametitle{Top 10}
\begin{enumerate}
\item Injection
\item Cross-site Scripting (XSS)
\item Authentication and Session Management
\item Insecure Direct Object References
\item Cross-site Request Forgery (CSRF)
\item Security Misconfiguration
\item Insecure Cryptographic Storage
\item Failure to Restrict URL Access
\item Insufficient Transport Layer Protection
\item Unvalidated Forwards and Redirects
\end{enumerate}
\note{
\begin{block}{Project: Top 10}
OWASP compiles a top 10 of most critical web application \emph{risks}.
\vspace{2em}
\begin{center}
\LARGE
\par Insist on risk!
\par Attacks are complementary!
\end{center}
\end{block}
}
\end{frame}

\begin{frame}
\frametitle{Demo website}
\begin{description}
\Large
\item[Technologies]
	Java website
	\hfill
	\begin{itemize}
	\Large
	\item Spring
	\item Hibernate - h2database
	\end{itemize}
\end{description}
\note{
Used technologies, etc. for the demo.
\begin{block}{Generality}
The issues presented afterwards are common to many technologies.
\end{block}
\begin{exampleblock}{The demo}
Run the normal usage demo.
\end{exampleblock}
\begin{block}{Demo application}
\begin{itemize}
\item Built with security flaws for the demo
\item Hotel rating web site
\item Shown hotels must be approved by an administrator
\end{itemize}
\end{block}
}
\end{frame}

\subsubsection{Attack 1 - Injection}

\begin{frame}
\frametitle{1. Injection}
\emphsidetext{widespread \newline severe impact}
\includeimage[scale=0.5]{client-server-injection}
\note{
\textbf{Injection} consists in sending untrusted data to an interpreter.
Attacker enters code to execute into input fields.
\begin{example}
Send code directly to some interpreter.
\end{example}
\begin{block}{Impacts}
Data loss, corruption, lack of accountability, etc.
\end{block}
\begin{alertblock}{Risks}
Easy to exploit, has a severe impact, it is quite common.
\end{alertblock}
}
\end{frame}

\begin{frame}
\begin{exampleblock}{\emph{THE} password}
\includeimage[height=0.4\textheight]{the-password}
\end{exampleblock}
\note{
Explanation in the following slide.
}
\end{frame}

\defverbatim[colored]\LstBad{
\begin{lstlisting}[style=beamer]
String query = "select * from users"
	+ " where user_name = '" @+ name +@ "'"
	+ " and password = '" @+ password +@ "'";
\end{lstlisting}
}
\defverbatim[colored]\LstGood{
\begin{lstlisting}[style=beamer]
String query = "select * from users"
	+ " where @user_name = ?@"
	+ " and @password = ?@";
PreparedStatement st = con.prepareStatement(query);
@st.setString(1, name);@
@st.setString(2, password);@
ResultSet rs = st.executeQuery();
\end{lstlisting}
}
\begin{frame}
\begin{embeddedcode}{bad}
\LstBad
\end{embeddedcode}
\badgoodsep
\begin{itemize}
\Large
\item Parameterized interface
\end{itemize}
\begin{embeddedcode}{good}
\LstGood
\end{embeddedcode}
\note{
%\begin{exampleblock}{SQL injection - demo}
%Getting the users' passwords as administrator.
%\end{exampleblock}
\begin{exampleblock}{JPQL injection - demo}
Searching hotels that have a manager that has the same password as a given
user.
\end{exampleblock}
The \emph{concatenation} is the cause of the injection: the query
is built from unvalidated input.
\begin{block}{Using parametrized interface}
Usage of placeholders for values let's the interpreter escape and validate
input values.
\end{block}
}
\end{frame}

\begin{frame}
\frametitle{Other Countermeasures}
\begin{columns}
\begin{column}{0.6\linewidth}
\begin{itemize}
\Large
\item Escaping routines
\item White list validation
\end{itemize}
\end{column}
\begin{column}{0.4\linewidth}
\includeimage[width=\linewidth]{exit}
\end{column}
\end{columns}
\note{
\begin{block}{Escaping routines}
Some languages have special functions to escape manually values.
\end{block}
\begin{block}{White list validation}
List of valid input patterns.
\end{block}
}
\end{frame}

\subsubsection{Attack 2 - Cross-site scripting (XSS)}

\begin{frame}
\frametitle{2. Cross-site scripting (XSS)}
\emphsidetext{very widespread}
\includeimage[scale=0.5]{client-server-xss}
\note{
\textbf{Cross-site scripting} allows attackers to inject code into the page
sent to the user. The injected code can be a script interpreted by the
browser.
\begin{example}
The attacker sends a script to a web page. Users who download the code execute
it in their browser.
\end{example}
\begin{block}{Impacts}
Hijack user sessions, change content, redirect the user.
\end{block}
\begin{alertblock}{Risks}
The most widespread vulnerability.
It requires an average knowledge to be exploited and the impacts are moderate.
\end{alertblock}
}
\end{frame}

\defverbatim[colored]\LstBad{
\begin{lstlisting}[style=beamer]
<div>
	your content goes here
</div>
\end{lstlisting}
}
\defverbatim[colored]\LstGood{
\begin{lstlisting}[style=beamer]
<div>
	@escape(@your content goes here@)@
</div>
\end{lstlisting}
}
\begin{frame}
\begin{embeddedcode}{bad}
\LstBad
\end{embeddedcode}
\pause
\begin{center}
You can inject a part of the page.
\end{center}
\badgoodsep
\begin{embeddedcode}{good}
\LstGood
\end{embeddedcode}
\note{
Text included directly in a page as is.
\begin{exampleblock}{Getting the user's cookies - demo}
Just use \lstinline!document.cookie! in JavaScript and send it to
\emph{another server}.
\newline \textbf{Introduce the other server!}
\end{exampleblock}
\begin{block}{Escaping the values}
All the values should be escaped before sending them to the users.
You could also use a light markup language.
\end{block}
\begin{block}{White list validation}
Output values can be white listed, but this is not a complete defence against
XSS as sometimes special characters must be accepted.
\end{block}
}
\end{frame}

\subsubsection{Attack 3 - Authentication and Session Management}

\begin{frame}
\frametitle{3. Authentication and Session Management}
\emphsidetext{severe impact}
\includeimage[scale=0.5]{client-server-session}
\note{
The attacker uses flaws in the \textbf{authentication or session management}
implemented for a given web site to steal someone else's identity.
\begin{example}
\begin{itemize}
\item Session hijacking
\item Session fixation
\end{itemize}
\end{example}
\begin{block}{Impacts}
Once an account has been stolen, the attacker may do \emph{anything} the user
can do.
\end{block}
\begin{alertblock}{Risks}
The impact is severe, this attack is common.
\end{alertblock}
}
\end{frame}

\defverbatim[colored]\Lst{
\begin{minipage}{0.5\linewidth}
\begin{lstlisting}[style=beamer,basicstyle=\ttfamily\LARGE,text-align=center]
  http://host.com/page@;
  jsessionid=dfd4fa35df2@...
\end{lstlisting}
\end{minipage}
}
\begin{frame}
\frametitle{Session Stealing}
\Lst
\vspace{3mm}
\centering\Large\bfseries
cross-site scripting (XSS)
\badgoodsep
\begin{center}
\huge\bfseries
Good authentication mechanism
\end{center}
\note{
\begin{example}
\begin{small}
\begin{itemize}
\item Public computer:
	Forgetting to log out...
\item Session fixation:
	In older application servers, the session ID could be set using the URL.
	You could send that URL to someone else.
\item Session guessing:
	Guess the next generated session ID.
\item XSS session hijacking (demo):
	The user's session can be obtained using XSS.
\end{itemize}
\end{small}
\end{example}
\badgoodsep
\begin{block}{Good authentication mechanism}
\begin{itemize}
\item \small Use a proven authentication mechanism
\item Prevent \emph{XSS attacks}
\item Good session timeouts and accessible log out buttons
\end{itemize}
\end{block}
}
\end{frame}

\subsubsection{Attack 4 - Insecure Direct Object References}

\defverbatim[colored]\Lst{
\begin{lstlisting}[style=beamer,basicstyle=\ttfamily\LARGE]
http://myhotel.com/user/@152@
\end{lstlisting}
}
\begin{frame}
\frametitle{4. Insecure Direct Object References}
\emphsidetext{easy to exploit}
\begin{columns}
\begin{column}{0.1\linewidth}\end{column}
\begin{column}{0.7\linewidth}
\Lst
\end{column}
\end{columns}
\badgoodsep
\begin{center}
{\huge\bfseries Check permissions}
\end{center}
\note{
\small
The attacker changes a parameter in the request to obtain a \textbf{direct
object reference} that they should not be able to access.
\begin{example}
Replace some object identifier in the request.
\end{example}
\begin{block}{Impacts}
Compromise the data that can be referenced.
\end{block}
\begin{alertblock}{Risks}
Easy to exploit and detect. The impact is moderate.
\end{alertblock}
\hrule
\begin{itemize}
\item An unvalidated hotel - demo: See an unvalidated hotel as a normal user.
\item Check permissions: Check whether the user has the permissions to access
the object before manipulating it.
\item References by session: Use indirect references for each session.
\end{itemize}
}
\end{frame}

\subsubsection{Attack 5 - Cross-site Request Forgery (CSRF)}

\begin{frame}
\frametitle{5. Cross-site Request Forgery (CSRF)}
\emphsidetext{widespread}
\includeimage[scale=0.5]{client-server-csrf}
\note{
\textbf{Cross-site Request Forgery} consists in generating requests when the
user visits the attacker's web page.
\begin{example}
\begin{itemize}
\item A user visits the attacker's web page
\item The requested page contains tricks to submit requests to the attacked
web site as the visiting user
\end{itemize}
\end{example}
\begin{block}{Impacts}
The attacker may perform actions as the victim.
\end{block}
\begin{alertblock}{Risks}
Widespread and easy to detect, the impact is moderate.
\end{alertblock}
}
\end{frame}

\defverbatim[colored]\LstBad{
\begin{lstlisting}[style=beamer]
<img src="http://@host.com@/forum/thread1/
@addComment@?text=hello%20world" alt="" />
\end{lstlisting}
}
\defverbatim[colored]\LstGood{
\begin{lstlisting}[style=beamer]
<form @method="post"@ action="addComment">
	<input @type="hidden"@ name="token" @value="sd5646sdfse8wd"@ />
	...
</form>
\end{lstlisting}
}
\begin{frame}
\begin{embeddedcode}{bad}
\LstBad
\end{embeddedcode}
\badgoodsep
\begin{embeddedcode}{good}
\LstGood
\end{embeddedcode}
\note{
\begin{exampleblock}{Post a comment}
The attacker's page may contain a URL that executes an \emph{action}.
\end{exampleblock}
\hrule
\begin{block}{Unpredictable token}
Each request contains a field that is a token linked to the session of
the user, the attacker's website will not be able to generate the correct
requests.
\newline
Tokens can be unique by session or by request.
\par
The token is usually stored in a cookie (preferably HTTP-only).
In forms, a hidden field contains the same value.
When the request is sent, both values are compared.
\par
Only the session ID is stored in the cookie; the session information is
stored only on the application server.
\end{block}
}
\end{frame}

\subsubsection{Attack 6 - Security Misconfiguration}

\begin{frame}
\frametitle{6. Security Misconfiguration}
\emphsidetext{easy to exploit}
\includeimage[width=0.7\linewidth]{misconfiguration}
\note{
\small
\textbf{Security configuration} can happen at any level: the application, the
web server, the framework, etc.
The default behavior is not always the wanted one.
\begin{example}
\begin{itemize}
\item \small Accessing: default accounts, unprotected files and directories
\item Gaining information: versions, stack traces, debug messages
\end{itemize}
\end{example}
\begin{block}{Impacts}
Attackers can gain unauthorized access, the system can be compromised.
\end{block}
\begin{alertblock}{Risks}
Easy to exploit and to detect.
\end{alertblock}
}
\end{frame}

\begin{frame}
\includeimage[height=0.7\textheight,width=\linewidth]{out-of-the-box}
\note{
The \emph{out-of-the-box} configuration is not always the best.
}
\end{frame}

\begin{frame}
\includeimage[width=\linewidth,height=0.7\textheight]{webpage-stacktrace}
\pause
\raggedleft\Large
... more information for the attacker
\note{
\begin{block}{Outdated libraries}
Bugs are published for known libraries and applications.
\end{block}
\begin{exampleblock}{Stack trace}
A stack trace may contain the used parameters, references to libraries, etc.
\end{exampleblock}
\hrule
\begin{block}{Up-to-date system}
Keep the used software up to date and manage versions.
\end{block}
\begin{block}{Good architecture}
Good separation between components is more secure.
\end{block}
\begin{block}{Automated scan}
Automated scans can detect misconfigurations and missing patches.
\end{block}
}
\end{frame}

\subsubsection{Attack 7 - Insecure Cryptographic Storage}

\begin{frame}
\frametitle{7. Insecure Cryptographic Storage}
\emphsidetext{compromised data}
\begin{center}
\begin{tabular}{|l|l|l|c|}
\hline
ID & NAME & PASSWORD & \ldots \\
\hline
1 & admin & 2ezf5zf2d & \ldots \\
2 & jacky & toto & \ldots \\
3 & k & 123456 & \ldots \\
\hline
\end{tabular}
\end{center}
\note{
Sensitive data is stored \textbf{insecurely without crypting}.
\begin{example}
All data is readable by the attacker once he gains access to it.
\end{example}
\begin{block}{Impacts}
Stolen data is compromised.
\end{block}
\begin{alertblock}{Risks}
All unencrypted data is compromised.
The attack is however difficult to perform.
\end{alertblock}
}
\end{frame}

\begin{frame}
\includeimage[width=\textwidth]{rainbow-table-example}
\note{
\begin{exampleblock}{Unencrypted passwords}
After dumping the table \texttt{USERS}, the passwords might be stored in
plain text.
\end{exampleblock}
\begin{exampleblock}{\textbf{Rainbow tables}}
Precomputed table for decoding hashed values.
\href{http://www.md5this.com/list.php?page=133&key=1}{\texttt{md5this.com}}
\end{exampleblock}
}
\end{frame}

\begin{frame}
\includeimage[height=0.6\textheight,width=0.8\linewidth]{backup}
\note{
\begin{exampleblock}{Unencrypted backup}
The database backup is lost.
\end{exampleblock}
}
\end{frame}

\subsubsection{Attack 8 - Failure to Restrict URL Access}

\defverbatim[colored]\Lst{
\begin{lstlisting}[style=beamer,basicstyle=\ttfamily\LARGE]
http://myhotel.com/@admin@
\end{lstlisting}
}
\begin{frame}
\frametitle{8. Failure to Restrict URL Access}
\emphsidetext{easy to detect}
\begin{columns}
\begin{column}{0.1\linewidth}\end{column}
\begin{column}{0.7\linewidth}
\Lst
\end{column}
\end{columns}
\badgoodsep
\begin{center}
\huge\bfseries
Authorization
\end{center}
\note{
\textbf{URL access is not restricted} to functions requiring permissions.
\begin{example}
The attacker changes his URL to a privileged page.
\end{example}
\begin{block}{Impacts}
Administrative functions may become available to the attacker.
\end{block}
\begin{alertblock}{Risks}
Easy to exploit, but the flaw is uncommon.
\end{alertblock}
\begin{exampleblock}{Administration page}
\url{http://myhotel.com/admin} is not referenced from any other page, but
the authentication is not enforced.
\end{exampleblock}
\hrule
\begin{block}{Check permissions}
Authorization mechanisms should be used.
\end{block}
}
\end{frame}

\subsubsection{Attack 9 - Insufficient Transport Layer Protection}

\begin{frame}
\frametitle{9. Insufficient Transport Layer Protection}
\emphsidetext{easy to detect}
\includeimage[width=0.8\linewidth]{wifi-sniffing}
\note{
\textbf{Traffic is unprotected} and anyone sniffing the network can see the
data in transit.
\begin{example}
\begin{itemize}
\item Sniffing the WiFi network
\item Scanning for HTTP requests
\item Analyzing conversations
\end{itemize}
\end{example}
\begin{block}{Impacts}
Account theft, expose individual user's data.
\end{block}
\begin{alertblock}{Risks}
Detectable using any sniffer.
However, often difficult to exploit.
\end{alertblock}
}
\end{frame}

\begin{frame}
\includeimage[width=0.5\linewidth]{https}
\note{
The attacker can see the data that is sent over the network.
\begin{exampleblock}{Sniffing passwords}
Sniff the network and find HTTP requests containing "password".
\end{exampleblock}
\hrule
\begin{block}{HTTP\textbf{S}}
SSL/TLS can be used to secure the connection.
\begin{itemize}
\item \emph{secure} cookies will not be sent through an unencrypted connection
\item Ensure that certificates are valid
\end{itemize}
\end{block}
}
\end{frame}

\subsubsection{Attack 10 - Unvalidated Forwards and Redirects}

\defverbatim[colored]\Lst{
\begin{lstlisting}[style=beamer,basicstyle=\ttfamily\Large]
http://host.com/forward.jsp?
@fwd=admin.jsp@
\end{lstlisting}
}
\begin{frame}
\frametitle{10. Unvalidated \textbf{Forwards} and Redirects}
\emphsidetext{easy to detect}
\includeimage[width=\linewidth,height=6cm]{webapp-forward}
\Lst
\note{
A URL may appear as trustworthy, but it contains an \textbf{unvalidated
forward or redirect}.
\begin{example}
The attacker may bypass authorization for \texttt{admin.jsp}.
\end{example}
\begin{block}{Impacts}
Tricking victims into disclosing information, unsafe forwards may allow access
control bypass.
\end{block}
\begin{alertblock}{Risks}
Easy to detect, uncommonly present.
\end{alertblock}
}
\end{frame}

\defverbatim[colored]\Lst{
\begin{lstlisting}[style=beamer,basicstyle=\ttfamily\large]
https://@bank.com/secured@/../redirect?
@url=http://evil-bank.com/secured@
\end{lstlisting}
}
\begin{frame}
\frametitle{10. Unvalidated Forwards and \textbf{Redirects}}
\emphsidetext{easy to detect}
\includeimage[width=\linewidth,height=6cm]{webapp-redirect}
\Lst
\note{
\begin{example}
Placing a URL with a redirect that seems trustworthy redirecting to a
malicious page.
\end{example}
\begin{exampleblock}{Malicious site}
Redirect to a malicious website using:
\\ \url{http://bank.com/redirect?url=evil-bank.com}
\end{exampleblock}
\hrule
\begin{block}{Avoid redirects and forwards}
Avoid \emph{parametrized} redirects and forwards.
\end{block}
\begin{block}{Authorization}
Validate and authorize the user after each redirect.
\end{block}
\begin{exampleblock}{Map input values}
Map input values to the actual URLs.
\end{exampleblock}
}
\end{frame}

%%%%%%%%%%%%%%%%%%%%%%%%%%%%%%%%%%%%%%%%%%%%%%%%%%%%%%%%%%%%%%%%%%%%%%%%%%%%%%

\begin{frame}
\frametitle{Some other attacks}
\begin{itemize}
\item Crashing a system
\item Altering content
\item Social engineering
\item Breaking any password
\end{itemize}
\note{
Intent for some other attacks explained just after this slide.
}
\end{frame}

%%%%%%%%%%%%%%%%%%%%%%%%%%%%%%%%%%%%%%%%%%%%%%%%%%%%%%%%%%%%%%%%%%%%%%%%%%%%%%
\subsection{DoS}

\begin{frametransition}{What about just crashing the system?}
\end{frametransition}

\begin{frame}
\frametitle{Denial of Service (DoS)}
\includeimage[width=0.8\linewidth,height=0.8\textheight]{overwhelmed-donkey}
\note{
Attack that aims to bring down (or make unavailable) some service usually by
causing server overload.
\begin{quote}
An \emph{important load} causes \emph{service overload}.
\end{quote}
FYI:
\begin{itemize}
\item SYN flood
\item Application level flood
\item R-U-Dead-Yet
	\newline
	(never-ending POST transmissions,
	example: \href{http://ha.ckers.org/slowloris/}{Slowloris})
\item DDoS
\item Unintentional DoS
\end{itemize}
}
\end{frame}

\begin{frame}
\frametitle{DoS attack types}
\begin{columns}
\begin{column}{0.5\linewidth}
\includeimage[width=\linewidth,height=0.5\textheight]{anonymous-group}
\end{column}
\pause
\begin{column}{0.5\linewidth}
\includeimage[width=\linewidth]{dos-unintentional}
\end{column}
\end{columns}
\note{
\frametitle{Denial of Service - example}
\begin{exampleblock}{DDoS}
The anonymous group attack.
\end{exampleblock}
\begin{exampleblock}{Unintentional DoS}
\begin{itemize}
\item page linked from \url{http://slashdot.org}
\item \href{http://www.joueurdugrenier.fr/}{Joueur du Grenier} - new video
\end{itemize}
\end{exampleblock}
\hrule
\begin{block}{Denial of Service - prevention}
\begin{itemize}
\item Firewall
\item Intrusion Prevention System
\item Black-holing
\end{itemize}
\end{block}
}
\end{frame}

%%%%%%%%%%%%%%%%%%%%%%%%%%%%%%%%%%%%%%%%%%%%%%%%%%%%%%%%%%%%%%%%%%%%%%%%%%%%%%
\subsection{MitM}

\begin{frametransition}{And altering the content?}
\end{frametransition}

\begin{frame}
\frametitle{Man in the Middle attack}
\includeimage[height=0.6\textheight]{phone-tapping}
\note{
The attacker plays the role of the server to the client and the role of the
client to the server.
All communication transits through the attacker who can modify messages as
they are sent.
% In french: "espionner le contenu et même l'altérer"
\begin{exampleblock}{Public WiFi}
Vulnerable on public WiFi networks.
\end{exampleblock}
\begin{exampleblock}{ARP poisoning}
Just become the router.
\\ Tools such as \texttt{ettercap} allow to perform the attack.
\end{exampleblock}
\begin{exampleblock}{Common example}
Submitting a bank transfer request and modifying the parameters.
\end{exampleblock}
\hrule
\begin{block}{Prevention}
Use cryptography: HTTPS, PKI (Public Key Infrastructure)
\end{block}
}
\end{frame}

\begin{frame}
\includeimage[width=\linewidth]{archi-webapplication}
\note{
You can attack anywhere in this schema!
}
\end{frame}

%%%%%%%%%%%%%%%%%%%%%%%%%%%%%%%%%%%%%%%%%%%%%%%%%%%%%%%%%%%%%%%%%%%%%%%%%%%%%%
\subsection{Social engineering}

\begin{frame}
\frametitle{Social engineering}
%\includeimage[width=0.8\linewidth]{social-engineering}
\includeimage[width=0.8\linewidth]{networking-sites}
\note{
\textbf{Social engineering} consists in manipulating people so they divulge
confidential information.
\\ Similar to fraud.
}
\end{frame}

\begin{frame}
\includeimage[height=0.75\textheight,width=\linewidth]{xkcd538_security}
\note{
Recall how you can break strong encryption.
\begin{block}{E-mail}
Once the e-mail account has been hacked, you can reset the password for other
services that would send an e-mail when recovering a lost password.
\par
A secret question could be targeted by social engineering.
\end{block}
}
\end{frame}

\begin{frame}
\includeimage[width=0.8\linewidth,height=0.6\textheight]{hannibal-lecter}
\begin{quote}
\begin{center}
Quid pro quo, Clarice...
\end{center}
\end{quote}
\note{
\begin{description}
\item[Pretexting]
Using an invented scenario (the pretext) to engage a targeted victim in
a manner that increases the chance the victim will divulge information
or perform actions that would be unlikely in ordinary circumstances.
\item[Phishing]
See the next point.
\item[Quid pro quo]
Something for something:
Help a user solve a problem, they might give you passwords in exchange.
\end{description}
\begin{block}{Story: Silence of the Lambs}
Hannibal Lecter helps the FBI agent (Clarice) in exchange for being transfered
to a less secure prison from which he can escape.
\end{block}
\begin{block}{Protection}
\begin{itemize}
\item Rising awareness % In french: "sensibilisation"
\item Framework of trust
\end{itemize}
\end{block}
}
\end{frame}

%%%%%%%%%%%%%%%%%%%%%%%%%%%%%%%%%%%%%%%%%%%%%%%%%%%%%%%%%%%%%%%%%%%%%%%%%%%%%%
\subsubsection{Phishing}

\begin{frame}
\frametitle{Phishing}
\includeimage[height=0.75\textheight,width=\linewidth]{phishing-mail}
\note{
\textbf{Phishing} is the attempt to acquire user information through
masquerading as a legitimate web site.
\begin{itemize}
\item Cloned websites
\item Link manipulation
\item Phone phishing
\item Etc.
\end{itemize}
\hrule
\begin{block}{Phishing - prevention}
\begin{itemize}
\item Education
\item Help to identify legitimate web sites
\end{itemize}
\end{block}
}
\end{frame}

\begin{frame}
\includeimage[height=0.75\textheight,width=\linewidth]{cloned-website}
\note{
A \emph{cloned} website.
}
\end{frame}

%%%%%%%%%%%%%%%%%%%%%%%%%%%%%%%%%%%%%%%%%%%%%%%%%%%%%%%%%%%%%%%%%%%%%%%%%%%%%%
\subsection{Other attacks}

\begin{frametransition}{Breaking a password?}
\end{frametransition}

\begin{frame}
\frametitle{Brute force}
%\includeimage[height=0.7\textheight]{letstryallthepasswords}
\includeimage[height=0.7\textheight]{hulk-smash}
\note{
\begin{block}{Try all the passwords}
Try all the possible combinations.
\newline
$k^n$ where $k$ is the size of the alphabet and $n$ the length of the
password.
\end{block}
\begin{block}{Dictionary attack}
Try a list of common passwords.
\\ English language contains only hundreds of thousands words.
\end{block}
}
\end{frame}

%\defverbatim[colored]\Lst{
%\begin{lstlisting}[style=beamer]
%http://host.com/forum/download.jsp
%?@path=/etc/shadow@
%\end{lstlisting}
%}
%\begin{frame}
%\frametitle{Path traversal}
%\Lst
%\includeimage[scale=0.3]{etc-shadow}
%\note{
%An input is a relative unvalidated path.
%You can download or upload any file that the server contains.
%\begin{exampleblock}{Download form}
%Get a file uploaded to a forum... or the password file.
%\end{exampleblock}
%}
%\end{frame}

%\begin{frame}
%\frametitle{Failure to restrict automation}
%\begin{itemize}
%\item<+-> Spamming
%\item<+-> REST API
%\end{itemize}
%\pause
%\vspace{1cm}
%\includeimage[width=0.6\linewidth]{recaptcha-example}
%\note{
%The web site is not protected against automation.
%\begin{itemize}
%\item Common for REST API
%\item Programs may overload the website unintentionally
%    \newline
%    (Fix: maximum number of requests per IP for a time period)
%\end{itemize}
%\begin{exampleblock}{Captcha - prevention}
%Challenge-response test easy for humans, difficult for machines.
%\end{exampleblock}
%\begin{block}{Captcha - cracking}
%The attacker has to make a website with high traffic on it (image boards,...). Once done, the evil website can present a captcha from another website. The users visiting the evil website answer the captcha and the evil website can use this answer to crack the protection of the attacked website.
%\end{block}
%}
%\end{frame}

\begin{frame}
\frametitle{And so on...}
\includeimage[width=0.9\linewidth,height=0.85\textheight]{attack-tag-cloud}
\note{
As technology evolves, new forms of attacks appear.
\par
Instead of trying to fix vulnerabilities, focus on establishing strong
security controls.
}
\end{frame}

%%%%%%%%%%%%%%%%%%%%%%%%%%%%%%%%%%%%%%%%%%%%%%%%%%%%%%%%%%%%%%%%%%%%%%%%%%%%%%
\subsection{Common points}

\begin{frame}[t]
\frametitle{Common points}
\vspace{2em}
\begin{itemize}
\item<2-> Improper input validation
	\only<3-6>{
	\begin{itemize}
	\item<3-> Injection
	\item<4-> Cross-site scripting (XSS)
	\item<5-> Session Stealing via XSS
	\item<6-> Unvalidated Forwards and Redirects
	\end{itemize}
	}
\item<7-> Missing authorization checks
	\only<8-9>{
	\begin{itemize}
	\item<8-> Insecure Direct Object References
	\item<9-> Failure to Restrict URL Access
	\end{itemize}
	}
\item<10-> Misconfiguration
\item<11-> Other assumptions
	\only<12-15>{
	\begin{itemize}
	\item<12-> Insecure Cryptographic Storage
	\item<13-> Insufficient Transport Layer Protection
	\item<14-> Denial of Service
	\item<15-> Man in the Middle
	\end{itemize}
	}
\item<16-> Education factor {\small (social engineering)}
	\only<16-20>{
	\begin{itemize}
	\item<17-> Session stealing (cybercafe)
	\item<18-> Social engineering
	\item<19-> Phishing
	\item<20-> Brute Force
	\end{itemize}
	}
\end{itemize}
\visible<21>{} % just to add a pause before ending the chapter
\note{
Bullet point
% In French: "Pour lutter contre le social engineering, il est bon de
% sensibiliser vos utilisateurs."
}
\end{frame}
