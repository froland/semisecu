\section{Attacks}

%%%%%%%%%%%%%%%%%%%%%%%%%%%%%%%%%%%%%%%%%%%%%%%%%%%%%%%%%%%%%%%%%%%%%%%%%%%%%%
\subsection{Attack trees}

\begin{frame}
\frametitle{An attack tree}

\begin{block}{What is it}
Model describing by what means an \emph{asset} can be attacked.
\end{block}
\begin{block}{Goal}
Used to determine and understand threats that may arise.
\end{block}
\end{frame}

\begin{frame}{Example}
\includeimage[width=0.8\linewidth,height=6cm]{attack-tree-admin}
\end{frame}

%%%%%%%%%%%%%%%%%%%%%%%%%%%%%%%%%%%%%%%%%%%%%%%%%%%%%%%%%%%%%%%%%%%%%%%%%%%%%%
\subsection{Threat agents}

\begin{frame}
\frametitle{Who are they?}
\begin{itemize}
\item External threats
	\begin{itemize}
	\item internet users
	\item viruses
	\item organized crime
	\end{itemize}
\item Internal threats
	\begin{itemize}
	\item employees
    \item intranet users
    \item administrators
    \end{itemize}
\item Natural threats (disasters)
\item Technical failures
	\begin{itemize}
	\item loss of essential services
		% (example: no electricity)
	\item hardware failure
	\item software failure
	\end{itemize}
\end{itemize}
\end{frame}

\begin{frame}
\frametitle{Why do they attack?}
\begin{itemize}
\item Gain
	\begin{itemize}
	\item financial
	\item information
	\item competition
	\end{itemize}
\item To show off
\item By accident (unintentional)
\end{itemize}
\begin{block}{Impact}
Threat agents have different skills, resources and different motivations that
may have a different influence on the risk.
\end{block}
\end{frame}

%%%%%%%%%%%%%%%%%%%%%%%%%%%%%%%%%%%%%%%%%%%%%%%%%%%%%%%%%%%%%%%%%%%%%%%%%%%%%%
\subsection{Risks and rating}

\begin{frame}
\frametitle{What is a \emph{risk}?}
\begin{quote}
The potential that a given threat will exploit vulnerabilities of
an asset or group of assets and thereby cause harm to the organization. It
is measured in terms of a combination of the probability of occurrence of an
event and its consequence.
\textit{-- Wikipedia}
\end{quote}
\end{frame}

\begin{frame}
\frametitle{What is a \emph{risk}?}
\[ \text{risk} = \text{likelihood} * \text{impact} \]
\begin{itemize}
\item Likelihood: probability of a successful attack
	\\ Influenced by threat agent skills and vulnerability factors
\item Impact: how much damage the attack causes
\end{itemize}
\end{frame}

\begin{frame}
\frametitle{Vulnerability factors}
\begin{itemize}
\item ease of discovery
\item ease of exploit
\item awareness
\item intrusion detection
\end{itemize}
\end{frame}

\begin{frame}
\frametitle{Impact factors}
\begin{itemize}
\item loss of confidentiality
\item loss of integrity
\item loss of availability
\item loss of accountability
\item financial damage
\item reputation damage
\item privacy violation
\end{itemize}
\end{frame}

%%%%%%%%%%%%%%%%%%%%%%%%%%%%%%%%%%%%%%%%%%%%%%%%%%%%%%%%%%%%%%%%%%%%%%%%%%%%%%
\subsection{Context of web applications}

\subsubsection{Network}

\begin{frame}
\frametitle{Internet}
\begin{itemize}
\item Applications available on the internet
\item Anyone can access the application
\end{itemize}
\begin{block}{Anonymity}
No identifier for a web user.
\\ Example: IP address shared by multiple users
\end{block}
\end{frame}

\subsubsection{Architecture of a web application}

\begin{frame}
\frametitle{A web application}
\includeimage[width=\linewidth,height=6cm]{archi-webapplication}
%- Browser: the client, uses scripting
%- Router: connected to the browser (by WiFi) - LAN
%- Internet: not detailed here - WAN
%- Web Application Firewall, IDS, Load Balancer
%- Application Server: with the contained application
%- Database: where the application stores its data
\end{frame}

%\subsubsection{Points of failure}

\begin{frame}
\frametitle{Points of failure}
Any component may fail; will the web site continue to work after an attack?
\begin{block}{Single point of failure}
A part of a system that prevents the entire system from working when it fails.
\\ Solution: duplication
\end{block}
\end{frame}

%%%%%%%%%%%%%%%%%%%%%%%%%%%%%%%%%%%%%%%%%%%%%%%%%%%%%%%%%%%%%%%%%%%%%%%%%%%%%%
\subsection{OWASP - Top 10}

\begin{frame}
\frametitle{OWASP}
\begin{block}{OWASP}
\centering Open Web Application Security Project
\begin{itemize}
\item standards
\item libraries
\item books
\item etc.
\end{itemize}
\end{block}
\begin{block}{Project: Top 10}
OWASP compiles at top 10 of most critical web application \emph{risks}.
\end{block}
\end{frame}

\subsubsection{Attack 1 - Injection}

\begin{frame}
\frametitle{1. Injection}
Sending untrusted data to an interpreter.
Attacker enters code to execute into input fields.
\begin{block}{Impacts}
Data loss, corruption, lack of accountability, etc.
\end{block}
\begin{block}{Risks}
\alert{Easy to exploit}, has a \alert{severe impact}, it is quite common.
\end{block}
\end{frame}

\begin{frame}
\frametitle{1. Injection - Attack scenario}
\begin{itemize}
\item The user enters a sequence of characters that is invalid for the
	interpreter. The server will send an invalid result.
	Probably, the executed code or error will be sent.
\item Complete the query with what you need the interpreter to execute.
\end{itemize}
\end{frame}

\begin{frame}[fragile]
\frametitle{1. Injection - Vulnerable code}
\begin{exampleblock}{Bypassing authentication}
This password may work:
\begin{lstlisting}
' or '' = '
\end{lstlisting}
When authentication checks the user as follows:
\begin{lstlisting}
String query = "select * from users"
	+ " where user_name = '" + name + "'"
	+ " and password = '" + password + "'";
\end{lstlisting}
\end{exampleblock}
\pause
\begin{exampleblock}{SQL injection}
Getting the users' passwords as administrator.
\end{exampleblock}
\pause
\begin{exampleblock}{JPQL injection}
Searching hotels that have a manager that has the same password as a given
user.
\end{exampleblock}
\end{frame}

\begin{frame}[fragile]
\frametitle{1. Injection - Preventing the attack}
\begin{block}{Using parametrized interface}
Usage of placeholders for values let's the interpreter escape input values.
\begin{lstlisting}
String query = "select * from users where user_name = ?";
PreparedStatement st = con.prepareStatement(query);
st.setString(1, name);
ResultSet rs = st.executeQuery();
\end{lstlisting}
\end{block}
\begin{block}{Escaping routines}
Some languages have special function to escape manually values.
\end{block}
\begin{block}{White list validation}
List of valid input patterns.
\end{block}
\end{frame}

\subsubsection{Attack X - NAME}

\begin{frame}
\frametitle{1. X}
DESCRIPTION
\begin{block}{Impacts}
TODO Data loss, corruption, lack of accountability, etc.
\end{block}
\begin{block}{Risks}
TODO ease of exploitation
\end{block}
\end{frame}

\begin{frame}
\frametitle{1. X - Attack scenario}
\begin{itemize}
\item TODO
\end{itemize}
\end{frame}

\begin{frame}[fragile]
\frametitle{1. X - Vulnerable code}
\begin{exampleblock}{TODO}
\end{exampleblock}
\end{frame}

\begin{frame}[fragile]
\frametitle{1. X - Preventing the attack}
\begin{block}{TODO}
\end{block}
\end{frame}

%%%%%%%%%%%%%%%%%%%%%%%%%%%%%%%%%%%%%%%%%%%%%%%%%%%%%%%%%%%%%%%%%%%%%%%%%%%%%%
\subsection{DoS}

%%%%%%%%%%%%%%%%%%%%%%%%%%%%%%%%%%%%%%%%%%%%%%%%%%%%%%%%%%%%%%%%%%%%%%%%%%%%%%
\subsection{MitM}

%%%%%%%%%%%%%%%%%%%%%%%%%%%%%%%%%%%%%%%%%%%%%%%%%%%%%%%%%%%%%%%%%%%%%%%%%%%%%%
\subsection{Social engineering}

%%%%%%%%%%%%%%%%%%%%%%%%%%%%%%%%%%%%%%%%%%%%%%%%%%%%%%%%%%%%%%%%%%%%%%%%%%%%%%
\subsubsection{Phishing}

%%%%%%%%%%%%%%%%%%%%%%%%%%%%%%%%%%%%%%%%%%%%%%%%%%%%%%%%%%%%%%%%%%%%%%%%%%%%%%
\subsection{Other attacks}

%%%%%%%%%%%%%%%%%%%%%%%%%%%%%%%%%%%%%%%%%%%%%%%%%%%%%%%%%%%%%%%%%%%%%%%%%%%%%%
\subsection{Common points}
