\section{Attacks}

%%%%%%%%%%%%%%%%%%%%%%%%%%%%%%%%%%%%%%%%%%%%%%%%%%%%%%%%%%%%%%%%%%%%%%%%%%%%%%
\subsection{Attack trees}

\begin{frame}
\frametitle{An attack tree}

\begin{block}{What is it}
Model describing by what means an \emph{asset} can be attacked.
\end{block}
\begin{block}{Goal}
Used to determine and understand threats that may arise.
\end{block}
\end{frame}

\begin{frame}{Example}
\includeimage[width=0.8\linewidth,height=6cm]{attack-tree-admin}
\end{frame}

%%%%%%%%%%%%%%%%%%%%%%%%%%%%%%%%%%%%%%%%%%%%%%%%%%%%%%%%%%%%%%%%%%%%%%%%%%%%%%
\subsection{Threat agents}

\begin{frame}
\frametitle{Who are they?}
\begin{itemize}
\item External threats
	\begin{itemize}
	\item internet users
	\item viruses
	\item organized crime
	\end{itemize}
\item Internal threats
	\begin{itemize}
	\item employees
    \item intranet users
    \item administrators
    \end{itemize}
\item Natural threats (disasters)
\item Technical failures
	\begin{itemize}
	\item loss of essential services
		% (example: no electricity)
	\item hardware failure
	\item software failure
	\end{itemize}
\end{itemize}
\end{frame}

\begin{frame}
\frametitle{Why do they attack?}
\begin{itemize}
\item Gain
	\begin{itemize}
	\item financial
	\item information
	\item competition
	\end{itemize}
\item To show off
\item By accident (unintentional)
\end{itemize}
\begin{block}{Impact}
Threat agents have different skills, resources and different motivations that
may have a different influence on the risk.
\end{block}
\end{frame}

%%%%%%%%%%%%%%%%%%%%%%%%%%%%%%%%%%%%%%%%%%%%%%%%%%%%%%%%%%%%%%%%%%%%%%%%%%%%%%
\subsection{Risks and rating}

\begin{frame}
\frametitle{What is a \emph{risk}?}
\begin{quote}
The potential that a given threat will exploit vulnerabilities of
an asset or group of assets and thereby cause harm to the organization. It
is measured in terms of a combination of the probability of occurrence of an
event and its consequence.
\textit{-- Wikipedia}
\end{quote}
\end{frame}

\begin{frame}
\frametitle{What is a \emph{risk}?}
\[ \text{risk} = \text{likelihood} * \text{impact} \]
\begin{itemize}
\item Likelihood: probability of a successful attack
	\\ Influenced by threat agent skills and vulnerability factors
\item Impact: how much damage the attack causes
\end{itemize}
\end{frame}

\begin{frame}
\frametitle{Vulnerability factors}
\begin{itemize}
\item ease of discovery
\item ease of exploit
\item awareness
\item intrusion detection
\end{itemize}
\end{frame}

\begin{frame}
\frametitle{Impact factors}
\begin{itemize}
\item loss of confidentiality
\item loss of integrity
\item loss of availability
\item loss of accountability
\item financial damage
\item reputation damage
\item privacy violation
\end{itemize}
\end{frame}

%%%%%%%%%%%%%%%%%%%%%%%%%%%%%%%%%%%%%%%%%%%%%%%%%%%%%%%%%%%%%%%%%%%%%%%%%%%%%%
\subsection{Context of web applications}

\subsubsection{Network}

\begin{frame}
\frametitle{Internet}
\begin{itemize}
\item Applications available on the internet
\item Anyone can access the application
\end{itemize}
\begin{block}{Anonymity}
No identifier for a web user.
\\ Example: IP address shared by multiple users
\end{block}
\end{frame}

\subsubsection{Architecture of a web application}

\begin{frame}
\frametitle{A web application}
\includeimage[width=\linewidth,height=6cm]{archi-webapplication}
%- Browser: the client, uses scripting
%- Router: connected to the browser (by WiFi) - LAN
%- Internet: not detailed here - WAN
%- Web Application Firewall, IDS, Load Balancer
%- Application Server: with the contained application
%- Database: where the application stores its data
\end{frame}

%\subsubsection{Points of failure}

\begin{frame}
\frametitle{Points of failure}
Any component may fail; will the web site continue to work after an attack?
\begin{block}{Single point of failure}
A part of a system that prevents the entire system from working when it fails.
\\ Solution: duplication
\end{block}
\end{frame}

%%%%%%%%%%%%%%%%%%%%%%%%%%%%%%%%%%%%%%%%%%%%%%%%%%%%%%%%%%%%%%%%%%%%%%%%%%%%%%
\subsection{OWASP - Top 10}

\begin{frame}
\frametitle{OWASP}
\begin{block}{OWASP}
\centering Open Web Application Security Project
\begin{itemize}
\item standards
\item libraries
\item books
\item etc.
\end{itemize}
\end{block}
\begin{block}{Project: Top 10}
OWASP compiles at top 10 of most critical web application \emph{risks}.
\end{block}
\end{frame}

\begin{frame}
\frametitle{The top 10}
% TODO list the top 10
\end{frame}

\subsubsection{Attack 1 - Injection}

\begin{frame}
\frametitle{1. Injection}
\textbf{Injection} consists in sending untrusted data to an interpreter.
Attacker enters code to execute into input fields.
\begin{block}{Impacts}
Data loss, corruption, lack of accountability, etc.
\end{block}
\begin{block}{Risks}
\alert{Easy to exploit}, has a \alert{severe impact}, it is quite common.
\end{block}
\end{frame}

\begin{frame}
\frametitle{1. Injection - Attack scenario}
\begin{itemize}
\item The user enters a sequence of characters that is invalid for the
	interpreter. The server will send an invalid result.
	Probably, the executed code or error will be sent.
\item Complete the query with what you need the interpreter to execute.
\end{itemize}
\end{frame}

\begin{frame}[fragile]
\frametitle{1. Injection - Vulnerability}
\begin{exampleblock}{Bypassing authentication}
This password may work:
\begin{lstlisting}
' or '' = '
\end{lstlisting}
When authentication checks the user as follows:
\begin{lstlisting}
String query = "select * from users"
	+ " where user_name = '" + name + "'"
	+ " and password = '" + password + "'";
\end{lstlisting}
\end{exampleblock}
\pause
\begin{exampleblock}{SQL injection - demo}
Getting the users' passwords as administrator.
\end{exampleblock}
\pause
\begin{exampleblock}{JPQL injection - demo}
Searching hotels that have a manager that has the same password as a given
user.
\end{exampleblock}
\end{frame}

\begin{frame}[fragile]
\frametitle{1. Injection - Preventing the attack}
\begin{block}{Using parametrized interface}
Usage of placeholders for values let's the interpreter escape input values.
\begin{lstlisting}
String query = "select * from users where user_name = ?";
PreparedStatement st = con.prepareStatement(query);
st.setString(1, name);
ResultSet rs = st.executeQuery();
\end{lstlisting}
\end{block}
\begin{block}{Escaping routines}
Some languages have special function to escape manually values.
\end{block}
\begin{block}{White list validation}
List of valid input patterns.
\end{block}
\end{frame}

\subsubsection{Attack 2 - XSS}

\begin{frame}
\frametitle{2. XSS}
\textbf{Cross-site scripting} allows attackers to inject code into the page
sent to the user. The injected code can be a script interpreted by the
browser.
\begin{block}{Impacts}
Hijack user sessions, change content, redirect the user.
\end{block}
\begin{block}{Risks}
This is \alert{the most widespread} vulnerability.
It requires an average knowledge to be exploited and the impacts are moderate.
\end{block}
\end{frame}

\begin{frame}
\frametitle{2. XSS - Attack scenario}
\begin{itemize}
\item The attacker sends some text containing a script to the server
\item The user downloads the page with the script as it was posted by the
	attacker
\item The script is executed on the user's browser
\end{itemize}
\end{frame}

\begin{frame}[fragile]
\frametitle{2. XSS - Vulnerability}
\begin{exampleblock}{JSP page}
The description of the hotel is sent as is.
\begin{lstlisting}
<div>
	${hotel.descriptionHTML}
</div>
\end{lstlisting}
\end{exampleblock}
\pause
\begin{exampleblock}{Getting the user's cookies - demo}
Just use \lstinline!document.cookie! in JavaScript and send it somewhere.
\end{exampleblock}
\end{frame}

\begin{frame}[fragile]
\frametitle{2. XSS - Preventing the attack}
\begin{block}{Escaping the values}
All the values should be escaped before sending them to the users.
\begin{lstlisting}
<div>
	<c:out value="${hotel.descriptionHTML}" />
</div>
\end{lstlisting}
\end{block}
\begin{block}{White list validation}
Output values can be white listed, but this is not a complete defence against
XSS as sometimes special characters must be accepted.
\end{block}
\end{frame}

\subsubsection{Attack 3 - Authentication and session management}

\begin{frame}
\frametitle{3. Authentication and session management}
The attacker uses flaws in the \textbf{authentication or session management}
implemented for a given web site to steal someone else's identity.
\begin{block}{Impacts}
Once an account stolen, the attacker may do \emph{anything} the user can do.
\end{block}
\begin{block}{Risks}
The impact is \alert{severe}, this attack is common.
\end{block}
\end{frame}

\begin{frame}
\frametitle{3. Authentication and session management - Attack scenario}
\begin{itemize}
\item In older application servers the session ID could be set using the URL.
	\\ You could let someone login using your session.
\item The user could forget to log out on a public computer.
\item The attacker could gain access to the users' database.
\end{itemize}
\end{frame}

\begin{frame}
\frametitle{3. Authentication and session management - Vulnerability}
\begin{exampleblock}{XSS session hijacking - demo}
The users' sessions can be obtained using XSS.
\end{exampleblock}
\end{frame}

\begin{frame}
\frametitle{3. Authentication and session management - Preventing the attack}
\begin{block}{Good authentication mechanism}
\begin{itemize}
\item Use a proven authentication mechanism
\item Prevent \emph{XSS attacks}
\item Good session timeouts and accessible log out buttons
\end{itemize}
\end{block}
\end{frame}

\subsubsection{Attack 4 - Insecure Direct Object References}

\begin{frame}
\frametitle{4. Insecure Direct Object References}
The attacker changes a parameter in the request to obtain a \textbf{direct
object reference} that they should not be able to access.
\begin{block}{Impacts}
Compromise the data that can be referenced.
\end{block}
\begin{block}{Risks}
\alert{Easy} to exploit and detect. The impact is moderate.
\end{block}
\end{frame}

\begin{frame}
\frametitle{4. Insecure Direct Object References - Attack scenario}
\begin{itemize}
\item Find a URL with an identifier as a part of the URL
\item Replace that identifier with another value and try to send the request
\end{itemize}
\end{frame}

\begin{frame}
\frametitle{4. Insecure Direct Object References - Vulnerability}
\begin{block}{Missing access verification}
Often objects are retrieved by their ID, the developer can forget to check
whether the user is allowed to see that object.
\end{block}
\pause
\begin{exampleblock}{An unvalidated hotel - demo}
See an unvalidated hotel as a normal user.
\end{exampleblock}
\end{frame}

\begin{frame}
\frametitle{4. Insecure Direct Object References - Preventing the attack}
\begin{block}{\textbf{Check permissions}}
Check whether the user has the permissions to access the object before
manipulating it.
\end{block}
\begin{block}{References by session}
Use indirect references for each session.
\end{block}
\end{frame}

\subsubsection{Attack 5 - CSRF}

\begin{frame}
\frametitle{5. CSRF}
DESCRIPTION
\begin{block}{Impacts}
TODO Data loss, corruption, lack of accountability, etc.
\end{block}
\begin{block}{Risks}
TODO ease of exploitation
\end{block}
\end{frame}

\begin{frame}
\frametitle{5. CSRF - Attack scenario}
\begin{itemize}
\item TODO
\end{itemize}
\end{frame}

\begin{frame}[fragile]
\frametitle{5. CSRF - Vulnerability}
\begin{exampleblock}{TODO}
\end{exampleblock}
\end{frame}

\begin{frame}[fragile]
\frametitle{5. CSRF - Preventing the attack}
\begin{block}{TODO}
\end{block}
\end{frame}

\subsubsection{Attack 6 - Security Misconfiguration}

\begin{frame}
\frametitle{6. Security Misconfiguration}
DESCRIPTION
\begin{block}{Impacts}
TODO Data loss, corruption, lack of accountability, etc.
\end{block}
\begin{block}{Risks}
TODO ease of exploitation
\end{block}
\end{frame}

\begin{frame}
\frametitle{6. Security Misconfiguration - Attack scenario}
\begin{itemize}
\item TODO
\end{itemize}
\end{frame}

\begin{frame}[fragile]
\frametitle{6. Security Misconfiguration - Vulnerability}
\begin{exampleblock}{TODO}
\end{exampleblock}
\end{frame}

\begin{frame}[fragile]
\frametitle{6. Security Misconfiguration - Preventing the attack}
\begin{block}{TODO}
\end{block}
\end{frame}

\subsubsection{Attack 7 - Insecure Cryptographic Storage}

\begin{frame}
\frametitle{7. Insecure Cryptographic Storage}
DESCRIPTION
\begin{block}{Impacts}
TODO Data loss, corruption, lack of accountability, etc.
\end{block}
\begin{block}{Risks}
TODO ease of exploitation
\end{block}
\end{frame}

\begin{frame}
\frametitle{7. Insecure Cryptographic Storage - Attack scenario}
\begin{itemize}
\item TODO
\end{itemize}
\end{frame}

\begin{frame}[fragile]
\frametitle{7. Insecure Cryptographic Storage - Vulnerability}
\begin{exampleblock}{TODO}
\end{exampleblock}
\end{frame}

\begin{frame}[fragile]
\frametitle{7. Insecure Cryptographic Storage - Preventing the attack}
\begin{block}{TODO}
\end{block}
\end{frame}

\subsubsection{Attack 8 - Failure to Restrict URL Access}

\begin{frame}
\frametitle{8. Failure to Restrict URL Access}
DESCRIPTION
\begin{block}{Impacts}
TODO Data loss, corruption, lack of accountability, etc.
\end{block}
\begin{block}{Risks}
TODO ease of exploitation
\end{block}
\end{frame}

\begin{frame}
\frametitle{8. Failure to Restrict URL Access - Attack scenario}
\begin{itemize}
\item TODO
\end{itemize}
\end{frame}

\begin{frame}[fragile]
\frametitle{8. Failure to Restrict URL Access - Vulnerability}
\begin{exampleblock}{TODO}
\end{exampleblock}
\end{frame}

\begin{frame}[fragile]
\frametitle{8. Failure to Restrict URL Access - Preventing the attack}
\begin{block}{TODO}
\end{block}
\end{frame}

\subsubsection{Attack 9 - Insufficient Transport Layer Protection}

\begin{frame}
\frametitle{9. Insufficient Transport Layer Protection}
DESCRIPTION
\begin{block}{Impacts}
TODO Data loss, corruption, lack of accountability, etc.
\end{block}
\begin{block}{Risks}
TODO ease of exploitation
\end{block}
\end{frame}

\begin{frame}
\frametitle{9. Insufficient Transport Layer Protection - Attack scenario}
\begin{itemize}
\item TODO
\end{itemize}
\end{frame}

\begin{frame}[fragile]
\frametitle{9. Insufficient Transport Layer Protection - Vulnerability}
\begin{exampleblock}{TODO}
\end{exampleblock}
\end{frame}

\begin{frame}[fragile]
\frametitle{9. Insufficient Transport Layer Protection - Preventing the attack}
\begin{block}{TODO}
\end{block}
\end{frame}

\subsubsection{Attack 10 - Unvalidated Redirects}

\begin{frame}
\frametitle{10. Unvalidated Redirects}
DESCRIPTION
\begin{block}{Impacts}
TODO Data loss, corruption, lack of accountability, etc.
\end{block}
\begin{block}{Risks}
TODO ease of exploitation
\end{block}
\end{frame}

\begin{frame}
\frametitle{10. Unvalidated Redirects - Attack scenario}
\begin{itemize}
\item TODO
\end{itemize}
\end{frame}

\begin{frame}[fragile]
\frametitle{10. Unvalidated Redirects - Vulnerability}
\begin{exampleblock}{TODO}
\end{exampleblock}
\end{frame}

\begin{frame}[fragile]
\frametitle{10. Unvalidated Redirects - Preventing the attack}
\begin{block}{TODO}
\end{block}
\end{frame}

%%%%%%%%%%%%%%%%%%%%%%%%%%%%%%%%%%%%%%%%%%%%%%%%%%%%%%%%%%%%%%%%%%%%%%%%%%%%%%
\subsection{DoS}

%%%%%%%%%%%%%%%%%%%%%%%%%%%%%%%%%%%%%%%%%%%%%%%%%%%%%%%%%%%%%%%%%%%%%%%%%%%%%%
\subsection{MitM}

%%%%%%%%%%%%%%%%%%%%%%%%%%%%%%%%%%%%%%%%%%%%%%%%%%%%%%%%%%%%%%%%%%%%%%%%%%%%%%
\subsection{Social engineering}

%%%%%%%%%%%%%%%%%%%%%%%%%%%%%%%%%%%%%%%%%%%%%%%%%%%%%%%%%%%%%%%%%%%%%%%%%%%%%%
\subsubsection{Phishing}

%%%%%%%%%%%%%%%%%%%%%%%%%%%%%%%%%%%%%%%%%%%%%%%%%%%%%%%%%%%%%%%%%%%%%%%%%%%%%%
\subsection{Other attacks}

%%%%%%%%%%%%%%%%%%%%%%%%%%%%%%%%%%%%%%%%%%%%%%%%%%%%%%%%%%%%%%%%%%%%%%%%%%%%%%
\subsection{Common points}
