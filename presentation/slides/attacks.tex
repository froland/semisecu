\section{Attacks}

%%%%%%%%%%%%%%%%%%%%%%%%%%%%%%%%%%%%%%%%%%%%%%%%%%%%%%%%%%%%%%%%%%%%%%%%%%%%%%
\subsection{Attack trees}

\begin{frame}
\frametitle{An attack tree}

\begin{block}{What is it}
Model describing by what means an \emph{asset} can be attacked.
\end{block}
\begin{block}{Goal}
Used to determine and understand threats that may arise.
\end{block}
\end{frame}

\begin{frame}{Example}
\includeimage[width=0.8\linewidth,height=6cm]{attack-tree-admin}
\end{frame}

%%%%%%%%%%%%%%%%%%%%%%%%%%%%%%%%%%%%%%%%%%%%%%%%%%%%%%%%%%%%%%%%%%%%%%%%%%%%%%
\subsection{Threat agents}

\begin{frame}
\frametitle{Who are they?}
\begin{itemize}
\item External threats
	\begin{itemize}
	\item internet users
	\item viruses
	\item organized crime
	\end{itemize}
\item Internal threats
	\begin{itemize}
	\item employees
    \item intranet users
    \item administrators
    \end{itemize}
\item Natural threats (disasters)
\item Technical failures
	\begin{itemize}
	\item loss of essential services
		% (example: no electricity)
	\item hardware failure
	\item software failure
	\end{itemize}
\end{itemize}
\end{frame}

\begin{frame}
\frametitle{Why do they attack?}
\begin{itemize}
\item Gain
	\begin{itemize}
	\item financial
	\item information
	\item competition
	\end{itemize}
\item To show off
\item By accident (unintentional)
\end{itemize}
\begin{block}{Impact}
Threat agents have different skills, resources and different motivations that
may have a different influence on the risk.
\end{block}
\end{frame}

%%%%%%%%%%%%%%%%%%%%%%%%%%%%%%%%%%%%%%%%%%%%%%%%%%%%%%%%%%%%%%%%%%%%%%%%%%%%%%
\subsection{Risks and rating}

\begin{frame}
\frametitle{What is a \emph{risk}?}
\begin{quote}
The potential that a given threat will exploit vulnerabilities of
an asset or group of assets and thereby cause harm to the organization. It
is measured in terms of a combination of the probability of occurrence of an
event and its consequence.
\textit{-- Wikipedia}
\end{quote}
\end{frame}

\begin{frame}
\frametitle{What is a \emph{risk}?}
\[ \text{risk} = \text{likelihood} * \text{impact} \]
\begin{itemize}
\item Likelihood: probability of a successful attack
	\\ Influenced by threat agent skills and vulnerability factors
\item Impact: how much damage the attack causes
\end{itemize}
\end{frame}

\begin{frame}
\frametitle{Vulnerability factors}
\begin{itemize}
\item ease of discovery
\item ease of exploit
\item awareness
\item intrusion detection
\end{itemize}
\end{frame}

\begin{frame}
\frametitle{Impact factors}
\begin{itemize}
\item loss of confidentiality
\item loss of integrity
\item loss of availability
\item loss of accountability
\item financial damage
\item reputation damage
\item privacy violation
\end{itemize}
\end{frame}

%%%%%%%%%%%%%%%%%%%%%%%%%%%%%%%%%%%%%%%%%%%%%%%%%%%%%%%%%%%%%%%%%%%%%%%%%%%%%%
\subsection{Context of web applications}

\subsubsection{Network}

\begin{frame}
\frametitle{Internet}
\begin{itemize}
\item Applications available on the internet
\item Anyone can access the application
\end{itemize}
\begin{block}{Anonymity}
No identifier for a web user.
\\ Example: IP address shared by multiple users
\end{block}
\end{frame}

\subsubsection{Architecture of a web application}

\begin{frame}
\frametitle{A web application}
\includeimage[width=\linewidth,height=6cm]{archi-webapplication}
%- Browser: the client, uses scripting
%- Router: connected to the browser (by WiFi) - LAN
%- Internet: not detailed here - WAN
%- Web Application Firewall, IDS, Load Balancer
%- Application Server: with the contained application
%- Database: where the application stores its data
\end{frame}

%\subsubsection{Points of failure}

\begin{frame}
\frametitle{Points of failure}
Any component may fail; will the web site continue to work after an attack?
\begin{block}{Single point of failure}
A part of a system that prevents the entire system from working when it fails.
\\ Solution: duplication
\end{block}
\end{frame}

%%%%%%%%%%%%%%%%%%%%%%%%%%%%%%%%%%%%%%%%%%%%%%%%%%%%%%%%%%%%%%%%%%%%%%%%%%%%%%
\subsection{OWASP - Top 10}

\begin{frame}
\frametitle{OWASP}
\begin{block}{OWASP}
\centering Open Web Application Security Project
\begin{itemize}
\item standards
\item libraries
\item books
\item etc.
\end{itemize}
\end{block}
\begin{block}{Project: Top 10}
OWASP compiles at top 10 of most critical web application \emph{risks}.
\end{block}
\end{frame}

\begin{frame}
\frametitle{The top 10}
% TODO list the top 10
\end{frame}

\subsubsection{Attack 1 - Injection}

\begin{frame}
\frametitle{1. Injection}
\textbf{Injection} consists in sending untrusted data to an interpreter.
Attacker enters code to execute into input fields.
\begin{block}{Impacts}
Data loss, corruption, lack of accountability, etc.
\end{block}
\begin{block}{Risks}
\alert{Easy to exploit}, has a \alert{severe impact}, it is quite common.
\end{block}
\end{frame}

\begin{frame}
\frametitle{1. Injection - Attack scenario}
\begin{itemize}
\item The user enters a sequence of characters that is invalid for the
	interpreter. The server will send an invalid result.
	Probably, the executed code or error will be sent.
\item Complete the query with what you need the interpreter to execute.
\end{itemize}
\end{frame}

\begin{frame}[fragile]
\frametitle{1. Injection - Vulnerability}
\begin{exampleblock}{Bypassing authentication}
This password may work:
\begin{lstlisting}
' or '' = '
\end{lstlisting}
When authentication checks the user as follows:
\begin{lstlisting}
String query = "select * from users"
	+ " where user_name = '" + name + "'"
	+ " and password = '" + password + "'";
\end{lstlisting}
\end{exampleblock}
\pause
\begin{exampleblock}{SQL injection - demo}
Getting the users' passwords as administrator.
\end{exampleblock}
\pause
\begin{exampleblock}{JPQL injection - demo}
Searching hotels that have a manager that has the same password as a given
user.
\end{exampleblock}
\end{frame}

\begin{frame}[fragile]
\frametitle{1. Injection - Preventing the attack}
\begin{block}{Using parametrized interface}
Usage of placeholders for values let's the interpreter escape input values.
\begin{lstlisting}
String query = "select * from users where user_name = ?";
PreparedStatement st = con.prepareStatement(query);
st.setString(1, name);
ResultSet rs = st.executeQuery();
\end{lstlisting}
\end{block}
\begin{block}{Escaping routines}
Some languages have special function to escape manually values.
\end{block}
\begin{block}{White list validation}
List of valid input patterns.
\end{block}
\end{frame}

\subsubsection{Attack 2 - XSS}

\begin{frame}
\frametitle{2. XSS}
\textbf{Cross-site scripting} allows attackers to inject code into the page
sent to the user. The injected code can be a script interpreted by the
browser.
\begin{block}{Impacts}
Hijack user sessions, change content, redirect the user.
\end{block}
\begin{block}{Risks}
This is \alert{the most widespread} vulnerability.
It requires an average knowledge to be exploited and the impacts are moderate.
\end{block}
\end{frame}

\begin{frame}
\frametitle{2. XSS - Attack scenario}
\begin{itemize}
\item The attacker sends some text containing a script to the server
\item The user downloads the page with the script as it was posted by the
	attacker
\item The script is executed on the user's browser
\end{itemize}
\end{frame}

\begin{frame}[fragile]
\frametitle{2. XSS - Vulnerability}
\begin{exampleblock}{JSP page}
The description of the hotel is sent as is.
\begin{lstlisting}
<div>
	${hotel.descriptionHTML}
</div>
\end{lstlisting}
\end{exampleblock}
\pause
\begin{exampleblock}{Getting the user's cookies - demo}
Just use \lstinline!document.cookie! in JavaScript and send it somewhere.
\end{exampleblock}
\end{frame}

\begin{frame}[fragile]
\frametitle{2. XSS - Preventing the attack}
\begin{block}{Escaping the values}
All the values should be escaped before sending them to the users.
\begin{lstlisting}
<div>
	<c:out value="${hotel.descriptionHTML}" />
</div>
\end{lstlisting}
\end{block}
\begin{block}{White list validation}
Output values can be white listed, but this is not a complete defence against
XSS as sometimes special characters must be accepted.
\end{block}
\end{frame}

\subsubsection{Attack 3 - Authentication and session management}

\begin{frame}
\frametitle{3. Authentication and session management}
The attacker uses flaws in the \textbf{authentication or session management}
implemented for a given web site to steal someone else's identity.
\begin{block}{Impacts}
Once an account stolen, the attacker may do \emph{anything} the user can do.
\end{block}
\begin{block}{Risks}
The impact is \alert{severe}, this attack is common.
\end{block}
\end{frame}

\begin{frame}
\frametitle{3. Authentication and session management - Attack scenario}
\begin{itemize}
\item In older application servers the session ID could be set using the URL.
	\\ You could let someone login using your session.
\item The user could forget to log out on a public computer.
\item The attacker could gain access to the users' database.
\end{itemize}
\end{frame}

\begin{frame}
\frametitle{3. Authentication and session management - Vulnerability}
\begin{exampleblock}{XSS session hijacking - demo}
The users' sessions can be obtained using XSS.
\end{exampleblock}
\end{frame}

\begin{frame}
\frametitle{3. Authentication and session management - Preventing the attack}
\begin{block}{Good authentication mechanism}
\begin{itemize}
\item Use a proven authentication mechanism
\item Prevent \emph{XSS attacks}
\item Good session timeouts and accessible log out buttons
\end{itemize}
\end{block}
\end{frame}

\subsubsection{Attack 4 - Insecure Direct Object References}

\begin{frame}
\frametitle{4. Insecure Direct Object References}
The attacker changes a parameter in the request to obtain a \textbf{direct
object reference} that they should not be able to access.
\begin{block}{Impacts}
Compromise the data that can be referenced.
\end{block}
\begin{block}{Risks}
\alert{Easy} to exploit and detect. The impact is moderate.
\end{block}
\end{frame}

\begin{frame}
\frametitle{4. Insecure Direct Object References - Attack scenario}
\begin{itemize}
\item Find a URL with an identifier as a part of the URL
\item Replace that identifier with another value and try to send the request
\end{itemize}
\end{frame}

\begin{frame}
\frametitle{4. Insecure Direct Object References - Vulnerability}
\begin{block}{Missing access verification}
Often objects are retrieved by their ID, the developer can forget to check
whether the user is allowed to see that object.
\end{block}
\pause
\begin{exampleblock}{An unvalidated hotel - demo}
See an unvalidated hotel as a normal user.
\end{exampleblock}
\end{frame}

\begin{frame}
\frametitle{4. Insecure Direct Object References - Preventing the attack}
\begin{block}{\textbf{Check permissions}}
Check whether the user has the permissions to access the object before
manipulating it.
\end{block}
\begin{block}{References by session}
Use indirect references for each session.
\end{block}
\end{frame}

\subsubsection{Attack 5 - CSRF}

\begin{frame}
\frametitle{5. CSRF}
\textbf{Cross-site Request Forgery} consists in generating request when the
user visits the attacker's web page.
\begin{block}{Impacts}
The attacker may perform any action the victim is authorized to do.
\end{block}
\begin{block}{Risks}
\alert{Widespread} and easy to detect, the impact is moderate.
\end{block}
\end{frame}

\begin{frame}
\frametitle{5. CSRF - Attack scenario}
\begin{itemize}
\item A user visits the attacker's web page
\item The requested page contains tricks to submit requests to the attacked
	web site as the visiting user.
\end{itemize}
\end{frame}

\begin{frame}[fragile]
\frametitle{5. CSRF - Vulnerability}
\begin{exampleblock}{Post a comment}
The attacker's page may contain:
\begin{lstlisting}
<img src="http://example.com/forum/thread1/addComment?
text=hello%20world" alt="" />
\end{lstlisting}
\end{exampleblock}
\end{frame}

\begin{frame}[fragile]
\frametitle{5. CSRF - Preventing the attack}
\begin{block}{Unpredictable token}
When a each request contains a field that is a token linked to the session of
the user, the attacker's website will not be able to generate the correct
requests.
Tokens can be unique by session or by request.
\end{block}
\end{frame}

\subsubsection{Attack 6 - Security Misconfiguration}

\begin{frame}
\frametitle{6. Security Misconfiguration}
\textbf{Security configuration} can happen at any level: the application, the
web server, the framework, etc.
The default behavior is not always the wanted one.
\begin{block}{Impacts}
Attackers can gain unauthorized access, the system can be compromised.
\end{block}
\begin{block}{Risks}
\alert{Easy} to exploit and to detect.
\end{block}
\end{frame}

\begin{frame}
\frametitle{6. Security Misconfiguration - Attack scenario}
\begin{itemize}
\item The attacker tries
	\begin{itemize}
	\item default accounts
	\item unused pages
	\item unpatched flaws
	\item find unprotected files and directories
	\end{itemize}
\item Gain information on the system
	\begin{itemize}
	\item version of used libraries/frameworks
	\item server version
	\item stack traces and other debug messages
	\end{itemize}
\end{itemize}
\end{frame}

\begin{frame}
\frametitle{6. Security Misconfiguration - Vulnerability}
\begin{block}{Outdated libraries}
Bugs are published for known libraries and applications.
\end{block}
\begin{exampleblock}{Stack trace}
A stack trace may contain the used parameters, references to libraries, etc.
\end{exampleblock}
\end{frame}

\begin{frame}
\frametitle{6. Security Misconfiguration - Preventing the attack}
\begin{block}{Up-to-date system}
Keep the used software up to date and manage versions.
\end{block}
\begin{block}{Good architecture}
Good separation between components is more secure.
\end{block}
\begin{block}{Automated scan}
Automated scans can detect misconfigurations and missing patches.
\end{block}
\end{frame}

\subsubsection{Attack 7 - Insecure Cryptographic Storage}

\begin{frame}
\frametitle{7. Insecure Cryptographic Storage}
Sensitive data is stored \textbf{insecurely without crypting}.
\begin{block}{Impacts}
Stolen data is compromised.
\end{block}
\begin{block}{Risks}
All unencrypted \alert{data} is compromised.
The attack is however difficult to perform.
\end{block}
\end{frame}

\begin{frame}
\frametitle{7. Insecure Cryptographic Storage - Attack scenario}
\begin{itemize}
\item The attacker gains access to the database and dumps tables
\item All the information is available to the attacker
\end{itemize}
\end{frame}

\begin{frame}
\frametitle{7. Insecure Cryptographic Storage - Vulnerability}
\begin{exampleblock}{Unencrypted passwords}
After dumping the table \texttt{USERS}, the passwords might be stored in
plain text.
\end{exampleblock}
\begin{exampleblock}{Unencrypted backup}
The database backup is lost.
\end{exampleblock}
\begin{exampleblock}{Rainbow tables}
Precomputed table for decoding hashed values.
\end{exampleblock}
\end{frame}

\begin{frame}
\frametitle{7. Insecure Cryptographic Storage - Preventing the attack}
\begin{block}{Encrypting}
Encrypt the data such as the considered threats will not be able to read it.
\end{block}
\begin{block}{Hash + Salt}
Ensure passwords are hashed with a standard algorithm and an appropriate salt
is used.
% TODO explain hash and salt
\end{block}
\end{frame}

\subsubsection{Attack 8 - Failure to Restrict URL Access}

\begin{frame}
\frametitle{8. Failure to Restrict URL Access}
\textbf{URL access is not restricted} to functions requiring permissions.
\begin{block}{Impacts}
Administrative functions may become available to the attacker.
\end{block}
\begin{block}{Risks}
\alert{Easy} to exploit, but the flaw is uncommon.
\end{block}
\end{frame}

\begin{frame}
\frametitle{8. Failure to Restrict URL Access - Attack scenario}
\begin{itemize}
\item Attacker changes his URL to a privileged page
\item No authentication is required
\end{itemize}
\end{frame}

\begin{frame}
\frametitle{8. Failure to Restrict URL Access - Vulnerability}
\begin{exampleblock}{Administration page}
\url{http://example.com/admin} is not referenced from any other page, but
the authentication is not enforced.
\end{exampleblock}
\end{frame}

\begin{frame}
\frametitle{8. Failure to Restrict URL Access - Preventing the attack}
\begin{block}{Check permissions}
Authorization mechanisms should be used.
\end{block}
\end{frame}

\subsubsection{Attack 9 - Insufficient Transport Layer Protection}

\begin{frame}
\frametitle{9. Insufficient Transport Layer Protection}
\textbf{Traffic is unprotected} and anyone sniffing the network can see the
data in transit.
\begin{block}{Impacts}
Account theft, expose individual user's data.
\end{block}
\begin{block}{Risks}
\alert{Detectable} using any sniffer.
However, often difficult to exploit.
\end{block}
\end{frame}

\begin{frame}
\frametitle{9. Insufficient Transport Layer Protection - Attack scenario}
\begin{itemize}
\item The attacker sniffs the WiFi network
\item Scan for interesting HTTP requests
\item Analyze the data sent between the client and the server
\end{itemize}
\end{frame}

\begin{frame}
\frametitle{9. Insufficient Transport Layer Protection - Vulnerability}
\begin{exampleblock}{Sniffing passwords}
Sniff the network and find HTTP requests containing "password".
\end{exampleblock}
\end{frame}

\begin{frame}
\frametitle{9. Insufficient Transport Layer Protection - Preventing the attack}
\begin{block}{HTTP\textbf{S}}
SSL/TLS can be used to secure the connection.
\begin{itemize}
\item \emph{secure} cookies will not be sent through an unencrypted connection
% TODO explain cookie
\item ensure that certificates are valid
\end{itemize}
\end{block}
\end{frame}

\subsubsection{Attack 10 - Unvalidated Redirects}

\begin{frame}
\frametitle{10. Unvalidated Redirects}
A URL may appear as trustworthy, but it contains an \textbf{unvalidated
redirect}.
\begin{block}{Impacts}
Tricking victims into disclosing information, unsafe forwards may allow access
control bypass.
\end{block}
\begin{block}{Risks}
Easy to detect, uncommonly present.
\end{block}
\end{frame}

\begin{frame}
\frametitle{10. Unvalidated Redirects - Attack scenario}
\begin{itemize}
\item The attacker places a URL with a redirect that seems trustworthy,
	however it redirects to a malicious page
\item The user clicks on that link and logs in
\end{itemize}
\end{frame}

\begin{frame}
\frametitle{10. Unvalidated Redirects - Vulnerability}
\begin{exampleblock}{Malicious site}
Redirect to a malicious website using:
\\ \url{http://bank.com/redirect?url=evil-bank.com}
\end{exampleblock}
\begin{exampleblock}{Bypass authentication}
\url{http://example.com/redirect.jsp?fwd=admin.jsp}
\\ The attacker may bypass authorization for \texttt{admin.jsp}.
\end{exampleblock}
\end{frame}

\begin{frame}
\frametitle{10. Unvalidated Redirects - Preventing the attack}
\begin{block}{Avoid redirects and forwards}
Avoid redirects and forwards and do not include parameters in them.
\end{block}
\begin{block}{Authorization}
Validate and authorize the user after each redirect.
\end{block}
\begin{exampleblock}{Map input values}
Map input values to the actual URLs.
\end{exampleblock}
\end{frame}

%%%%%%%%%%%%%%%%%%%%%%%%%%%%%%%%%%%%%%%%%%%%%%%%%%%%%%%%%%%%%%%%%%%%%%%%%%%%%%
\subsection{DoS}

\begin{frame}
\frametitle{Denial of Service (DoS) attack}
Attack that aims to bring down (or make unavailable) some service usually by
causing server overload.
\begin{itemize}
\item SYN flood
\item Application level flood
\item R-U-Dead-Yet
	(never-ending POST transmissions,
	example: \href{http://ha.ckers.org/slowloris/}{Slowloris})
\item DDoS
\item Unintentional DoS
\end{itemize}
\end{frame}

\begin{frame}
\frametitle{Denial of Service - example}
\begin{exampleblock}{DDoS}
The anonymous attack.
\end{exampleblock}
\begin{exampleblock}{Unintentional DoS}
\begin{itemize}
\item page linked from \url{http://slashdot.org}
\item \href{http://www.joueurdugrenier.fr/}{Joueur du Grenier} - new video
\end{itemize}
\end{exampleblock}
\end{frame}

\begin{frame}
\frametitle{Denial of Service - prevention}
\begin{itemize}
\item Firewall
\item Intrusion Prevention System
\item Black-holing
\end{itemize}
\end{frame}

%%%%%%%%%%%%%%%%%%%%%%%%%%%%%%%%%%%%%%%%%%%%%%%%%%%%%%%%%%%%%%%%%%%%%%%%%%%%%%
\subsection{MitM}

\begin{frame}
\frametitle{Man in the Middle attack}
The attacker plays the role of the server to the client and the role of the
client to the server.
All communication transits through the attacker who can modify messages as
they are sent.
\includeimage[height=0.4\textheight]{Man_in_the_Middle}
\end{frame}

\begin{frame}
\frametitle{Man in the Middle - example and prevention}
\begin{exampleblock}{Public WiFi}
Vulnerable on public WiFi networks.
\\ Tools such as \texttt{ettercap} allow to perform the attack.
\end{exampleblock}
\begin{block}{Prevention}
Use cryptography: HTTPS, PKI (Public Key Infrastructure)
\end{block}
% TODO explain PKI
\end{frame}

%%%%%%%%%%%%%%%%%%%%%%%%%%%%%%%%%%%%%%%%%%%%%%%%%%%%%%%%%%%%%%%%%%%%%%%%%%%%%%
\subsection{Social engineering}

\begin{frame}
\frametitle{Social engineering}
\textbf{Social engineering} consists in manipulating people so they divulge
confidential information.
\\ Similar to fraud.
\pause
\includeimage[width=0.6\linewidth]{social-engineering}
\end{frame}

\begin{frame}
\frametitle{Social engineering - example and prevention}
\begin{itemize}
\item pretexting
	% Using an invented scenario (the pretext) to engage a targeted victim in
	% a manner that increases the chance the victim will divulge information
	% or perform actions that would be unlikely in ordinary circumstances.
\item phishing \small{(next point)}
\item quid pro quo
	% Something for something:
	% Help a user solve a problem, they might give you passwords in exchange.
\end{itemize}
\begin{exampleblock}{Survey}
\begin{quote}
In a 2003 information security survey, 90\% of office workers gave
researchers what they claimed was their password in answer to a survey
question in exchange for a cheap pen.
\end{quote}
\end{exampleblock}
\begin{block}{Protection}
\begin{itemize}
\item education
\item framework of trust
\end{itemize}
\end{block}
\end{frame}

%%%%%%%%%%%%%%%%%%%%%%%%%%%%%%%%%%%%%%%%%%%%%%%%%%%%%%%%%%%%%%%%%%%%%%%%%%%%%%
\subsubsection{Phishing}

\begin{frame}
\frametitle{Phishing}
\textbf{Phishing} is the attempt to acquire user information through
masquerading as a legitimate web site.
\begin{itemize}
\item cloned websites
\item link manipulation
\item phone phishing
\item etc.
\end{itemize}
\end{frame}

\begin{frame}
\frametitle{Phishing - prevention}
\begin{itemize}
\item education
\item help to identify legitimate web sites
\end{itemize}
\includeimage[height=0.6\textheight]{phishing-facebook}
\end{frame}

%%%%%%%%%%%%%%%%%%%%%%%%%%%%%%%%%%%%%%%%%%%%%%%%%%%%%%%%%%%%%%%%%%%%%%%%%%%%%%
\subsection{Other attacks}

\begin{frame}
\frametitle{Brute force}
Let's try \textbf{all} the passwords.
\begin{block}{Try all}
Try all the possible combinations.
\newline
$k^n$ where $k$ is the size of the alphabet and $n$ the length of the
password.
\end{block}
\begin{block}{Dictionary attack}
Try a list of common passwords.
\\ English language contains only hundreds of thousands words.
\end{block}
\end{frame}

% XXX not treated, not much to write about it
%\begin{frame}
%\frametitle{Sequence prediction}
%Predicting the next session ID.
%\end{frame}

\begin{frame}
\frametitle{Path traversal}
An input is a relative unvalidated path.
You can download or upload any file that the server contains.
\begin{exampleblock}{Download form}
Get a file uploaded to a forum... or the password file.
\newline
\url{http://example.com/forum/download.jsp?path=/etc/shadow}
\end{exampleblock}
\end{frame}

\begin{frame}
\frametitle{Failure to restrict automation}
The web site is not protected against automation.
\begin{itemize}
\item Common for REST API
\item Programs may overload the website unintentionally
\end{itemize}
\end{frame}

\begin{frame}
\frametitle{And so on...}
As technology evolves, new forms of attacks appear.
Instead of trying to fix vulnerabilities, focus on establishing strong
security controls.
\end{frame}

%%%%%%%%%%%%%%%%%%%%%%%%%%%%%%%%%%%%%%%%%%%%%%%%%%%%%%%%%%%%%%%%%%%%%%%%%%%%%%
\subsection{Common points}

\begin{frame}
\frametitle{Common points}
\begin{itemize}
\item improper input validation
\item missing authorization checks
\item misconfiguration
\item other assumptions
\item education factor \small{(social engineering)}
\end{itemize}
\end{frame}

